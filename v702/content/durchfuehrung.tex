\section{Durchführung}
\label{sec:Durchführung}
Für die Durchführung dieses Versuches werden zwei Proben instabiler Isotope benötigt. Dieses Versuchsprotokoll befasst sich mit den Isotopen $\ce{^{52}_{23}V}$ und 
$\ce{^{104}_{45}}$. Beide Isotope müssen, bevor der Verusch durchgeführt werden kann, durch Neutronenbestrahlung \textit{aktiviert} werden. Dies geschieht in einem dafür
konstruierten Behälter. $\ce{^{104}_{45}}$ muss sich für mindestens $\qty{20}{\minute}$ in diesem Behälter befinden, bevor damit eine Messung durchgeführt werden kann. 
$\ce{^{52}_{23}V}$ muss lediglich $\qty{15}{\minute}$ reaktiviert werden. Sind die Proben einsatzbereit werden diese in die Messapparatur eingesetzt, welche in
\autoref{fig:Messapparatur} dargestellt wird.  

\begin{figure}
    \centering
    \includegraphics[width = .7\textwidth]{content/Skizzeapparatur.png}
    \caption{In dieser Abbildung ist der Aufbau der verwendeten Messapparatur skizziert. \cite{v702}}
    \label{fig:Messapparatur}
\end{figure}

Die Messung wird nun mit dem Rhodium-Isotrop begonnen. Dazu wird die Probe eingeführt und an dem \textit{Zählgerät} wird ein Messzeitintervall $\symup{\Delta}t = \qty{15}{\second}$
eingestellt. Das Zählgerät besitzt zwei Zähler. Nach dem Durchlauf eines Messzeitintervalls schlägt der eingebaute Schalter um und der andere Zähler läuft weiter. Durch dieses 
Verfahren werden für Rhodium