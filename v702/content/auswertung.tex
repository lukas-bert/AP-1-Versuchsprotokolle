\section{Auswertung}
\label{sec:Auswertung}
Um aus den aufgenommenen Zerfallskurven die Halbwertszeiten bestimmen zu können, bietet es sich an, den Logarithmus der Messwerte gegen die Zeit aufzutragen.
Durch dieses Vorgehen kann \autoref{eqn:N_dt} als lineare Funktion 
\begin{equation}
  \label{eqn:linear}
  \symup{log}\left(N_{\symup{\Delta}t}(t) \right) = \symup{log}\left(N_0 \left(1- \symup{e}^{-\lambda \symup{\Delta}t}\right)\right) - \lambda t := f(t) = at + b
\end{equation}
dargestellt werden, womit sich durch lineare Ausgleichsrechnung die Zerfallskonstante $\lambda$ bestimmen lässt.
Vor den Messungen der Zerfallskurven wird der Nulleffekt, also der durch äußere Einflüsse hervorgerufene Hintergrund der Zählraten, ermittelt. Dazu werden $\num{340}$
Impulse in einer Zeit von $\qty{600}{\second}$ gemessen. Da die Zählraten eines Geiger-Müller-Zählrohres einer Poissonverteilung folgen, ergibt sich zu jeder Zählrate 
$N$ eine Unsicherheit von $\symup{\Delta}N = \sqrt{N}$. Damit folgt für den Nulleffekt pro Sekunde $N_0 = \qty{0.567+-0.031}{\second^{-1}}$.

\subsection{Bestimmung der Halbwertszeit von Vanadium}
\label{subsec:A_Vanadium}
Zur Bestimmung der Halbwertszeit von Vanadium werden die Messwerte aus \autoref{tab:Vanadium} verwendet. Die Länge der Messintervalle beträgt 
$\symup{\Delta}t = \qty{30}{\second}$. Vor der Ausgleichsrechnung zur Bestimmung der Zerfallskonstante wird der Nulleffekt für $t = \qty{30}{\second}$ subtrahiert.
Da sich dabei aufgrund von statistischen Schwankungen negative Werte ergeben, die sich jedoch nicht für die weitere Rechnung eignen, werden nur die neuen Zählraten
mit einem Wert über $0$ betrachtet.

\begin{table} [H]
  \caption{Messwerte der Zählraten für Vanadium inklusive der jeweiligen Fehler.}
  \label{tab:Vanadium}
  \centering
  \begin{tabular}{S[table-format = 3.0] S[table-format = 2.0] @{${}\pm{}$} S}
    \toprule
    {Messzeit $t/\unit{\second}$} & \multicolumn{2}{c}{Zählrate $N$} \\
    \midrule
     30 & 76 &  9 \\
     60 & 94 & 10 \\
     90 & 72 &  8 \\
    120 & 70 &  8 \\
    150 & 62 &  8 \\
    180 & 66 &  8 \\
    210 & 66 &  8 \\
    240 & 53 &  7 \\
    270 & 46 &  7 \\
    300 & 49 &  7 \\
    330 & 56 &  7 \\
    360 & 45 &  7 \\
    390 & 36 &  6 \\
    420 & 31 &  6 \\
    450 & 30 &  5 \\
    \bottomrule
  \end{tabular}
  \begin{tabular}{S[table-format = 3.0] S[table-format = 2.0] @{${}\pm{}$} S}
    \toprule
    {Messzeit $t/\unit{\second}$} & \multicolumn{2}{c}{Zählrate $N$} \\
    \midrule
    480 & 19 & 4 \\
    510 & 25 & 5 \\
    540 & 31 & 6 \\
    570 & 23 & 5 \\
    600 & 25 & 5 \\
    630 & 23 & 5 \\
    660 & 21 & 5 \\
    690 & 19 & 4 \\
    720 & 18 & 4 \\
    750 & 23 & 5 \\
    780 & 16 & 4 \\
    810 & 18 & 4 \\
    840 & 21 & 5 \\
    870 & 19 & 4 \\
    900 & 12 & 3 \\
    \bottomrule
  \end{tabular}
\end{table}

Durch Auftragen des Logarithmus der Zählraten pro Zeitintervall $\symup{\Delta}t$ gegen die Zeit $t$ entsteht das Diagramm \ref{fig:Vanadium}. Die Messwerte modellieren einen 
annähernd linearen Verlauf. Mithilfe des Ansatzes \eqref{eqn:linear} wird eine lineare Regression mittels \textit{scipy} \cite{scipy} durchgeführt.
Es ergeben sich die Parameter
\begin{align*}
  a &= \qty{-0.0048+-0.0004}{\second^{-1}} & b &= \num{ 4.49+-0.20}
\end{align*}
der Ausgleichsgeraden $f(t) = at + b$. Aus \autoref{eqn:linear} folgt sofort $\lambda_\text{V} = -a = \qty{0.0048+-0.0004}{\second^{-1}}$. Die Halbwertszeit kann nun gemäß 
\autoref{eqn:Halbwertszeit} berechnet werden. Dabei ergibt sich durch die Gauß'sche Fehlerfortpflanzung eine Unsicherheit von 
\begin{equation}
  \label{eqn:delta_T}
  \symup{\Delta}T = \frac{\ln(2)}{\lambda^2} \cdot \symup{\Delta} \lambda. 
\end{equation}
Es folgt der Wert $T_\text{V} = \qty{144+-12}{\second}$ für die Halbwertszeit von Vanadium. 

\begin{figure}
  \centering
  \includegraphics{Vanadium.pdf}
  \caption{Logarithmus der Zählraten zu Vanadium und lineare Regression der Messwerte. Erstellt mit \textit{matplotlib} \cite{matplotlib}.}
  \label{fig:Vanadium}
\end{figure}

\subsection{Bestimmung der Zerfallseigenschaften von Rhodium}
\label{subsec:A_Rhodium}

Nun wird der Zerfall des Rhodium-Isotopes ausgewertet. Die Messdaten zu diesem Isotop werden in \autoref{tab:Rhodium} dargestellt. Die Abweichungen der Zählwerte
verteilen sich gemäß einer Poissonverteilung. Da nur ganzzahlige Zählstände möglich sind werden die Abweichungen auf ganze Zahlen gerundet. Das verwendete Rhodium-Isotop kann 
den instabilen Zustand über zwei unterschiedliche Zerfälle verlassen, abhängig davon welches Isotop bei der Kernreaktion mit einem Neutron entstanden ist. Die Produkte und
weiteren Zerfälle wurden bereits im \autoref{subsec:Zerfall} diskutiert.
An dieser Stelle ist es lediglich relevant, dass  $\ce{^{104i}_{45}Rh}$ wesentlich schneller zerfällt, im Vergleich zum langsamen Zerfall des $\ce{^{104}_{45}Rh}$.
Da die Zählung der Impulse aber nicht zwischen den Zerfällen unterscheidet, muss diese dementsprechend ausgewertet werden. Daher wird zunächst nur der 
Zerfall des langlebigeren $\ce{^{104}_{45}Rh}$-Isotopes untersucht.

\subsection{Zerfallsprozess von Rhodium-104}
\label{subsec:Rh104}
Die verwendeten Messdaten werden in der \autoref{tab:Rhodium} aufgelistet. Diese Daten werden nun halblogarithmisch abgebildet, da so die exponentielle Zerfallskurve in eine Gerade
übergeht. Diese halblogarithmische Darstellung wird in \autoref{fig:Rhodium} dargestellt. Aufgrund der längeren Zerfallsdauer der Rhodium-104-Isotope, muss die daraus 
resultierende Zerfallskurve in logarithmischer Darstellung eine geringere Steigung haben, als die Zerfallskurve des Rh-104i-Isotops. Da in \autoref{fig:Rhodium} eine Überlagerung
beider Zerfallskurven dargestellt wird, muss ein sinnvoller \textit{Cut} gefunden werden, ab welchem der Zerfall des Rh-104i-Isotops vernachlässigt werden kann. 
Der gesuchte Bereich sollte einen linearen Verlauf darstellen, da hier näherungsweise nur noch ein Zerfall stattfindet.
Dieser \textit{Cut} wird bei
$t^* = \qty{330}{\second}$ festgelegt. Es wird für $t > t^*$
eine lineare Regression durchgeführt. Dabei werden offensichtliche Ausreißer nicht berücksichtigt, da diese das Ergebnis verfälschen würden. Diese \glqq Ausreißer\grqq \;
werden in \autoref{fig:Rhodium} mit grauen Markern dargestellt.
Durch die Parameter der Regression ergeben sich die Zerfallskonstante $\lambda_\text{Rh, lang} = -a = \qty{2.9(0.4)e-3}{\second}$ und 
$b = \num{63 +- 14}$.

Daraus ergibt sich gemäß \autoref{eqn:Halbwertszeit} die Halbwertszeit 
\begin{equation*}
  T_\text{Rh, lang} = \qty{238 +- 33}{\second}
\end{equation*}
mit einem Fehler gemäß der Gaußschen Fehlerfortpflanzung nach \autoref{eqn:delta_T}.

\begin{figure}
  \centering
  \includegraphics{Rhodium.pdf}
  \caption{Logarithmus der Messwerte zu Rhodium und Regressionsgeraden der verschiedenen Bereiche.}
  \label{fig:Rhodium}
\end{figure}

\subsection{Zerfall von Rhodium-104i}
\label{subsec:Rh104i}

Nun kann die Zerfallskurve von Rh-104i untersucht werden. Damit nun die reine Zerfallskurve der Rh-104i-Isotope betrachtet werden kann, muss mittels der zuvor
bestimmten Halbwertszeit des Rhodium-104 Isotopes eine Anpassung der Messwerte vorgenommen werden. Außerdem ist der mittlere Bereich der Messung nicht geeignet für eine Betrachtung, da die 
Vermischung der beiden Zerfälle dort am größten ist. Daher werden nun lediglich Messwerte bis zu einem festgelegten $t_\text{max}$ betrachtet. Es wird ein 
$t_\text{max} = \qty{270}{\second}$ gewählt. 

\begin{table}
  \caption{Messwerte der Zählraten für Rhodium inklusive der jeweiligen Fehler.}
  \label{tab:Rhodium}
  \centering
  \begin{tabular}{S[table-format = 3.0] S[table-format = 3.0] @{${}\pm{}$} S}
    \toprule
    {Messzeit $t/\unit{\second}$} & \multicolumn{2}{c}{Zählrate $N$} \\
    \midrule
      0 & 398 & 20 \\
     15 & 292 & 17 \\
     30 & 258 & 16 \\
     45 & 205 & 14 \\
     60 & 183 & 14 \\
     75 & 161 & 13 \\
     90 & 128 & 11 \\
    105 & 127 & 11 \\
    120 & 117 & 11 \\
    135 &  79 &  9 \\
    150 &  71 &  8 \\
    165 &  76 &  9 \\
    180 &  62 &  8 \\
    195 &  61 &  8 \\
    210 &  56 &  7 \\
    225 &  41 & 6 \\
    240 &  47 & 7 \\
    255 &  42 & 6 \\
    270 &  44 & 7 \\
    285 &  32 & 6 \\
    300 &  33 & 6 \\
    315 &  30 & 5 \\
    330 &  30 & 5 \\
    345 &  37 & 6 \\
    360 &  35 & 6 \\
    \bottomrule
  \end{tabular}
  \begin{tabular}{S[table-format = 3.0] S[table-format = 2.0] @{${}\pm{}$} S}
    \toprule
    {Messzeit $t/\unit{\second}$} & \multicolumn{2}{c}{Zählrate $N$} \\
    \midrule
    375 &  33 & 6 \\
    390 & 27 & 5 \\
    405 & 30 & 5 \\
    420 & 13 & 4 \\
    435 & 23 & 5 \\
    450 & 25 & 5 \\
    465 & 23 & 5 \\
    480 & 22 & 5 \\
    495 & 27 & 5 \\
    510 & 25 & 5 \\
    525 & 19 & 4 \\
    540 & 16 & 4 \\
    555 & 18 & 4 \\
    570 & 22 & 5 \\
    585 & 19 & 4 \\
    600 & 20 & 4 \\
    615 & 13 & 4 \\
    630 & 11 & 3 \\
    645 & 17 & 4 \\
    660 & 22 & 5 \\
    675 & 18 & 4 \\
    690 & 19 & 4 \\
    705 & 18 & 4 \\
    720 & 15 & 4 \\
    \bottomrule
    \\
  \end{tabular}
\end{table}

Die Anpassung der Messwerte wird nun über eine Differenz der Gesamtanzahl aller Isotope zur Anzahl der Rh-104-Isotope bestimmt.
Daher ergibt sich
\begin{equation*}
  N_{\text{kurz}}(t) = N_\text{Ges}(t) - N_\text{lang}(t) = N_\text{Ges}(t) -  N_{0,\text{lang}}e^{-\lambda_\text{lang}t}
\end{equation*}

Die Unsicherheit dieser Anpassung ergibt sich gemäß der Gaußschen Fehlerfortpflanzung durch 
\begin{equation*}
  \label{eqn:delta_kurz}
  \symup{\Delta}N_\text{kurz}(t) = \sqrt{\left(\symup{\Delta}N_\text{Ges}\right)^2 + \left(tN_{0,\text{lang}}e^{-\lambda_\text{lang}t}\right)^2(\symup{\Delta}\lambda_\text{lang})^2 + \left(N_{0,\text{lang}}e^{-\lambda_\text{lang}t}\right)^2(\symup{\Delta}N_{0,\text{lang}})^2}
\end{equation*}

Nun wird eine lineare Regression auf die korrigierten Messwerte angewendet. Dabei ergeben sich aus den Parametern der Regression 
\begin{align*}
  \lambda_\text{Rh, kurz} &= -a = \qty{15.5(0.7)e-3}{\second} \\
  &\text{und} \\
   b &= \num{306(31)} 
\end{align*}
Die Abweichungen ergeben sich gemäß der bereits verwendeten Fehlerformeln.
Daraus kann nun gemäß \autoref{eqn:Halbwertszeit} mit dem bereits verwendeten Fehler eine Halbwertszeit 
\begin{equation*}
  T_\text{Rh, lang} = \qty{44.7+-1.9}{\second}.
\end{equation*}
des Zerfalles bestimmt werden.
