\section{Auswertung}
\label{sec:Auswertung}
Um aus den aufgenommenen Zerfallskurven die Halbwertszeiten bestimmen zu können, bietet es sich an, den Logarithmus der Messwerte gegen die Zeit aufzutragen.
Durch dieses Vorgehen kann \autoref{eqn:N_dt} als lineare Funktion 
\begin{equation}
  \label{eqn:linear}
  \symup{log}\left(N_{\symup{\Delta}t}(t) \right) = \symup{log}\left(N_0 \left(1- \symup{e}^{-\lambda \symup{\Delta}t}\right)\right) - \lambda t := f(t) = at + b
\end{equation}
dargestellt werden, womit sich durch lineare Ausgleichsrechnung die Zerfallskonstante $\lambda$ bestimmen lässt.
Vor den Messungen der Zerfallskurven wird der Nulleffekt, also der durch äußere Einflüsse hervorgerufene Hintergrund der Zählraten, ermittelt. Dazu werden $\num{340}$
Impulse in einer Zeit von $\qty{600}{\second}$ gemessen. Da die Zählraten eines Geiger-Müller-Zählrohres einer Poissonverteilung folgen, ergibt sich zu jeder Zählrate 
$N$ eine Unsicherheit von $\symup{\Delta}N = \sqrt{N}$. Damit folgt für den Nulleffekt pro Sekunde $N_0 = \qty{0.567+-0.031}{\second^{-1}}$.

\subsection{Bestimmung der Halbwertszeit von Vanadium}
\label{subsec:A_Vanadium}
Zur Bestimmung der Halbwertszeit von Vanadium werden die Messwerte aus \autoref{tab:Vanadium} verwendet. Die Länge der Messintervalle beträgt 
$\symup{\Delta}t = \qty{30}{\second}$. Vor der Ausgleichsrechnung zur Bestimmung der Zerfallskonstante wird der Nulleffekt für $t = \qty{30}{\second}$ subtrahiert.
Da sich dabei aufgrund von statistischen Schwankungen negative Werte ergeben, die sich jedoch nicht für die weitere Rechnung eignen, werden nur die neuen Zählraten
mit einem Wert über $0$ betrachtet.

\begin{table} [H]
  \caption{Messwerte der Zählrate für Vanadium inklusive der jeweiligen Fehler.}
  \label{tab:Vanadium}
  \centering
  \begin{tabular}{S[table-format = 3.0] S[table-format = 2.0] @{${}\pm{}$} S}
    \toprule
    {Messzeit $t/\unit{\second}$} & \multicolumn{2}{c}{Zählrate $N$} \\
    \midrule
     30 & 76 &  9 \\
     60 & 94 & 10 \\
     90 & 72 &  8 \\
    120 & 70 &  8 \\
    150 & 62 &  8 \\
    180 & 66 &  8 \\
    210 & 66 &  8 \\
    240 & 53 &  7 \\
    270 & 46 &  7 \\
    300 & 49 &  7 \\
    330 & 56 &  7 \\
    360 & 45 &  7 \\
    390 & 36 &  6 \\
    420 & 31 &  6 \\
    450 & 30 &  5 \\
    \bottomrule
  \end{tabular}
  \begin{tabular}{S[table-format = 3.0] S[table-format = 2.0] @{${}\pm{}$} S}
    \toprule
    {Messzeit $t/\unit{\second}$} & \multicolumn{2}{c}{Zählrate $N$} \\
    \midrule
    480 & 19 & 4 \\
    510 & 25 & 5 \\
    540 & 31 & 6 \\
    570 & 23 & 5 \\
    600 & 25 & 5 \\
    630 & 23 & 5 \\
    660 & 21 & 5 \\
    690 & 19 & 4 \\
    720 & 18 & 4 \\
    750 & 23 & 5 \\
    780 & 16 & 4 \\
    810 & 18 & 4 \\
    840 & 21 & 5 \\
    870 & 19 & 4 \\
    900 & 12 & 3 \\
    \bottomrule
  \end{tabular}
\end{table}

Durch Auftragen des Logarithmus der Zählraten pro Zeitintervall $\symup{\Delta}t$ gegen die Zeit $t$ ensteht das Diagramm \ref{fig:Vanadium}. Die Messwerte modellieren einen 
annähernd linearen Verlauf. Mithilfe des Ansatzes \eqref{eqn:linear} wird eine lineare Regression mittels \textit{scipy} \cite{scipy} durchgeführt.
Es ergeben sich die Parameter
\begin{align*}
  a &= \qty{-0.0048+-0.0004}{\second^{-1}} & b &= \num{ 4.49+-0.20}
\end{align*}
der Ausgleichsgeraden $f(t) = at + b$. Aus \autoref{eqn:linear} folgt sofort $\lambda_\text{V} = -a = \qty{0.0048+-0.0004}{\second^{-1}}$. Die Halbwertszeit kann nun gemäß 
\autoref{eqn:Halbwertszeit} berechnet werden. Dabei ergibt sich durch die Gaußsche Fehlerfortpflanzung eine Unsicherheit von 
\begin{equation}
  \label{eqn:delta_T}
  \symup{\Delta}T = \frac{\ln(2)}{\lambda^2} \cdot \symup{\Delta} \lambda. 
\end{equation}
Es folgt der Wert $T_\text{V} = \qty{144+-12}{\second}$ für die Halbwertszeit von Vanadium. 

\begin{figure}
  \centering
  \includegraphics{Vanadium.pdf}
  \caption{Logarithmus der Zählraten zu Vanadium und lineare Regression der Messwerte. Erstellt mit \textit{matplotlib} \cite{matplotlib}.}
  \label{fig:Vanadium}
\end{figure}

\subsection{Bestimmung der Zerfallseigenschaften von Rhodium}
\label{subsec:A_Rhodium}

\begin{figure}
  \centering
  \includegraphics{Rhodium.pdf}
  \caption{Plot.}
  \label{fig:Rhodium}
\end{figure}
