\section{Diskussion}
\label{sec:Diskussion}
Zunächst wird das Vanadium-Isotop betrachtet. Bei diesem wurde eine Halbwertszeit $T_{\text{V}} = \qty{144(12)}{\second}$ bestimmt. In der Literatur gilt für dieses 
Vanadium-Isotop eine Halbswertszeit von $T_\text{V, lit} = \qty{224}{\second}$ \cite{Zerfall}. Daraus ergibt sich eine Abweichung von $\symup{\Delta}T_{\text{V}} = \qty{36}{\percent}$.
Die Abweichung eines Messwertes $x$ zu einem Literaturwert $x^*$ kann dabei über die Gleichung
\begin{equation}
    \symup{\Delta}x = \frac{|x - x^*|}{x^*}
\end{equation}
bestimmt werden.
Diese Abweichung ist relativ groß und kann nicht ausschließlich durch satistische Abweichungen begründet werden. Der wahrscheinlichste Grund für diese Abweichung ist, dass 
die Probe vor Messbeginn nicht ausreichend reaktiviert wurde, obgleich die vorgeschriebene Zeit eingehalten wurde. Ebenfalls wirken sich die sehr geringen Zählraten mancher
Messwerte stark aus, was auch an den Fehlerbalken in \autoref{fig:Vanadium} festgestellt werden kann. Einige Werte weichen stark von der Ausgleichsgeraden ab, schneiden diese 
jedoch in ihrem Fehlerbereich. 


Für das Rhodium-104-Isotop wurde eine Halbwertszeit $T_\text{Rh, lang} = \qty{238(33)}{\second}$ bestimmt. Zu diesem Zerfall lautet der Literaturwert 
$T_\text{Rh,lang,lit} = \qty{260}{\second}$ \cite{Zerfall}. Daher weicht der errechnete Wert um $\symup{\Delta}T_\text{Rh, lang} = \qty{8}{\percent}$ von dem Literaturwert ab.
Diese Abweichung ist klein genug, damit sie im Unsicherheitsbereich liegt und somit als qualitativ angesehen werden kann. Gründe für die Abweichung sind demnach vor allem 
die allgemeine Messunsicherheit durch die vergleichsweise kleinen Zählraten.

Zuletzt wird nun die Halbwertszeit des Rh-104i-Isotops betrachtet. Diese ergab sich zu $T_\text{Rh, kurz} = \qty{44.7(1.9)}{\second}$. Mit dem Literaturwert 
$T_\text{Rh, kurz, lit} = \qty{42.3}{\second}$ \cite{104iRh} ergibt sich eine Abweichung von $\symup{\Delta}T_\text{Rh, kurz} = \qty{6}{\percent}$. Diese Abweichung ist ähnich klein wie die 
des Rh-104-Isotopes und unterliegt den gleichen Unsicherheiten.

Zusammenfassend konnten die Halbswertszeiten der Rhodium-Isotope zufriedenstellend genau bestimmt werden. Die experimentell bestimmte Halbswertszeit von Vanadium
weicht jedoch stark von dem genannten Literaturwert ab.
