\section{Durchführung}
\label{sec:Durchführung}
In diesem Versuch werden, wie im \autoref{sec:Ziel} beschrieben, die Trägheitsmomente verschiedener Körper berechnet. 

\subsection{Vorbereitungsaufgaben}
\label{subsec:D_Va}
Vorbereitend zu diesem Versuch werden zunächst ein paar beispielhafte Drehmomente $M_{\text{Bsp}}$
zu verschiedenen Abständen $r_{\text{Bsp}}$ berechnet. Dazu wird die Formel $M = Fr\, \text{cos}(\frac{\pi}{2})$ genutzt. Die errechneten Werte können \autoref{tab:D_VA} entnommen werden.
\begin{table}
    \centering
    \caption{Berechnete Werte der Vorbereitungsaufgabe} 
    \label{tab:D_VA}
    \begin{tabular}{c c}
        \toprule
        $r_{\text{Bsp}}\mathrm{/} \unit{\centi\metre}$ & $M_{\text{Bsp}}\mathrm{/}\unit{{\milli\newton\metre}}$\\
        \midrule
        5 & 3.563 \\
        7.5 & 5.3 \\
        10 & 7.07 \\
        12.5 & 8.8 \\
        15 & 10.6 \\
        17.5 & 12.37 \\
        20 & 14.14 \\
        22.5 & 15.9 \\
        25 & 17.67 \\
        \bottomrule 
    \end{tabular}
\end{table}