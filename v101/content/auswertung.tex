\section{Auswertung}
\label{sec:Auswertung}
Die Messunsicherheiten des folgenden Kapitels wurden mit \textit{Python} unter Verwendung des Paketes \textit{scipy} \cite{scipy} bestimmt. Sie folgen aus der gaußschen
Fehlerfortpflanzung
\begin{equation}
  \label{eqn:Gauss}
  \Delta F = \sqrt{\sum_i\left(\frac{\symup{d}F}{\symup{d}y_i}\Delta y_i \right)^2}.
\end{equation} 

\subsection{Bestimmung der Winkelrichtgröße und des Eigenträgheitsmoments der Drehachse}
\label{subsec:A_Apparatenkonstanten}
Bevor mit dem eigentlichen Versuch begonnnen werden kann, müssen die Winkelrichtgröße $D$ der Feder und das Eigenträgheitsmoments $I_D$ der Drehachse ermittelt werden. 
Erstere kann mithilfe der Messwerte aus \autoref{tab:Winkelricht} durch \autoref{eqn:Winkelrichtgröße} bestimmt werden. Neben den Messwerten finden sich die jeweiligen 
Werte der Winkelrichtgröße in der genannten Tabelle.
\begin{table}
  \centering
  \caption{Messdaten zur Bestimmung der Winkelrichtgröße zum festen Abstand a = \qty{20}{\centi\metre}}
  \label{tab:Winkelricht}
  \begin{tabular}{S[table-format = 3.0] S[table-format = 1.3] S}
    \toprule
    {$\phi \mathbin{/} °$} & {$F \mathbin{/} \unit{\newton}$} & {$D \mathbin{/} \unit{\newton\metre\per\radian}$} \\
    \midrule
     20 & 0.025 & 0.029 \\
     30 & 0.045 & 0.034 \\
     40 & 0.067 & 0.038 \\
     50 & 0.1   & 0.046 \\
     60 & 0.124 & 0.047 \\
     70 & 0.145 & 0.047 \\
     80 & 0.17  & 0.049 \\
     90 & 0.25  & 0.064 \\
    100 & 0.27  & 0.062 \\
    110 & 0.3   & 0.063 \\
    \bottomrule
  \end{tabular}
\end{table}
Durch Mittelung der experimentellen Werte für die Winkelrichtgröße ergibt sich der Mittelwert $D = \qty{0.048 +- 0.012}{\newton\metre\per\radian}$.

Zur Bestimmung des Eigendrehmoments der Drehachse wird \autoref{eqn:Schwingungsdauer} betrachtet. Für das Quadrat der Schwingungsdauer $T$ ergibt sich
durch einsetzen des Gesamtträgheitsmoments $I = I_D + I_\text{Zylinder}$ unter Verwendung des Satzes von Steiner \eqref{eqn:Steiner}
\begin{equation}
  \label{eqn:I_D}
  T^2 = \frac{4\pi^2}{D}\left(I_D + 2I_\text{Z,h} + 2 m a^2) \right)
\end{equation}
mit der Masse $m = \qty{261.2}{\gram}$, dem Trägheitsmoment $I_\text{Z,h}$ eines zylinderförmigen Gewichtes und dem Abstand $a$ der Zylinder zur Drehachse. $I_\text{Z,h}$
berechnet sich dabei nach der Gleichung für einen horizontalen Zylinder aus \autoref{fig:Trägheitsmomente}.
Dies stellt eine Geradengleichung der Form $f(x) = mx^2 +b$ mit $f(x) = T^2(a^2)$ dar. In \autoref{fig:plot} sind die 
Quadrate der Messwerte für die Schwingungsdauer $T$ zum Abstand $a$ zur Drehachse aufgeführt. 
\begin{figure}
  \centering
  \includegraphics[width=0.8\textwidth]{plot.pdf}
  \caption{Graph der Quadrate der Messwerte und Ausgleichsgerade der linearen Regression. \cite{matplotlib}}
  \label{fig:plot}
\end{figure}
Eine lineare Regression mittels \textit{scipy} \cite{scipy} ergibt die Geradenparameter $m = \qty{732 +- 5}{\second\squared\per\metre\squared}$ und 
$b = \qty{5.62 +- 0.27}{\second\squared}$. Ein Koeffizientenvergleich von \autoref{eqn:I_D} und der Geradengleichung ergibt 
\begin{equation}
  b = \frac{4\pi^2}{D}(I_D + 2I_\text{Z,h}),
\end{equation}
woraus sich das Eigenträgheitsmoment $I_D$ bestimmen lässt. Mit dem Durchmesser $d = \qty{4.5}{\centi\metre}$ und der Höhe $h = \qty{2}{\centi\metre}$ der zylinderförmigen
Gewichte folgt $I_D = \qty{6.2 +- 1.6}{\gram\square\metre}$. Dieser Wert liegt eine Größenordnung über den im Folgenden zu bestimmenden Trägheitsmomenten, was nicht der 
Wahrheit entsprechen kann, da das Trägheitsmoment der Drehachse selbst sehr viel kleiner ist. Näheres hierzu findet sich in \autoref{sec:Diskussion}. Der wahre Wert von
$I_D$ wird für weitere Rechnungen als vernachlässigbar gering angenommen.
