\section{Ziel}
\label{sec:Ziel}
In diesem Versuch sollen die Trägheitsmomente zweier Körper und einer Modellpuppe experimentell ermittelt werden. Die experimentell Bestimmten Trägheitsmomente lassen sich
anschließend mit den theoretisch berechneten Werten vergleichen um den \textit{Steiner'schen Satz} zu verifizieren. 

\section{Theorie}
\label{sec:Theorie}
Analog zur Masse eines Körpers bei einer Translation gibt es ein Trägheitsmoment $I$, welches die Trägheit des Körpers gegenüber Rotationen beschreibt. Für eine punktförmige
Masse $m$ mit senkrechtem Abstand $r$ zur Drehachse berechnet sich das Drehmoment zu $I = mr^2$. Bei einer kontinuierlichen Massenverteilung ergibt sich mit der 
Dichte $\rho(r)$
\begin{equation}
    \label{eqn:Trägheitsmoment}
    I = \int r^2 \symup{d}m = \int_V \rho(r) r^2 \symup{d}V
\end{equation}
für das Trägheitsmoment bezüglich einer Achse durch den Körperschwerpunkt. Verläuft die Drehachse nicht durch den Schwerpunkt des Körpers, lässt sich das Trägheitsmoment 
bezüglich dieser Achse mithilfe des Steiner'schen Satzes berechnen. Dieser liefert für eine Drehchse, welche parallel zu einer Achse durch den Schwerpunkt liegt und den Abstand
$a$ zu dieser hat, das Trägheitsmoment 
\begin{equation}
    \label{eqn:Steiner}
    I = I_\text{S} + ma^2,
\end{equation}    
wobei $I_\text{S}$ das Trägheitsmoment bezüglich der Schwerpunktsachse ist.
Für einfache Geometrien ergeben sich bei konstanter Dichte $\rho$ grundlegende Trägheitsmomente, die \autoref{fig:Trägheitsmomente} zu entnehmen sind.
\begin{figure}
    \centering
    \caption{Trägheitsmomente einfacher Körper mit homogener Dichte \cite{v101}.}
    \label{fig:Trägheitsmomente}
    \includegraphics[width = 0.85\textwidth]{content/Trägheitsmomente.jpg}
\end{figure}

In diesem Versuch wird ein System betrachtet, bei welchem eine Drehachse über das rücktreibende Drehmoment einer Feder in Schwingung versetzt werden kann, wenn die Drehachse
um einen Winkel $\phi$ ausgelenkt wird. Ein Drehmoment $\vec{M}$ wird durch eine Kraftwirkung $\vec{F}$ entlang eines Hebels $\vec{r}$ senkrecht zur Drehachse bewirkt und 
lässt sich über den Zusammenhang $\vec{M} = \vec{F} \times \vec{r}$ berechnen.
Das beschriebene System ist ein Beispiel eines harmonischen Oszillators, dessen Differentialgleichung durch eine Schwingung mit der Schwingungsdauer 
\begin{equation*}
    \label{eqn:Schwingungsdauer}
    T = 2\pi \sqrt{\frac{I}{D}}
\end{equation*}
gelöst wird. $D$ ist dabei die Winkelrichtgröße (\textit{Direktionsmoment}), welche der Federkonstante bei einer Translation entspricht. Aus dieser Gleichung lässt sich das Drehmoment 
des Körpers --abzüglich des Eigendrehmoments $I_\text{D}$ der Drehachse selbst-- zu
\begin{equation}
    \label{eqn:I_K}
    I_\text{K} = \frac{T^2}{4\pi^2}D - I_\text{D}
\end{equation}
bestimmen.
Die Winkelrichtgröße beschreibt den Zusammenhang 
\begin{equation}
    \label{eqn:Winkelrichtgröße}
    M = D \cdot \phi
\end{equation}
zwischen Auslenkung $\phi$ und dem Betrag $M$ des wirkenden Drehmoments. Der Betrag des Drehmoments kann gemäß der Gleichung $M = |\vec{F}|\cdot|\vec{r}| \sin{\alpha}$ mit 
dem Winkel $\alpha$ zwischen $\vec{r}$ und $\vec{F}$ berechnet werden.
