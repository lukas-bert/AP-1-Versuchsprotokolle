\section{Diskussion}
\label{sec:Diskussion}
Um die experimentellen Werte der Trägheitsmomente in diesem Versuch aus den Messwerten ermitteln zu können wurden zu Anfang die Winkelrichtgröße $D$ und das Eigenträgheitsmoment
$I_D$ bestimmt. Die Winkelrichtgröße ist eine konstante der Feder, die eine Proportionalität zwischen Auslenkwinkel und dem wirkenden Drehmoment beschreibt. Betrachtet man die
einzelnen Werte dieser Größe in \autoref{tab:Winkelricht} fällt auf, dass die Werte nicht konstant sind, sondern kontinuierlich steigen. Eine Ursache dafür ist ein Wechsel des 
Messinstrumentes, was allerdings nur den Sprung der Werte bei $\phi = 90°$ erklärt. Allgemein könnte die Ungenauigkeit der Federwaagen eine Ursache darstellen, da diese 
möglicherweise defekt waren. Ein systematischer Fehler ist ebenfalls nicht auszuschließen. Auf Grund der großen Streuung der Messwerte ist die Güte des experimentell bestimmten 
Wertes der Winkelrichtgröße $D = \qty{0.048 +- 0.012}{\newton\metre\per\radian}$ mangelhaft, was sich auch auf die folgenden experimentellen Werte auswirkt. Bei der Bestimmung 
des Eigenträgheitsmoments $I_D$ sollte der Metallstab als masselos angenommen werden. Dies ist jedoch nicht sinnvoll, da die Masse des Stabes im Vergleich zu jener der Drehachse
selbst groß ist und somit, auch durch die waagerechte Ausrichtung des Stabs, das bestimmte Eigenträgheitsmoment $I_D = \qty{6.2 +- 1.6}{\gram\square\metre}$ wohl zu einem großen
Teil auf das Trägheitsmoment des Stabs zurückzuführen ist. Wie in \autoref{subsec:A_Apparatenkonstanten} wurde der eigentliche Wert des Eigenträgheitsmoments als vernachlässigbar
klein angenommen.

Bei der Bestimmung der Trägheitsmomente der Kugel und des Zylinders ergaben sich die Werte $I_{\text{z,theo}} = \qty{0.444}{\gram\metre\squared}$ und 
$I_{\text{z,exp}} = \qty{0.70+-0.18}{\gram\metre\squared}$ für den Zylinder, sowie $I_{\text{k,theo}} = \qty{2.53}{\gram\metre\squared}$ und 
$I_{\text{k,exp}} = \qty{4.2+-1.1}{\gram\metre\squared}$ für die Kugel. Die relative Abweichung eines Messwertes $x$ zu einem Theoriewert $x^*$ lässt sich dabei über den
Zusammenhang
\begin{equation}
    \label{eqn:rel_Abw}
    \symup{\Delta_\text{relativ}}(x) = \frac{|x^* - x|}{x^*}
\end{equation}
berechnen. Damit folgt für die relativen Abweichungen $\symup{\Delta_\text{rel}}(I_\text{z}) = \qty{57.74}{\percent}$ und 
$\symup{\Delta_\text{rel}}(I_\text{k}) = \qty{65.62}{\percent}$. Ursachen für diese starken Abweichungen sind die zuvor beschriebene schlechte Güte der Winkelrichtgröße und 
Ungenauigkeiten in der Zeitmessung der Periodendauer.

Bei der Bestimmung der Trägheitsmomente der Modellpuppe ergaben sich die Werte $I_{1,\text{,theo}} = \qty{0.246+-0.040}{\gram\metre\squared}$ und 
$I_{1\text{,exp}} = \qty{0.684+-0.180}{\gram\metre\squared}$ für die erste Stellung, sowie $I_{2\text{,theo}} = \qty{0.247+-0.037}{\gram\metre\squared}$ und 
$I_{2\text{,exp}} = \qty{0.493+-0.125}{\gram\metre\squared}$. Nach \autoref{eqn:rel_Abw} ergeben sich die relativen Abweichungen 
$\symup{\Delta_\text{rel}}(I_1) = \qty{99.51}{\percent}$ und $\symup{\Delta_\text{rel}}(I_2) = \qty{177.97}{\percent}$. Diese extremen Abweichungen können nur zum Teil
auf die zuvor genannten Ursachen zurückgeführt werden. Ein weiterer Grund ist, dass die theoretischen Trägheitsmomente durch Näherung der Modellpuppe mittels Zylinder berechnet
wurden, was nur grob der Realität entspricht. Die Gelenke der Puppe, der Metallstab der Halterung und eventuelle Neigungen der Gliedmaßen wurden vernachlässigt.

Insgesamt ergeben sich bei allen Messungen des Versuchs starke Abweichung, die teilweise auf systematische Fehler, aber auch auf Näherungen zurückzuführen sind. Besonders die 
Bestimmung des Eigenträgheitsmoments ist in der beschriebenen Form ungeeignet. 
