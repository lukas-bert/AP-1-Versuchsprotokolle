\section{Diskussion}
\label{sec:Diskussion}
Der in \autoref{subsec:bragg} experimentell bestimmte Glanzwinkel $\theta_\text{Glanz} = \qty{13.65}{\degree}$ weicht um $\symup{\Delta} \theta_\text{Glanz} = \qty{2.5}{\percent}$
von dem Theoriewert $\theta_\text{Glanz, theo} = \qty{14}{\degree}$ ab. Diese Abweichung ist im Rahmen der Messgenauigkeit akzeptabel, wirkt sich jedoch auf die Güte 
der weiteren Messwerte aus. Ursachen für diese Abweichung könnten in der Kalibrierung des Messgerätes und in Ungenauigkeiten der Winkeleinstellung des Stellmotors liegen.

Bei der Untersuchung des Emissionsspektrums der Kupferanode wurden experimentelle Werte der $K$-Linien bestimmt. Diese weisen Abweichungen 
von $\Delta K_{\alpha} = \qty{0.3}{\percent}$ und $\Delta K_{\beta} = \qty{1.34}{\percent}$ auf, was ebenfalls auf die Kalibrierung des Gerätes zurückzuführen sein könnte.
In Anbetracht der zuvor bestimmten Messunsicherheit von $\qty{2.5}{\percent}$ bestätigen diese Messwerte die Theorie.
Wie in \autoref{subsec:emission} beschrieben konnte kein Grenzwinkel experimentell aufgenommen werden, obgleich der Bereich von $\theta_{\text{Grenz,theo}}$
abgemessen wurde. Eine Ursache dafür könnte ein Hintergrundrauschen des Geiger-Müller-Zählrohres sein, welches keine Differenziation des Beginns der Röntgenphänomene 
zum Grundrauschen erlaubt.  

Anschließend wurde das Auflösungsvermögen der Messapparatur bestimmt. Der Wert lässt sich nicht mit einem Literaturwert vergleichen. Allerdings kann angenommen werden, 
dass der Wert aussagekräftig ist, da die nötigen Werte zur Berechnung des Auflösungsvermögens keinen großen Abweichungen unterliegen.

Im letzten Teil des Messabschnittes zum Emissionsspektrum wurden die Abschirmkonstanten für Kupfer aus den Emissionsenergien bestimmt. Zu $\sigma_1$ kann kein experimenteller
Wert bestimmt werden, da sich mit der vorhandenen Apparatur $E_{K \text{,abs}}$ nicht messen lässt.
Die Abschirmkonstante $\sigma_2$ wurde mit einer Abweichung von $1.71\%$ und $\sigma_3$ mit einer Abweichung von $21.66\%$ bestimmt.
Eine Ursache für die Größe der zweiten Abweichung konnte nicht ermittelt werden.

Im letzten Messabschnitt wurden die Absorptionsenergieen und die Abschirmkonstanten für fünf verschiedene Stoffe berechnet. Dabei liegen die Abweichungen der 
Absorptionsenergien lediglich zwischen $0.82\%$ und $3.94\%$. Eine zusätzliche Fehlerqulle könnte in der Installation der verschiedenen Absorberstoffe in der Messapparatur
liegen, da geringfügige Verstellungen der Winkel der einzelnen Geräte nicht auszuschließen sind.
Die Abweichungen der Abschirmkonstanten zu den Literaturwerten liegen zwischen $3.25\%$ und $17.92\%$. Eine Urache hierfür könnte die Verstärkung der Messunsicherheiten 
durch Approximationen und Fehlerfortpflanzung in der Berechnung der Abschirmkonstanten aus den Absorptionsenergien sein.

Zuletzt wurde mithilfe des Moseley-Gesetzes die Rydbergenergie bestimmt. Diese weicht um 
$6.26\%$ von dem Literaturwert ab, was ebenfalls auf selbige Gründe zurückzuführen ist. 

Es ergibt sich, dass die Qualität der Messwerte zur Verifikation der Theorieerwartungen ausreicht, jedoch nicht für präzise Messungen geeignet ist.
