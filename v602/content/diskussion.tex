\section{Diskussion}
\label{sec:Diskussion}
Im ersten Teil des Versuches wurde die Braggbedingung überprüft. Dabei dafür wurde der Glanzwinkel berechnet. Der gemessene Glanzwinkel hat eine relative Abweichung von 
$\Delta \theta_\text{Glanz} = 2.5\%$ zum Theoriewert. Diese Abweichung genüg dem Anspruch an das verwendete Gerät und um die Braggbedingung zu bestätigen. Diese Abweichung 
könnte unter anderem von einer zu großen Schrittweite der Winkel stammen. Da der Röntgenstrahl in dem verwendeten Gerät nicht durch ein Vakuum geleitet wurde kann die Abweichung
zusätzlich durch mögliche Störungen der Umgebungsluft entstanden sein.  
Im nächsten Teil wurde das Emissionsspektrum einer Kupferanode untersucht. Dabei wurde die Energie der K-Linien experimentell bestimmt. Diese weisen eine Abbewichung 
von $\Delta K_{\alpha} = 0.30\%$ und $\Delta K_{\beta} = 1.34\%$. Diese Abweichungen unterliegen den selben möglichen Fehlern, welche im ersten Teil schon genannt wurden.
Allerdings sind die Abbewichungen klein genug, um sie als qualitativ anzusehen.
Wie in \autoref{subsec:emission} schon beschrieben konnte kein Grenzwinkel experimentell aufgenommen werden, obgleich der Bereich von $\theta_{\text{Grenz,theo}}$
sehrwohl abgemessen wurde. Dies könnte für eine größere Ungenauigkeit der Messapparatur bei kleinen Winkeln sprechen. 
Danach wurde das Auflösungsvermögen der Messapparatur berechnet. Der Wert lässt sich nicht mit einem Literaturwert vergleichen. Allerdings kann angenommen werden, das der
Wert relativ gut ist, da die nötigen Werte zu Berechnung des Auflösungsvermögens keinen großen Abweichungen unterliegen. 
Zuletzt wurden in diesem Messabschnitt die Abschirmkonstanten für Kupfer aus den Emissionsenergien bestimmt. $\sigma_1$ konnte nicht aus den Messwerten bestimmt werden. 
$\sigma_2$ wurde mit einer Abweichung von $1.71\%$ bestimmt. Dies ist eine reltiv kleine Abweichung. $\sigma_3$ wurde mit einer Abbewichung von $21.66\%$ bestimmt. Diese 
Abweichung ist sehr groß und ist nicht eindeutig begründbar.
Im letzten Messabschnitt wurde die Absorptionsenergie und die Abschirmkonstanten für fünf verschiedene Stoffe berechnet. Dabei liegen die Abweichungen der 
Absorptionsenergien lediglich zwischen $0.82\%$ und $3.94\%$. Dies sind allesamt kleine relativ kleine Abweichungen und sind lediglich auf die oben bereits erwähnten Fehler 
zurückzuführen. Dahingegen liegen die Abweichungen der Abschirmkonstanten zwischen $3.25\%$ und $17.92\%$. Da alle Absorber händisch vor das \textit{GMZ} installiert wurden, 
könnten durch Anwinkelung und Luftschichten die Messwerte gestört worden sein. Zuletzt wurde durch das Moseley-Gesetz die Rydbergkonstante bestimmt. Diese weicht um 
$6.26\%$ von dem Literaturwert ab. 
Zusammenfassend waren die Messungen generell relativ qualitativ, allerdings gab es einzelne Messung mit sehr großen Abweichungen. Dennoch ist die Messung mit der verwendeten
Messapparatur nicht für sehr präzise Messungen nicht zu empfehlen.