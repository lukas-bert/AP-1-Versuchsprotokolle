\section{Zielsetzung}
\label{sec:Ziel}
In diesem Versuch wird das Emissionsspektrum einer Kupfer-Anode untersucht und seine Charakteristika analysiert. Des Weiteren wird das Auflösungsvermögen
des Messgeräts und die Absorptionsspektren verschiedener Stoffe betrachtet. 

\section{Theorie}
\label{sec:Theorie}
Röntgenstrahlen können mithilfe einer Elektronenkanone erzeugt werden. Dabei treffen beschleunigte Elektronen auf das Anodenmaterial und geben (kinetische)
Energie an dieses ab. Es wird zwischen zwei Arten der Energieabgabe unterschieden. Ein Elektron kann im Coulombfeld eines Atomkerns abgebremst werden,
wodurch es Energie verliert. Diese Energie wird in Form eines Photons (Röntgenquants) emittiert. Da auf diese Weise ein beliebiger Teil der Energie des
Elektrons abgegeben werden kann, ensteht ein kontinuierliches Spektrum, welches \textit{Bremsspektrum} genannt wird.
Bei vollständiger Abgabe der kinetischen Energie $E_\text{kin} = U_\text{B} \cdot e$ des Elektrons wird eine minimale Wellenlänge
\begin{equation}
    \label{eqn:lambda_min}
    \lambda_\text{min} = \frac{h c}{e U_\text{B}}
\end{equation}
erreicht. Dabei ist $U_\text{B}$ die Beschleunigungsspannung der Apparatur und $e$ die Elementarladung. 
Bei der zweiten Form der Energieabgabe wird ein Atom den Anodenmaterials ionisiert, sodass eine Leerstelle in einer inneren Schale des Atoms entsteht. 
Beim Zurückfallen eines Elektrons einer äußeren Schale wird dann wieder ein Phtoton emittiert. Das Elektron kann jedoch nur diskrete Energien abgeben,
welche der Differenz zwischen den Energieniveaus der ELektronen der verschiedenen Schalen entspricht, wodurch die \textit{charakteristische Röngenstrahlung}
entsteht.
