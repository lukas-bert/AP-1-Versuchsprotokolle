\section{Auswertung}
\label{sec:Auswertung}
Im folgendem werden alle Fehler gemäß der gaußschen Fehlerfortpflanzung mittels \textit{Scipy}\cite{scipy} berechnet.
\subsection{Überprüfung der Braggbedingung}
\label{subsec:bragg}
Um die Braggbedingung zu überprüfen wurden aus einer Kupferanode entstandene Röntgenstrahlen mittels der Drehwinkelmethode reflektiert und dann detektiert. Wie in 
\autoref{fig:bragg} zu sehen ist, liegt der experimentell bestimmte Glanzwinkel bei $\theta_\text{Glanz, exp} = \qty{27.3}{\degree}$. Die Braggbedingung sagt einen 
Glanzwinkel von $\theta_\text{Glanz, theo} = \qty{28}{\degree}$ voraus.
Die relative Abweichung eines Messwertes $x$ zu einem Literaturwert $x^*$ lässt sich zu 
\begin{equation}
  \label{eqn:Delta_rel}
  \symup{\Delta}_\text{rel}(x) = \frac{|x - x^*|}{x^*}
\end{equation}
bestimmen.
Es ergibt sich eine relative Abweichung von $\Delta \theta_\text{Glanz} = 2.5\%$.
\begin{figure}
    \centering
    \includegraphics{plotbragg.pdf}
    \caption{Messwerte zur Überpüfung der Braggbedingung. Matplotlib bitte noch citen}
    \label{fig:bragg}
\end{figure}

\subsection{Emissionsspektrum einer Cu-Röntgenröhre}
\label{subsec:emission}
Zunächst wurde eine größere Messreihe zum Emissionsspektrum der Cu-Röntgenröhre aufgenommen. Diese ist in \autoref{fig:emission} dargestellt. In dieser Abbildung ist 
sowohl der Bremsberg, als auch die $\text{K}_{\alpha}$ und $\text{K}_{\beta}$-Linie zu sehen. Es wurde erwartet, dass dieser Messreihe ein Grenzwinkel entnommen werden kann. Dieser ist 
der Messreihe allerdings nicht zu entnehmen. Der Theoriewert für den Grenzwinkel einer Kupferanode liegt bei $\theta_{\text{Grenz}} = \qty{5.05}{\degree}$.
Dazu gehört dann die minimale Wellenlänge von $\lambda_{\text{min}} = \qty{35.45}{\pico\metre}$. 
\begin{figure}
    \centering
    \includegraphics{plotemission.pdf}
    \caption{Messwerte zum Emissionsspektrum eine Cu-Röntgenröhre.}
    \label{fig:emission}
\end{figure}
Damit das Emissionsspektrum besser ausgewertet werden kann, wird eine detailliertere Messung über die beiden K-Linien aufgenommen. Diese Messdaten werden in \autoref{fig:detail}
dargestellt. Zu den K-Linien kann nun mittles des Detailspektrums die Halbwertsbreite der beiden Peaks bestimmt werden. Dazu wurde mittels einer konstanten Gerade, welche
den Wert $\frac{\text{N}_{\text{max}}}{2}$ hat, graphisch ausgewertet an welchen Stellen sich der jeweilige Peak und die jeweilige Gerade sich schneiden. Diese sind 
ebenfalls in \autoref{fig:detail} dargestellt. Daraus ergeben sich für jeden Peak 2 Winkel $2\theta$. Es zu jedem Winkel, durch die Braggbedingung \eqref{eqn:Bragg},
eine Wellenlänge zugeordnet werden und zu jeder Wellenlänge auch eine Energie. Daher kann nun eine Energiedifferenz zwischen den beiden Winkeln eines Peaks errechnet werden.
Diese sind in \autoref{tab:detailspektrum} dargestellt. Außerdem kann nun ein genauerer Energiewert $E$ für die beiden K-Linien gemessen werden. 
Aus der gemessenen Energie an den Peaks und der Energiedifferenz der Halbwertsbreite kann das Auflösungsvermögen für die beiden K-Linien bestimmt werden.
Das Auflösungsvermögen lässt sich nach 
\begin{equation*}
    A = \frac{E}{\Delta E}
\end{equation*}
berechnet werden. Das Auflösungsvermögen der beiden Peaks wird ebenfalls in \autoref{tab:detailspektrum} dargestellt.
\begin{table}
    \centering
    \caption{In dieser Tabelle werden die Energie $E$, die Energiedifferenz $\Delta E$ und das Auflösungsvermögen $A$ der K-Linien dargestellt.}
    \label{tab:detailspektrum}
    \begin{tabular}{c S[table-format = 1.2] S[table-format = 1.2] S[table-format = 2.2]}
      \toprule
       {$\text{K-Linie}$} & {$E \mathbin{/} \unit{\electronvolt}$} & {$\Delta E \mathbin{/} \unit{\electronvolt}$} & {$A$}\\
      \midrule
        {$\alpha$} & 7.98 & 0.16 & 49.94 \\
        {$\beta$}  & 8.83 & 0.19 & 47.68 \\
      \bottomrule
    \end{tabular}
\end{table}

\begin{figure}
    \centering
    \includegraphics{detailspektrum.pdf}
    \caption{Detailmesswerte zum Emissionsspektrum eine Cu-Röntgenröhre.}
    \label{fig:detail}
\end{figure}

In dieser Messung wurden die Energien der K-Linien genauer aufgenommen. Daher werden nun auch die Abschirmkonstanten für Kupfer bestimmt. Diese werden mit den Gleichungen 
\eqref{eqn:Sigma_Kupfer}, \eqref{eqn:Sigma_Kupfer2} und \eqref{eqn:Sigma_Kupfer3} berechnet. Es wird $E_{K,\text{abs}} = \qty{8.988}{\kilo\electronvolt}$
angenommen. Mit den Energien aus \autoref{tab:detailspektrum} ergeben sich die Abschirmkonstanten $\sigma_{1,exp} = 3.29$, $\sigma_{2,exp} = 11.75$ und 
$\sigma_{3,exp} = 18.79$. Die Theoriewerte dazu lauten $\sigma_{1,theo} = 3.29$, $\sigma_{2,theo} = 11.95$ und 
$\sigma_{3,theo} = 23.99$.

\subsection{Absorptionsspektren verschiedener Stoffe}
\label{subsec:Absorption}
Im letzen Teil dieses Versuchs werden die Absorptionsspektren verschiedener Stoffe betrachtet. Anhand der $K$-Kanten der Spektren lassen sich die Absorptionsenergien 
der jeweiligen Stoffe grafisch ermitteln. Diese können wiederum genutzt werden um die Abschirmkonstanten $\sigma_K$ nach \autoref{eqn:Sigma_K} abzuschätzen.

\subsubsection{Bestimmung der Absorptionsenergie und der Abschirmkonstante}
Zur Bestimmung der Absorptionsenergie $E_\text{abs}$ der $K$-Kante wird ein Mittelwert aus den absoluten Maxima und Minima der Messwerte gebildet.
Alternativ könnten auch alle Messwerte außerhalb der Kantenstruktur gemittelt werden, jedoch führt dies aufgrund der statistischen Fluktuation der Messwerte nicht
unbedingt zu einem besseren Ergebnis. Durch Einzeichnen des Mittelwertes in die Grafik kann ein Schnittpunkt einer horizontalen Linie mit der Messwertkurve bestimmt werden.
Da häufig kein Messwert in unmittelbarer Nähe des Mittelwertes zu finden ist, wird ein linearer Zusammenhang für die K-Kante zwischen den benachbarten Messwerten
approximiert. Dieses Vorgehen ist in \autoref{fig:Zn30} zu sehen.

\begin{figure}
  \centering
  \includegraphics[width = 0.8\textwidth]{Zn30.pdf}
  \caption{Absorptionsspektrum von Zink-30.}
  \label{fig:Zn30}
\end{figure}

Für Zink ergibt sich nach Feststellen eines Winkels $\theta_K = \qty{18.76}{\degree}$ eine Absorptionsenergie von $E_\text{abs} = \qty{9.57}{\kilo\electronvolt}$.
Diese folgt durch Anwenden der Braggbedingung (\autoref{eqn:Bragg}) und Umrechnen der Wellenlänge in eine Energie mittels \autoref{eqn:E_lambda}.
Die Abschirmkonstante $\sigma_K$ kann anschließend über \autoref{eqn:Sigma_K} bestimmt werden. 
Für Zink ergibt sich $\sigma_K = 3.68$

Analog wird das Verfahren für vier weitere Stoffe durchgeführt, die jeweiligen Grafiken können den Abbildungen \ref{fig:Ga31} bis \ref{fig:Zr40} im Anhang 
entnommen werden.  Die Ergebnissse der Auswertung dieser Grafiken sind in \autoref{tab:Absorption} den Theoriewerten gegenüber gestellt.
Die relativen Abweichungen berechnen sich gemäß \autoref{eqn:Delta_rel}.

!!!Wie berechnen sich die Theoriewerte!!! Bzw. Quellenangabe

\begin{table}
  \centering
  \caption{Gegenüberstellung der Messergbenisse und Literaturwerte zur Bestimmung der Absorptionsenergien und Abschirmkonstanten.}
  \label{tab:Absorption}
  \begin{tabular}{l S[table-format = 2.2] S S[table-format = 1.2] S[table-format = 2.2] S[table-format = 1.2] S S[table-format = 2.2]}
    \toprule
     {} & {$\theta^\text{Lit}_K \mathbin{/} \unit{\degree}$} & {$E^\text{Lit}_\text{abs} \mathbin{/} \unit{\kilo\electronvolt}$} &% 
     {$\sigma^\text{Lit}_K$} & {$E_\text{abs} \mathbin{/} \unit{\kilo\electronvolt}$} & {$\sigma_K$} & {$\symup{\Delta}_\text{rel}(E_\text{abs}) \mathbin{/} \%$} &%
     {$\symup{\Delta}_\text{rel}(\sigma_K) \mathbin{/}$ \%}\\
    \midrule
      {Zink-30}      & 18.60 &  9.65 & 3.56 &  9.57 & 3.68 & 0.82 &  3.25 \\ 
      {Gallium-31}   & 10.37 & 10.37 & 3.61 & 10.22 & 3.81 & 1.43 &  5.52 \\
      {Brom-35}      & 13.23 & 13.47 & 3.85 & 13.19 & 4.19 & 2.11 &  8.71 \\
      {Zirconium-40} &  9.86 & 17.99 & 4.00 & 15.63 & 4.51 & 2.93 & 12.87 \\
      {Strontium-38} & 11.04 & 16.10 & 4.10 & 17.28 & 4.83 & 3.94 & 17.92 \\
    \bottomrule
  \end{tabular}
\end{table}

\subsection{Bestimmung der Rydbergkonstante}
\label{subsec:Rydberg}
Nach \autoref{eqn:Bindungsenergie} ist die Bindungsenergie $E_n$ proportional zum Quadrat der effektiven Kernladungszahl $z_\text{eff}$. Mit $n = 1$ ist der 
Proportionalitätsfaktor die Rydbergenergie $R_\infty = \qty{13.606}{\electronvolt}$. Aus den zuvor bestimmten Messergebnissen der Absorptionsenergien und 
Abschirmkonstanten lässt sich ein experimenteller Wert der Konstante ermitteln. Dazu wird die Wurzel der Absorptionsenergie $E_\text{abs}$ gegen die effektive
Kernladungszahl aufgetragen. Durch eine lineare Regression können die Parameter der Geraden $f(x) = ax + b$ bestimmt werden. Durch das Quadrat der Steigung $a$ 
ist ein experimenteller Wert der Rydbergenergie gegeben.

\begin{figure}
  \centering
  \includegraphics[width = 0.8\textwidth]{Rydberg.pdf}
  \caption{Bestimmung der Rydbergenergie mittels linearer Regression durch \textit{Scipy} \cite{scipy}.}
  \label{fig:Rydberg}
\end{figure}

Die Parameter der linearen Regression ergeben sich zu 
\begin{align*}
  a &= \qty{3.80}{\sqrt{\unit{\electronvolt}}} & b &= \qty{-2.26}{\sqrt{\unit{\electronvolt}}}.
\end{align*}
In der Theorie sollte $b = 0$ gelten. 
Durch quadrieren folgt die Rydbergenergie
\begin{equation*}
  R_\infty = \qty{14.45}{\electronvolt}
\end{equation*}
Dies stellt eine relative Abweichung gemäß \autoref{eqn:Delta_rel} von $\symup{\Delta}(R_\infty) = \qty{6.2}{\percent}$ dar.
