\section{Auswertung}
\label{sec:Auswertung}
Im folgendem werden alle Fehler gemäß der gaußschen Fehlerfortpflanzung mittels \textit{Scipy}\cite{scipy} berechnet.
\subsection{Überprüfung der Braggbedingung}
\label{subsec:bragg}
Um die Braggbedingung zu überprüfen wurden aus einer Kupferanode entstandene Röntgenstrahlen mittels der Drehwinkelmethode reflektiert und dann detektiert. Wie in 
\autoref{fig:bragg} zu sehen ist, liegt der experimentell bestimmte Glanzwinkel bei $\theta_\text{Glanz, exp} = \qty{27.3}{\degree}$. Die Braggbedingung sagt einen 
Glanzwinkel von $\theta_\text{Glanz, theo} = \qty{28}{\degree}$ voraus. Daher ergibt sich eine relative Abweichung von $\Delta \theta_\text{Glanz} = 2.5\%$.
\begin{figure}
    \centering
    \includegraphics{plotbragg.pdf}
    \caption{Messwerte zur Überpüfung der Braggbedingung.}
    \label{fig:bragg}
\end{figure}

\subsection{Emissionsspektrum einer Cu-Röntgenröhre}
\label{subsec:emission}



\subsection{Absorptionsspektrum}
\label{subsec:absorption}
