\section{Auswertung}
\label{sec:Auswertung}
Im folgendem werden alle Fehler gemäß der gaußschen Fehlerfortpflanzung mittels \textit{Scipy}\cite{scipy} berechnet.
\subsection{Überprüfung der Braggbedingung}
\label{subsec:bragg}
Um die Braggbedingung zu überprüfen wurden aus einer Kupferanode entstandene Röntgenstrahlen mittels der Drehwinkelmethode reflektiert und dann detektiert. Wie in 
\autoref{fig:bragg} zu sehen ist, liegt der experimentell bestimmte Glanzwinkel bei $\theta_\text{Glanz, exp} = \qty{27.3}{\degree}$. Die Braggbedingung sagt einen 
Glanzwinkel von $\theta_\text{Glanz, theo} = \qty{28}{\degree}$ voraus. Daher ergibt sich eine relative Abweichung von $\Delta \theta_\text{Glanz} = 2.5\%$.
\begin{figure}
    \centering
    \includegraphics{plotbragg.pdf}
    \caption{Messwerte zur Überpüfung der Braggbedingung.}
    \label{fig:bragg}
\end{figure}

\subsection{Emissionsspektrum einer Cu-Röntgenröhre}
\label{subsec:emission}
Zunächst wurde eine größere Messreihe zum Emissionsspektrum der Cu-Röntgenröhre aufgenommen. Diese ist in \autoref{fig:emission} dargestellt. In dieser Abbildung ist 
sowohl der Bremsberg, als auch die $\text{K}_{\alpha}$ und $\text{K}_{\beta}$-Linie zu sehen. Es wurde erwartet, dass dieser Messreihe ein Grenzwinkel entnommen werden kann. Dieser ist 
der Messreihe allerdings nicht zu entnehmen. Der Theoriewert für den Grenzwinkel einer Kupferanode liegt bei $\theta_{\text{Grenz}} = \qty{5.05}{\degree}$.
Dazu gehört dann die minimale Wellenlänge von $\lambda_{\text{min}} = \qty{35.45}{\pico\metre}$. 
\begin{figure}
    \centering
    \includegraphics{plotemission.pdf}
    \caption{Messwerte zum Emissionsspektrum eine Cu-Röntgenröhre.}
    \label{fig:emission}
\end{figure}
Damit das Emissionsspektrum besser ausgewertet werden kann, wird eine detailliertere Messung über die beiden K-Linien aufgenommen. Diese Messdaten werden in \autoref{fig:detail}
dargestellt. Zu den K-Linien kann nun mittles des Detailspektrums die Halbwertsbreite der beiden Peaks bestimmt werden. Dazu wurde mittels einer konstanten Gerade, welche
den Wert $\frac{\text{N}_{\text{max}}}{2}$ hat, graphisch ausgewertet an welchen Stellen sich der jeweilige Peak und die jeweilige Gerade sich schneiden. Diese sind 
ebenfalls in \autoref{fig:detail} dargestellt. Daraus ergeben sich für jeden Peak 2 Winkel $2\theta$. Es zu jedem Winkel, durch die Braggbedingung \eqref{eqn:Bragg},
eine Wellenlänge zugeordnet werden und zu jeder Wellenlänge auch eine Energie. Daher kann nun eine Energiedifferenz zwischen den beiden Winkeln eines Peaks errechnet werden.
Diese sind in \autoref{tab:detailspektrum} dargestellt. Außerdem kann nun ein genauerer Energiewert $E$ für die beiden K-Linien gemessen werden. 
Aus der gemessenen Energie an den Peaks und der Energiedifferenz der Halbwertsbreite kann das Auflösungsvermögen für die beiden K-Linien bestimmt werden.
Das Auflösungsvermögen lässt sich nach 
\begin{equation*}
    A = \frac{E}{\Delta E}
\end{equation*}
berechnet werden. Das Auflösungsvermögen der beiden Peaks wird ebenfalls in \autoref{tab:detailspektrum} dargestellt.
\begin{table}
    \centering
    \caption{In dieser Tabelle werden die Energie $E$, die Energiedifferenz $\Delta E$ und das Auflösungsvermögen $A$ der K-Linien dargestellt.}
    \label{tab:detailspektrum}
    \begin{tabular}{c S[table-format = 1.2] S[table-format = 1.2] S[table-format = 2.2]}
      \toprule
       {$\text{K-Linie}$} & {$E \mathbin{/} \unit{\electronvolt}$} & {$\Delta E \mathbin{/} \unit{\electronvolt}$} & {$A$}\\
      \midrule
        {$\alpha$} & 7.98 & 0.16 & 49.94 \\
        {$\beta$}  & 8.83 & 0.19 & 47.68 \\
      \bottomrule
    \end{tabular}
  \end{table}

\begin{figure}
    \centering
    \includegraphics{detailspektrum.pdf}
    \caption{Detailmesswerte zum Emissionsspektrum eine Cu-Röntgenröhre.}
    \label{fig:detail}
\end{figure}

In dieser Messung wurden die Energien der K-Linien genauer aufgenommen. Daher werden nun auch die Abschirmkonstanten für Kupfer bestimmt. Diese werden mit den Gleichungen 
\eqref{eqn:Sigma_Kupfer} berechnet. Die Werte für $n$, $m$ und $l$ können \autoref{subsec:Abschirmkonstante} entnommen werden. Es wird $E_{K,\text{abs}} = \qty{8.988}{\kilo\electronvolt}$
angenommen. Mit den Energien aus \autoref{tab:detailspektrum} ergeben sich die Abschirmkonstanten $\sigma_{1,exp} = 3.29$, $\sigma_{2,exp} = 11.75$ und 
$\sigma_{3,exp} = 18.79$. Die Theoriewerte dazu lauten $\sigma_{1,theo} = 3.29$, $\sigma_{2,theo} = 11.95$ und 
$\sigma_{3,theo} = 23.99$.

\subsection{Absorptionsspektrum}
\label{subsec:absorption}
