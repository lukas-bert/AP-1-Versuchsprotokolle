\section{Zielsetzung}
\label{sec:Ziel}
In diesem Versuch wird die Elementarladung $e$ mithilfe des historischen \textit{Milikan-Versuchs} bestimmt. Dabei wird das Verhalten geladener
Öltröpfchen in dem $\vec{E}$-Feld eines Plattenkondensators untersucht. Ebenso werden die Advogradokonstante $\symup{N}_{\text{A}}$ und die Faraday-Konstante
experimentell ermittelt.

\section{Theorie}
\label{sec:Theorie}
Bei der Bestimmung der Elementarladung über die Milikan-Methode werden Öltröpfchen in das elektrische Feld eines Plattenkondensators zerstäubt.
Durch Reibung der Tröpfchen mit der Luft, werden diese elektrisch geladen. Die Ladung $q$ der Tröpfchen kann dabei nur ein ganzzahliges
Vielfaches der Elementarladung sein.
Das elektrische Feld des Plattenkondensators ist vertikal ausgerichtet, sodass die auf geladene Teilchen wirkende elektrische Kraft 
$\vec{F}_\text{el}$ genau parallel (oder antiparallel) zur Gravitationskraft $\vec{F}_\text{g}$ wirkt. Da die Teilchen sich mit einer 
Geschwindigkeit $\vec{v}$ durch den luftgefüllten Raum bewegen, wirkt zusätzlich die Stokesche Reibungskraft $\vec{F}_\text{R}$ entgegen der Bewegungsrichtung.
Die Wirkung dieser Kräfte auf ein Teilchen können gemäß
\begin{align}
    \label{eqn:Kraefte}
    \vec{F}_\text{g} &= m \vec{g} \\
    \vec{F}_\text{el} &= q \vec{E} \\
    \vec{F}_\text{g} &= -6\symup{\pi}r\eta_\text{L} \vec{v}
\end{align}
berechnet werden. Das Teilchen hat die Masse $m$, $\vec{g}$ ist die Fallbeschleunigung, $\eta_\text{L}$ die
Viskosität der Luft und $r$ der Radius des Teilchens.
Nach kurzer Zeit stellt sich ein Kräftegleichgewicht ein, wodurch sich die Tröpfchen dann mit einer konstanten Geschwindigkeit bewegen.
Bei abgeschalteten elektrischen Feld bewegen sich die Öltröpfchen dann mit der Geschwindigkeit $v_0$ und erhalten durch den Auftrieb der Luft den Radius
\begin{equation*}
    \label{eqn:Radius}
    r = \sqrt{\frac{9 \eta_\text{L}v_0}{2g(\rho_\text{Oel}- \rho_\text{L})}}.
\end{equation*}
Das Kräftegleichgewicht führt in diesem Fall zu der Gleichung
\begin{equation*}
    \frac{4\symup{\pi}}{3}r^3(\rho_\text{Oel}- \rho_\text{L})g = 6 \symup{\pi} \eta_\text{L}r v_0.
\end{equation*}
Je nach Polung des $\vec{E}$-Feldes wirken die elektrostatische- und die Reibungskraft in verschiedene Richtungen. 
Die Orientierung der Kräfte kann \autoref{fig:Kraeftegleichgewicht} entnommen werden.

\begin{figure}
    \centering
    \includegraphics[width = .9\textwidth]{content/Kräftegleichgewicht.jpg}
    \label{fig:Kraeftegleichgewicht}
    \caption{Orientierung der wirkenden Kräfte bei unterschiedlicher Polung des elektrischen Feldes. \cite{v503}}
\end{figure}

Wenn die obere Platte des Kondensators positiv geladen ist und eine ausreichend große Spannung anliegt, bewegt sich das Öltröpfchen mit der Geschwindigkeit $v_\text{auf}$
nach oben. 
Es folgt das Kräftegleichgewicht
\begin{equation*}
    \label{eqn:Kraefte_v_auf}
    \frac{4\symup{\pi}}{3}r^3(\rho_\text{Oel} + \rho_\text{L})g + 6 \symup{\pi} \eta_\text{L}r v_\text{auf} = qE.
\end{equation*}
Bei entgegengesetzter Polung ergibt sich 
\begin{equation*}
    \label{eqn:Kraefte_v_auf}
    \frac{4\symup{\pi}}{3}r^3(\rho_\text{Oel}- \rho_\text{L})g - 6 \symup{\pi} \eta_\text{L}r v_ab = -qE.
\end{equation*}
mit der nach unten gerichteten Geschwindigkeit $v_\text{ab}$.

Aus diesen beiden Gleichungen kann die Ladung $q$ des Öltröpfchen zu 
\begin{equation}
    \label{eqn:q}
    q = \frac{9}{2} \symup{\pi} \sqrt{\frac{\eta_\text{L}^3(v_\text{ab} - v_\text{auf})}{g(\rho_\text{Oel}- \rho_\text{L})}} \cdot \frac{v_\text{ab} + v_\text{auf}}{E},
\end{equation}
wobei $E$ der Betrag des elektrischen Feldes ist. Für die Geschwindigkeiten folgt der Zusammenhang
\begin{equation}
    \label{eqn:v_0}
    v_0 = v_\text{ab} - v_\text{auf}.
\end{equation}    
Da die Tröpfchen in diesem Versuch kleiner sind als die mittlere freie Weglänge $\overline{l}$ in Luft, gilt die das Stokesche Reibungsgesetzt nicht in der genannten Form.
Um dies zu korrigieren muss eine effektive Viskosität der Luft über den \textit{Cunningham-Korrekturterm}
\begin{equation}
    \label{eqn:n_eff}
    \eta_\text{eff} = \eta_\text{L} \left(\frac{1}{1 + B \frac{1}{pr}}\right)
\end{equation}
berechnet werden. Dazu wird der Luftdruck $p$ und die experimentell bestimmbare Konstante $B =  \num{6.17e-3}\, \text{Torr}\cdot\unit{\centi\metre}$ \cite{v503} verwendet.
Es gilt $1\,\text{Torr} \approx \qty{133.322}{\pascal}$ \textbf{Quellenangabe maybe??}.
Für die korrigierte Ladung gilt
\begin{equation}
    \label{eqn:q_korrigiert}
    q_\text{real} = q_0 \left(1+ \frac{B}{pr}\right)^{3/2}.
\end{equation}
