\section{Diskussion}
\label{sec:Diskussion}
Die experimentellen Werte der Elementarladungen aus korrigierten und unkorregierten Ladungen wurden in \autoref{sec:Auswertung} zu 
$e_0 = \qty{1.155(0.348)e-19}{\coulomb}$ und $e_{0 \text{, korr.}} = \qty{1.742(0.361)e-19}{\coulomb}$ bestimmt. Dies entspricht einer
relativen Abweichung zum Theoriewert $e = \qty{1.602e-19}{\coulomb}$ \cite{Ingenieurwissen} gemäß \autoref{eqn:Delta_rel} von $\symup{\Delta}(e_0) = \qty{27.9}{\percent}$ und 
$\symup{\Delta}(e_{0 \text{korr.}}) = \qty{8.75}{\percent}$. Es wird deutlich, dass die Rechnung mit korrigierten Ladungen zu einem weitaus besseren Ergebnis führt. 
Da der Literaturwert der Elementarladung sogar im Messunischerheits-Bereich des experimentell bestimmten Wertes liegt, zeigt sich, dass die Bestimmung mithilfe der 
Milikan-Methode erfolgreich war. Die verhältnismäßig große Unsicherheit des Wertes von $\qty{0.361e-19}{\coulomb}$ ensteht auf Grund der Fehlerfortpflanzung durch
die zahlreichen Berechnungen und auf Grund der Abweichungen in der Geschwindigkeitsmessung der Tröpfchen. Bei der Messung der Geschwindigkeiten ergaben sich einige
Schwierigkeiten, da selbst bei guter Kalibrierung der Messaparatur die meisten Tröpfchen nicht ausreichend gut zu sehen sind, kein gewünschtes Verhalten aufweisen 
oder nach einiger Zeit in der Unschärfe des Mikroskops verschwinden. Des Weiteren bewegen sich die Öltröpfchen nicht nur vertikal, was die Abweichungen verstärkt.

Die Advogradokonstante konnte mit dem experimentellen Wert zu $N_\text{A, exp} = \qty{5.5374(1.1469)e23}{\per\mol}$ bestimmt werden, was eine relative Abweichung
von $\qty{8.05}{\percent}$ darstellt. Die hier zu benennenden Fehlerquellen sind die Gleichen wie die oben Genannten, da die Advogradokonstante aus der
experimentell bestimmten Elementarladung folgt.

Insgesamt ergeben sich trotz der hohen Schwierigkeit des Experiments Werte, die im Rahmen der Unsicherheiten in einem akzeptablen Bereich um die Literaturwerte
liegen.
