\section{Auswertung}
\label{sec:Auswertung}
Die in diesem Kapitel erwähnten Fehler genügen der Gaußschen Fehlerfortpflanzung beziehungsweise dem Standardfehler des Mittelwertes und wurden mit \textit{uncertainties}
\cite{uncertainties} in \textit{Python} berechnet. Die relative Abweichung eines Messwertes $x$ zu einem Theoriewert $x^*$ wird nach 
\begin{equation}
  \label{eqn:Delta_rel}
  \symup{\Delta}_\text{rel}(x) = \frac{|x^* - x|}{x^*}
\end{equation}
berechnet.
\subsection{Überprüfung der Messdaten im Rahmen der Messungenauigkeit}
\label{subsec:bedingung}
Zunächst soll für die einzelnen Messungen überprüft werden, ob diese die Formel \eqref{eqn:v_0} erfüllen. Sollte dies nicht der Fall sein werden diejenigen Messwerte aus der 
Auswertung ausgelassen, da sich die Ladung dieser Teilchen im Verlauf der Messung geändert haben sollte. Die Messwerte von $v_{\text{auf}}$, $v_{\text{ab}}$ und $v_0$ 
sind in der folgenden \autoref{fig:SCHEIßTABELLEKOMMT NOCH} dargestellt. Ebenfalls wird in dieser Tabelle die Bedingung mit zugehöriger Abweichung der Werte dargestellt.

%%
%
%       TABELLE
%
%

Trotz der hohen Abweichungen werden für die nötige Quantität der Messreihe alle Messwerte im weiteren betrachtet. Daher wird auch angenommen, dass sich die Ladung im Laufe der 
Messung nicht geändert haben. 

\subsection{Bestimmung der Ladung und der Radien der Öltröpfchen}
\label{subsec:ladungradius}
Nun wird zunächst der Radius der Öltröpfchen bestimmt. Dieser kann gemäß Formel \eqref{eqn:Radius} berechnet werden. Der Radius ist allerdings abhängig von der Viskosität
der Luft, der Dichte der Luft und der Dichte des Öltröpfchens. Die Viskosität der Luft hängt von der Temperatur der Luft ab. Die Temperatur kann jedoch aus dem gemessenen 
Thermowiderstand bestimmt werden. Der Wert kann der Tabelle in \autoref{fig:Temperatur} entgenommen werden. Die Dichte der Luft beträgt $\qty{1.204}{\kilo\gram\per\cubic\metre}$
und die Dichte des Öls beträgt $\qty{886}{\kilo\gram\per\cubic\metre}$. Außerdem sind die gemessenen Geschwindigkeiten relevant. Diese können \autoref{fig:SCHEIßTABELLEKOMMT NOCH}
entnommen werden. Damit lassen sich die Radien bestimmen und werden dann in \autoref{fig:KOMMT NOCH IWAN MA KAGGE} dargestellt. 

Nun wird die Ladung der einzelnen Öltröpfchen bestimmt. Dies geschieht gemäß Formel \eqref{eqn:q}. Dabei bestehen zunächst noch die gleichen Abhänigkeiten wie die des Radiuses.
Hier kommt allerdings noch die Energie des elektrischen Feldes im Kondensator hinzu. Diese kann durch den Abstand der Platten $d = \qty{7.6250(0.0051)}{\milli\metre}$ und die 
Spannung am Kondensator bestimmt werden. Die verwendeten Spannungen für die Teilchen können ebenfalls \autoref{fig:ganzoberetabbelle duasua} entnommen werden. Die berechneten,
aber noch unkorregierten, Ladungen können \autoref{fig:TABLELLE VON DIESES ABSCHNITT JUNGE} entnommen werden. Aus diesen Ladungen kann nun eine gemittelte Elementarladung $e_0$
bestimmmt werden. Diese werden in \autoref{fig:NÄCHSTETABELLEHALTDUSAU} dargestellt.