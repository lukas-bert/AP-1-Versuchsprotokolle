\section{Auswertung}
\label{sec:Auswertung}
Die in diesem Kapitel erwähnten Fehler genügen der Gaußschen Fehlerfortpflanzung beziehungsweise dem Standardfehler des Mittelwertes und wurden mit \textit{uncertainties}
\cite{uncertainties} in \textit{Python} berechnet. Die relative Abweichung eines Messwertes $x$ zu einem Theoriewert $x^*$ wird nach 
\begin{equation}
  \label{eqn:Delta_rel}
  \symup{\Delta}_\text{rel}(x) = \frac{|x^* - x|}{x^*}
\end{equation}
berechnet.
\subsection{Überprüfung der Messdaten im Rahmen der Messungenauigkeit}
\label{subsec:bedingung}
Zunächst wird für die einzelnen Messungen überprüft, ob diese den Zusammenhang \eqref{eqn:v_0} der Geschwindigkeiten erfüllen. 
Sollte dies nicht der Fall sein, werden diejenigen Messwerte aus der 
Auswertung ausgelassen, da sich die Ladung dieser Teilchen im Verlauf der Messung geändert haben könnte. Die Messwerte von $v_{\text{auf}}$, $v_{\text{ab}}$ und $v_0$ 
sind in der folgenden \autoref{tab:Geschwindigkeiten} dargestellt. Ebenfalls wird in dieser Tabelle die Abweichung Geschwindigkeiten zur oben genannten Bedingung
\eqref{eqn:v_0} dargestellt.

\begin{table}[H]
    \centering
    \caption{Geschwindigkeiten der Öltröpfchen und Abweichung zur Idealbedingung $2v_0 = v_\text{ab}- v_\text{auf}$. Zwecks Übersichtlichkeit wird auf die Angabe der 
    Messunsicherheiten verzichtet. $N$: Nummer des Öltröpfchens.}
    \label{tab:Geschwindigkeiten}
    \begin{tabular}{c l S[table-format = 2.2] S S S[table-format = 1.2] S[table-format = 2.2]}
      \toprule
        {} & {$N$} & {$v_\text{auf} \mathbin{/} 10^{-5} \unit{\metre\per\second}$} & {$v_\text{ab} \mathbin{/} 10^{-5} \unit{\metre\per\second}$} &% 
        {$v_0 \mathbin{/} 10^{-5} \unit{\metre\per\second}$} & {$\symup{\Delta} v_0 \mathbin{/} \unit{\percent}$} \\
        \midrule
        {$\qty{175}{\volt}:$} &  {1} & 3.42  &  8.03 & 1.59 & 45.05 \\
        {                   } &  {2} & 11.80 & 12.19 &      &       \\
        {                   } &  {3} &  5.04 &  6.44 &      &       \\
        {                   } &  {4} & 13.36 & 19.17 &      &       \\
        {                   } &  {5} &  7.01 &  8.83 & 1.79 & 49.14 \\
        {$\qty{200}{\volt}:$} &  {6} &  9.68 & 10.83 & 0.82 & 29.84 \\
        {                   } &  {7} & 15.34 & 16.56 & 0.60 &  0.50 \\
        {                   } &  {8} &  2.12 &  4.84 & 1.89 & 28.03 \\
        {                   } &  {9} &  1.60 &  5.75 & 2.50 & 17.07 \\
        {                   } & {10} &  7.17 &  9.46 & 1.98 & 41.94 \\ 
        {$\qty{225}{\volt}:$} & {11} &  4.57 &  8.57 & 2.16 &  7.29 \\ 
        {                   } & {12} &  4.12 &  8.26 & 1.61 & 28.85 \\
        {                   } & {13} &  9.06 & 14.15 & 3.46 & 26.54 \\
        {                   } & {14} & 11.97 & 15.08 & 1.71 &  8.80 \\
        {                   } & {15} &  4.03 & 11.39 & 3.35 &  9.98 \\
        {$\qty{250}{\volt}:$} & {16} &  2.98 &  9.49 & 3.07 &  6.00 \\
        {                   } & {17} &  8.75 & 15.98 & 3.38 &  6.83 \\
        {                   } & {18} & 14.70 & 16.32 & 1.04 & 22.06 \\
        {                   } & {19} &  5.09 & 11.97 & 3.67 &  6.31 \\
        {                   } & {20} &  3.98 &  9.22 & 1.81 & 45.18 \\
        {$\qty{275}{\volt}:$} & {21} &  4.08 & 12.11 & 3.48 & 15.19 \\
        {                   } & {22} & 15.21 & 19.06 & 2.29 & 15.73 \\
        {                   } & {23} &  9.51 & 18.61 & 4.94 &  7.72 \\
        {                   } & {24} & 13.17 & 27.85 & 7.90 &  7.05 \\
        {                   } & {25} &  8.72 & 15.08 & 2.61 & 21.90 \\
      \bottomrule
    \end{tabular}
  \end{table}

Wie sich in \autoref{tab:Geschwindigkeiten} erkennen lässt, sind nicht für alle Messpunkte die Geschwindigkeiten $v_0$ bekannt. Des Weiteren weichen einzelne Werte um fast 
$\qty{50}{\percent}$ von der Idealbedingung $2v_0 = v_\text{ab} - v_\text{auf}$ ab. Dennoch werden alle Messwerte in den folgenden Berechnungen verwendet um die Quantität
zu sichern. 

\subsection{Bestimmung der Ladung und der Radien der Öltröpfchen}
\label{subsec:ladungradius}
Der Radius der Öltröpfchen kann gemäß \autoref{eqn:Radius} berechnet werden. Dieser hängt von der Viskosität-
und der Dichte der Luft, sowie von der Dichte der Öltröpfchen ab. Die Viskosität wiederum, hängt von der Umgebungstemperatur ab. Die Temperatur kann aus dem gemessenen 
Thermowiderstand mithilfe der Tabelle \ref{fig:Temperatur} bestimmmt werden. Die Dichte der Luft beträgt $\qty{1.204}{\kilo\gram\per\cubic\metre}$
und die Dichte des Öls beträgt $\qty{886}{\kilo\gram\per\cubic\metre}$ \cite{v503}. Die relevanten Geschwindigkeiten können der oben aufgeführten \autoref{tab:Geschwindigkeiten}
entnommen werden. Die ermittelten Radien sind \autoref{tab:Ladungen} zu entnehmen. 

Zur Bestimmung der Ladungen der Öltröpfchen wird \autoref{eqn:q} verwendet.
Der Betrag des elektrischen Feldes des Plattenkondensators kann über den Zusammenhang $E = U/d$ berechnet werden. Der Abstand der Platten beträgt $d = \qty{7.6250(0.0051)}{\milli\metre}$.
Die verwendeten Spannungen für die Teilchen können ebenfalls \autoref{tab:Geschwindigkeiten} entnommen werden. Die berechneten,
aber noch unkorregierten, Ladungen sind in \autoref{tab:Ladungen} dargestellt. Die korrigierte Ladung berechnet sich nach \autoref{eqn:q_korrigiert} und ist ebenfalls in
\autoref{tab:Ladungen} zu finden.

\begin{table}
    \centering
    \caption{Aus den Messwerten bestimmte, unkorrigierte und korrigierte Ladungen $q$, sowie Radien der Öltröpfchen. Es wird erneut auf die Angabe der 
            Messunischerheiten verzichtet. $N$: Nummer des Öltröpfchens.}
    \label{tab:Ladungen}
    \begin{tabular}{c l S[table-format = 1.2] S[table-format = 2.2] S}
      \toprule
        {} & {$N$} & {$r \mathbin{/} 10^{-7} \unit{\metre}$} & {$q \mathbin{/} {10}^{-19} \unit{\coulomb}$} & {$q_\text{korr.} \mathbin{/} {10}^{-19} \unit{\coulomb}$} \\
        \midrule
        {$\qty{175}{\volt}:$} &  {1} & 4.68 &  5.14 &  5.14 \\
        {                   } &  {2} & 1.40 &  2.46 &  5.00 \\
        {                   } &  {3} & 2.57 &  2.22 &  3.37 \\
        {                   } &  {4} & 5.26 & 12.87 & 16.01 \\
        {                   } &  {5} & 2.97 &  3.62 &  5.22 \\
        {$\qty{200}{\volt}:$} &  {6} & 2.34 &  4.97 &  4.97 \\
        {                   } &  {7} & 2.40 &  5.05 &  7.85 \\
        {                   } &  {8} & 3.63 &  1.69 &  2.30 \\
        {                   } &  {9} & 4.48 &  2.21 &  2.84 \\
        {                   } & {10} & 3.33 &  3.71 &  5.17 \\
        {$\qty{225}{\volt}:$} & {11} & 4.40 &  4.46 &  4.46 \\
        {                   } & {12} & 4.48 &  3.30 &  4.26 \\
        {                   } & {13} & 4.96 &  6.87 &  8.64 \\
        {                   } & {14} & 3.88 &  6.26 &  8.35 \\
        {                   } & {15} & 5.97 &  5.49 &  6.66 \\
        {$\qty{250}{\volt}:$} & {16} & 5.58 &  4.54 &  4.54 \\
        {                   } & {17} & 5.89 &  7.73 &  9.41 \\
        {                   } & {18} & 2.79 &  4.60 &  6.78 \\
        {                   } & {19} & 5.74 &  5.20 &  6.36 \\
        {                   } & {20} & 5.01 &  3.51 &  4.41 \\
        {$\qty{275}{\volt}:$} & {21} & 6.24 &  5.95 &  5.95 \\
        {                   } & {22} & 4.30 &  7.25 &  9.41 \\
        {                   } & {23} & 6.65 &  9.14 & 10.89 \\
        {                   } & {24} & 8.44 & 16.93 & 19.47 \\
        {                   } & {25} & 5.60 &  6.47 &  7.96 \\
      \bottomrule
    \end{tabular}
  \end{table}

\subsection{Bestimmung der Elementarladung}
\label{subsec:Elementarladung}

Die Messdaten zur korrigierten Ladung der Öltröpfchen sind in \autoref{fig:plot} grafisch dargestellt. Es lässt sich erkennen, dass die meisten Fehler 
(bis auf einige Ausnahmen) in einem Bereich von unter $\qty{1e-19}{\coulomb}$ liegen. Ebenfalls gibt es Häufungspunkte der Messwerte, die in einem 
Abstand genau unter dieser Schwelle liegen. Bei diesen Werten liegt die Vermutung nahe, dass es sich um gleich geladene Tröpfchen handelt, deren
berechneter Ladungswert sich nur auf Grund von Messunsicherheiten unterscheidet.

\begin{figure}
    \centering
    \includegraphics{build/plot.pdf}
    \caption{Messdaten der korrigierten Ladungen zu den verschiedenen Spannungen (mit Fehlerbalken). Erstellt mit \textit{matplotlib} \cite{matplotlib}.}
    \label{fig:plot}
\end{figure}

Die Elementarladung $e$ ist der größte gemeinsame Teiler der Werte. Dieser Teiler ist jedoch nicht exakt berechenbar, da es sich um fehlerbehaftete Größen handelt, die
kein ganzzahliges Vielfaches voneinander sind. Aus diesem Grund wird zur Bestimmung der Elementarladung der geringste Abstand zwischen zwei Messpunkten 
einer Messreihe ermittelt, dabei wird beachtet, dass Abstände unter der oben begründeten Grenze von $\qty{1e-19}{\coulomb}$ verworfen werden, da diese
dieselbe Ladung beschreiben.
Das Verfahren wird für alle Messreihen mit den Ergebnissen der korrigierten und unkorregierten Ladung durchgeführt. Die bestimmten Elementarladungen werden anschließend
gemittelt um einen experimentellen Wert aus allen Messreihen zu erhalten. Die Zwischenergebnisse sind \autoref{tab:elementar} zu entnehmen.

\begin{table}
    \centering
    \caption{Ergebnisse der Ermittlung der Elementarladung aus korrigierten und unkorregierten Ladungen.}
    \label{tab:elementar}
    \begin{tabular}{l S[table-format = 1.3] S}
      \toprule
        {} & {$e_0 \mathbin{/} 10^{-19} \unit{\coulomb}$} & {$e_{0\text{, korr.}} \mathbin{/} 10^{-19} \unit{\coulomb}$} \\
        \midrule
        {$\qty{175}{\volt}$} & 1.161 & 1.627 \\
        {$\qty{200}{\volt}$} & 1.252 & 2.125 \\
        {$\qty{225}{\volt}$} & 1.033 & 1.689 \\
        {$\qty{250}{\volt}$} & 1.030 & 1.817 \\
        {$\qty{275}{\volt}$} & 1.301 & 1.454 \\
        {Mittelwert}         & 1.155 & 1.742 \\
      \bottomrule
    \end{tabular}
  \end{table}
Die experimentell bestimmten Mittelwerte lauten $e_0 = \qty{1.155(0.348)e-19}{\coulomb}$ und $e_{0 \text{, korr.}} = \qty{1.742(0.361)e-19}{\coulomb}$. 
\subsection{Berechnung der Advogradokonstante}
\label{subsec:Avogadro}
Die Advogradokonstante gibt an wie viele Teilchen einer chemischen Verbindung in einem $\unit{\mol}$ enthalten sind.
Sie lässt sich mithilfe der \textit{Faraday-Konstante} $F$ und der Elementarladung durch
\begin{equation*}
    \label{eqn:N_A}
    N_\text{A} = \frac{F}{e}
\end{equation*}
berechnen. Die Faraday-Konstante hat den Wert $F = \qty{96485.3399(0.0024)}{\coulomb\per\mol}$ \cite{Ingenieurwissen}.
Mit dem experimentell bestimmten Wert der Elementarladung aus den korrigierten Ladungen folgt $N_\text{A, exp} = \qty{5.5374(1.1469)e23}{\per\mol}$.
Der Literaturwert beträgt $N_\text{A} = \qty{6.02214179e23}{\per\mol}$ \cite{Ingenieurwissen}.
