\section{Auswertung}
\label{sec:Auswertung}
Die in diesem Kapitel erwähnten Fehler genügen der Gaußschen Fehlerfortpflanzung beziehungsweise dem Standardfehler des Mittelwertes und wurden mit \textit{uncertainties}
\cite{uncertainties} in \textit{Python} berechnet.

\subsection{Filterkurve des Selektivverstärkers}
\label{subsec:A_Filterkurve}
Im ersten Teil des Versuches wird die Filterkurve des Selektivverstärkers analysiert. Die Messwertepaare der Frequenz und Spannung werden in einem Diagramm gegeneinander
aufgetragen, wodurch sich eine Glockenkurve bildet, was in \autoref{fig:plot} zu sehen ist. Um die markanten Punkte der Glockenkurve zu verdeutlichen, ist die Spannung 
relativ zur maximalen Spannung (Eingangsspannung) $U_\text{max} = \qty{8.5}{\volt}$ dargestellt. 
Die Glockenkurve hat die Gestalt
\begin{equation*}
  \label{eqn:Gaussglocke}
  f(x) = \symup{exp}\left(-\alpha(x-b)^2\right).
\end{equation*}

\begin{figure}[H]
  \centering
  \includegraphics{plot.pdf}
  \caption{Filterkurve des Selektivverstärkers. Erstellt mit \textit{matplotlib} \cite{matplotlib} unter Verwendung von \textit{scipy} \cite{scipy}.}
  \label{fig:plot}
\end{figure}

Der Parameter $b$ dient dabei zur Verschiebung der Glockenkurve entlang der $x$-Achse und entspricht dem Wert $\nu_0$ des Maximums. 
Durch eine Regression mittels \textit{scipy} \cite{scipy} ergeben sich die Parameter
\begin{align}
  \label{eqn:Parameter}
  \alpha &= \qty{2.28 +- 0.3}{\per\kilo\hertz\squared} &&\text{und} & b &= \qty{21.75 +- 0.04}{\kilo\hertz}.
\end{align}
Es folgt sofort $\nu_0 = b = \qty{21.75 +- 0.04}{\kilo\hertz}$, die Frequenzen $\nu_-$ und $\nu_+$ können als Schnittpunkte der Funktion mit der Geraden $y = 1/\sqrt{2}$
grafisch ermittel werden. Dieses Vorgehen kann ebenfalls \autoref{fig:plot} entnommen werden. 
Es ergeben sich die Frequenzen $\nu_- = \qty{21.36}{\kilo\hertz}$ und $\nu_+ = \qty{22.14}{\kilo\hertz}$. 
Durch Einsetzen in \autoref{eqn:Guete} kann der Wert der Güte berechnet werden.
Unter Beachtung der Gaußschen Fehlerfortpflanzung folgt
\begin{align*}
  Q &= \num{27.89 +- 0.06}.
\end{align*}

\subsection{Bestimmung der Suszeptibilitäten}
\label{subsec:A_Suszep}
Die Suszptibilität wurde in diesem Versuch auf zwei Arten bestimmt, welche in \autoref{subsec:Messapparatur} beschrieben worden sind. Damit $\chi$ aus den Widerständen gemäß
Formel \eqref{eqn:chimethode2} berechnet werden kann, wurden zunächst die Querschnittsflächen der Proben bestimmt. Diese wurden zu 
$Q_{\text{Dy2O3}} = \qty{12.04}{\milli\metre\squared}$, $Q_{\text{Gd2O3}} = \qty{8.89}{\milli\metre\squared}$ und $Q_{\text{C6O12Pr3}} = \qty{56.75}{\milli\metre\squared}$
bestimmt. Mit einem Spulenquerschnitt von $F = \qty{86,6}{\milli\metre\squared}$, dem Widerstand $R_3 = \qty{998}{\ohm}$ und einem $\symup{\Delta}R$, welches den Orginalmessdaten 
im Anhang entnommen werden kann, ergeben sich die Suszeptibilitäten der Proben. Diese werden in \autoref{tab:Chi_exp} dargestellt.
\begin{table}
  \centering
  \caption{Experimentell ermittelte Suszeptibilitäten.}
  \label{tab:Chi_exp}
  \begin{tabular}{c S[table-format = 1.4] S S r c l}
    \toprule
      {Aus Widerständen} & {$\chi_1$} & {$\chi_2$} & {$\chi_3$} &
      \multicolumn{3}{c}{$\overline{\chi}_\text{exp}$}  \\
    \midrule
      {$\symup{Dy}_2\symup{O}_3$}                 & 0.0198 & 0.0224 & 0.0235 & 0.0219 & {$\pm$} & 0.0016 \\
      {$\symup{Gd}_2\symup{O}_3$}                 & 0.0268 & 0.0303 & 0.0319 & 0.0297 & {$\pm$} & 0.0021 \\
      {$\symup{C}_6\symup{O}_{12}\symup{Pr}_2$}   & 0.0042 & 0.0048 & 0.0050 & 0.0047 & {$\pm$} & 0.0003 \\
    \bottomrule
      {Aus Spannungen} & & & & & \\
    \bottomrule 
      {$\symup{Dy}_2\symup{O}_3$}                 & 0.0203 & 0.0228 & 0.0219 & 0.0217 & {$\pm$} & 0.0011 \\
      {$\symup{Gd}_2\symup{O}_3$}                 & 0.0275 & 0.0309 & 0.0298 & 0.0294 & {$\pm$} & 0.0014 \\
      {$\symup{C}_6\symup{O}_{12}\symup{Pr}_3$}   & 0.0043 & 0.0048 & 0.0047 & 0.0046 & {$\pm$} & 0.0002 \\
    \bottomrule 
  \end{tabular}
\end{table}

Um die Suszeptibilität aus den Spannungen gemäß \autoref{eqn:chi:näherung} zu berechnen, sind ebenfalls die bereits beschriebenen Querschnitte nötig. Außerdem wird ein 
$U_{\text{Sp}} = \qty{8.5}{\volt}$ verwendet und die gemessenen Brückenspannungen $U_{\text{Br}}$, welche ebenfalls den Orginalmessdaten im Anhang entnommen werden können.
Mit diesen Werten ergeben sich dann erneut experimentelle Werte der Suszeptibilitäten. Diese werden in \autoref{tab:Chi_exp} dargestellt.
Theoriewerte der Suszeptibilitäten lassen sich nach Formel \eqref{eqn:chi1} berechnen. Die Theoriewerte unterliegen allerdings der Annahme, dass die Proben bei konstanter 
Raumtemperatur untersucht wurden. Daher wurde eine Raumtemperatur von $T = \qty{293.15}{\kelvin}$ verwendet. Außerdem werden die Quantenzahlen $J$, $S$ und $L$ benötigt. Diese
können aus den Hundschen Regeln bestimmt werden. Für $\symup{Dy}_2\symup{O}_3$ lauten diese $J = 7.5$, $S = 2.5$ und $L = 5$. $\symup{Gd}_2\symup{O}_3$ hat die Quantenzahlen
$J = 3.5$, $S = 3.5$ und $L = 0$. Da zu $\symup{C}_6\symup{O}_{12}\symup{Pr}_2$ keine Dichte in sämtlichen frei zugänglichen Stoffdatenbanken zu finden ist, 
konnten auch keine Theoriewerte bestimmt werden.
Mit dem Bohrschen Magneton und dem \textit{Landé-Faktor}, welche einfach von dem bekannten Quantenzahlen abhängt, können dann die theoretischen Suszeptibilitäten der Proben
bestimmt werden. Die Theoriewerte werden in \autoref{tab:Chi_abwU} dargestellt. Aus diesen wird dann in derselben Tabelle die Abweichung des experimentellen
Wertes zum Theoriewert berechnet. 

\begin{table}
  \centering
  \caption{Theoriewerte der Suszeptibilitäten und experimentelle Abweichung. (Spannungs-Methode)}
  \label{tab:Chi_abwU}
  \begin{tabular}{c S[table-format = 1.4] S @{${}\pm{}$} l S}
    \toprule
       & {$\chi_{\text{theo}}$} & \multicolumn{2}{c}{$\overline{\chi}_\text{exp}$} & {$\symup{\Delta}\chi\mathbin{/}\%$} \\
    \midrule
      {$\symup{Dy}_2\symup{O}_3$}                & 0.0254 & 0.0217 & 0.0011 & 13.74  \\
      {$\symup{Gd}_2\symup{O}_3$}                & 0.0138 & 0.0294 & 0.0014 & 115.37 \\
      {$\symup{C}_6\symup{O}_{12}\symup{Pr}_3$}  & {-}    & 0.0046 & 0.0002 & {-}    \\
    \bottomrule
  \end{tabular}
\end{table}
\begin{table}
  \centering
  \caption{Theoriewerte der Suszeptibilitäten und experimentelle Abweichung. (Widerstand-Methode)}
  \label{tab:Chi_abwR}
  \begin{tabular}{c S[table-format = 1.4] S @{${}\pm{}$} l S}
    \toprule
       & {$\chi_{\text{theo}}$} & \multicolumn{2}{c}{$\overline{\chi}_\text{exp}$} & {$\symup{\Delta}\chi\mathbin{/}\%$} \\
    \midrule
      {$\symup{Dy}_2\symup{O}_3$}                & 0.0254 & 0.0219 & 0.0016 & 14.59 \\
      {$\symup{Gd}_2\symup{O}_3$}                & 0.0138 & 0.0297 & 0.0021 & 113.27 \\
      {$\symup{C}_6\symup{O}_{12}\symup{Pr}_3$}  & {-} & 0.0047 & 0.0003 & {-} \\
    \bottomrule
  \end{tabular}
\end{table}