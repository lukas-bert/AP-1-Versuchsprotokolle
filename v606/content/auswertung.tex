\section{Auswertung}
\label{sec:Auswertung}
Die in diesem Kapitel erwähnten Fehler genügen der Gaußschen Fehlerfortpflanzung beziehungsweise dem Standardfehler des Mittelwertes und wurden mit \textit{uncertainties}
\cite{uncertainties} in \textit{Python} berechnet. Die relative Abweichung eines Messwertes $x$ zu einem Theoriewert $x^*$ wird nach 
\begin{equation}
  \label{eqn:Delta_rel}
  \symup{\Delta}_\text{rel}(x) = \frac{|x^* - x|}{x^*}
\end{equation}
berechnet.

\subsection{Filterkurve des Selektivverstärkers}
\label{subsec:A_Filterkurve}
Im ersten Teil des Versuches wird die Filterkurve des Selektivverstärkers analysiert. Die Messwertepaare der Frequenz und Spannung werden in einem Diegramm gegeneinander
aufgetragen, wodurch sich eine Glockenkurve bildet, was in \autoref{fig:plot} zu sehen ist. Um die markanten Punkte der Glockenkurve zu verdeutlichen ist die Spannung 
relativ zur maximalen Spannung (Eingangsspannung) $U_\text{max} = \qty{8.5}{\volt}$ dargestellt. 
Die Glockenkurve hat die Gestalt
\begin{equation*}
  \label{eqn:Gaussglocke}
  f(x) = \symup{exp}\left(-\alpha(x-b)^2\right).
\end{equation*}
Der Parameter $b$ dient dabei zur Verschiebung der Glockenkurve entlang der $x$-Achse und entspricht dem Wert $\nu_0$ des Maximums. 
Durch eine Regression mittels \textit{scipy} \cite{scipy} ergeben sich die Parameter
\begin{align}
  \label{eqn:Parameter}
  \alpha &= \qty{2.28 +- 0.3}{\per\kilo\hertz\squared} & b &= \qty{21.75 +- 0.04}{\kilo\hertz}.
\end{align}
Es folgt sofort $\nu_0 = b = \qty{21.75 +- 0.04}{\kilo\hertz}$, die Frequenzen $\nu_-$ und $\nu_+$ können als Schnittpunkte der Funktion mit der Geraden $y = 1/\sqrt{2}$
grafisch ermittel werden. Dieses Vorgehen kann ebenfalls \autoref{fig:plot} entnommen werden. 
Es ergibt sich $\nu_- = \qty{21.36}{\kilo\hertz}$ und $\nu_+ = \qty{22.14}{\kilo\hertz}$. 
Durch Einsetzen in \autoref{eqn:Guete} kann der Wert der Güte berechnet werden.
Unter Beachtung der Gaußschen Fehlerfortpflanzung folgt
\begin{align*}
  Q &= \qty{27.89 +- 0.06}{} & \symup{\Delta}(Q) = \qty{39.45}{\percent}.
\end{align*}
Die relative Abweichung berechnet sich nach \autoref{eqn:Delta_rel}. Als Theoriewert der Güte wird der am Sinusgenerator einstellbare Wert $Q_\text{real} = 20$ verwendet.

\begin{figure}
  \centering
  \includegraphics{plot.pdf}
  \caption{Filterkurve des Selektivverstärkers. Erstellt mit \textit{matplotlib} \cite{matplotlib} unter Verwendung von \textit{scipy} \cite{scipy}.}
  \label{fig:plot}
\end{figure}

\subsection{Bestimmung der Suszeptibilitäten}
\label{subsec:A_Suszep}

\begin{table}
  \centering
  \caption{Experimentell ermittelte Suszeptibilitäten.}
  \label{tab:Chi_exp}
  \begin{tabular}{c S[table-format = 1.4] S S r c l}
    \toprule
      {Aus Widerständen} & {$\chi_1$} & {$\chi_2$} & {$\chi_3$} &
      \multicolumn{3}{c}{$\overline{\chi}_\text{exp}$}  \\
    \midrule
      {$\symup{Dy}_2\symup{O}_3$}                 & 0.0198 & 0.0224 & 0.0235 & 0.0219 & {$\pm$} & 0.0016 \\
      {$\symup{Gd}_2\symup{O}_3$}                 & 0.0268 & 0.0303 & 0.0319 & 0.0297 & {$\pm$} & 0.0021 \\
      {$\symup{C}_6\symup{O}_{12}\symup{Pr}_3$}   & 0.0042 & 0.0048 & 0.0050 & 0.0047 & {$\pm$} & 0.0003 \\
    \bottomrule
      {Aus Spannungen} & & & & & \\
    \bottomrule 
      {$\symup{Dy}_2\symup{O}_3$}                 & 0.0203 & 0.0228 & 0.0219 & 0.0217 & {$\pm$} & 0.0011 \\
      {$\symup{Gd}_2\symup{O}_3$}                 & 0.0275 & 0.0309 & 0.0298 & 0.0294 & {$\pm$} & 0.0014 \\
      {$\symup{C}_6\symup{O}_{12}\symup{Pr}_3$}   & 0.0043 & 0.0048 & 0.0047 & 0.0046 & {$\pm$} & 0.0002 \\
    \bottomrule 
  \end{tabular}
\end{table}
