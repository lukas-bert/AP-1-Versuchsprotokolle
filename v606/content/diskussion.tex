\section{Diskussion}
\label{sec:Diskussion}
Bei der Analyse der Filterkurve ergab sich aus den Messwerten eine Güte $Q = \num{27.89 +- 0.06}{}$. Am Sinusgenerator wurde eine Güte $Q_\text{real} = 20$
eingestellt, was bedeutet, dass das Messergebnis um $\symup{\Delta}(Q) = \qty{39.45}{\percent}$ von dem realen Wert abweicht.
Die relative Abweichung eines Messwertes $x$ zu einem Theoriewert $x^*$ wird nach 
\begin{equation}
  \label{eqn:Delta_rel}
  \symup{\Delta}_\text{rel}(x) = \frac{|x^* - x|}{x^*}
\end{equation}
berechnet.
Jedoch modelliert -- wie in 
\autoref{fig:plot} zu erkennen ist -- der Fit, mit welchem die Werte der Frequenzen $\nu_-$, $\nu_+$ und $\nu_0$ ermittelt wurden, nicht ideal den Verlauf
der Messwerte. Dies liegt der Tatsache zu Grunde, dass die Filterkurve einen sehr starken Anstieg verzeichnet und nicht genügend Messwerte in diesem Bereich
genommen werden konnten, obwohl bereits eine sehr kleine Schrittweite im Bereich des Maximums gewählt wurde. Möglicherweise könnte ein besseres Ergebnis erzielt 
werden, indem nur die Messwerte um das Maximum zur Interpolation verwendet würden. Die hohe Abweichung lässt sich hauptsächlich mit der Ungenauigkeit des Fits und 
den daraus resultierenden Schwierigkeiten in der Bestimmung der Frequenzen erklären. Eine genauere Bestimmung wäre nur mit einer besseren Auflösung 
der Frequenzen möglich.

Im Hauptteil des Versuches wurden die Suszeptibilitäten der verschiedenen Stoffe auf zwei Weisen ermittelt. Schon bei der Messung fiel auf, dass die Messwerte
der Spannungen für zwei Stoffe nur sehr schwierig von der Grundspannung der Brücke differenziert werden können. Des Weiteren wurden unerklärliche Sprünge der
Anzeige des Voltmeters bemerkt, die in der Größenordnung der Messwerte lagen. Die Messergebnisse werden in den Tabellen \ref{tab:Chi_abwU} und \ref{tab:Chi_abwR}
mit den Theoriewerten vergliechen. Zum Stoff $\symup{C}_6\symup{O}_{12}\symup{Pr}_3$ der Praseodymoxalat-Gruppe konnte kein Theoriewert bestimmt werden, da
keine Angaben zur Dichte des Materials und zu den quantenmechanischen Eigenschaften zu finden sind. Der Stoff $\symup{Nd}_2\symup{O}_3$ (Neodym(III)-oxid)
wurde nicht untersucht, da keine Veränderung der Messanzeigen beim Einführen der Probe in die Spule festgestellt werden konnte. Die experimentellen Werte
der Suszeptibilitäten der verbleibenden beiden Stoffe unterscheiden sich für die Ermittlung des Wertes aus den Spannungen respektive Widerständen kaum.
Dies ist ein Indiz dafür, dass beide Bestimmungsmethoden gleichwertig sind. Die relativen Abweichungen zu den Theoriewerten fallen für Dysprosium(III)-oxid
($\symup{Dy}_2\symup{O}_3$) relativ gering aus und liegen bei $\symup{\Delta}\chi_\text{Spannung} = \qty{13.74}{\percent}$ und
$\symup{\Delta}\chi_\text{Widerstand} = \qty{14.59}{\percent}$. Dies stellt eine akzeptable Abweichung im Rahmen der Messgenauigkeit und der zuvor getroffenen 
Näherungen dar. Die Suszeptibilitäten des Gadolinium(III)-oxids ($\symup{Gd}_2\symup{O}_3$) weichen um $\symup{\Delta}\chi_\text{Spannung} = \qty{115.37}{\percent}$
und $\symup{\Delta}\chi_\text{Widerstand} = \qty{113.27}{\percent}$ ab. Eine spezielle Ursache hierfür kann nicht ermittelt werden. 

Insgesamt ist die Bestimmung der Suszeptibilitäten mit dem verwendeten Verfahren inakkurat. Die Messwerte sind in einer ähnlichen Größenordnung wie potenzielle
Fehlersignale und es werden viele Näherungen getroffen, die sich in Summe stark auf die Messgenauigkeit auswirken. 
