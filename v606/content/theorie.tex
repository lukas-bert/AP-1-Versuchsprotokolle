\section{Zielsetzung}
\label{sec:ziel}
In diesem Versuch wird die magnetische Suszeptibilität paramagnetischer Stoffe bestimmt. Dies geschieht für drei verschiedene seltene Erden. Die Messung der dafür nötigen
Größen geschieht über eine Brückenschlatung.

\section{Theorie}
\label{sec:Theorie}

\subsection{Paramagnetismus}
\label{subsec:paramagnetismus}
Paramagnetismus ist eine eine Eigenschaft von Materie in einem Magnetfeld. Allerdings trifft diese Eigenschaft nicht auf jegliche Materie. Paramagnetische Stoffe haben ohne 
äußeres Magnetfeld keine eigene magnetische Ordnung. Außerdem ist bei solchen Stoffen das äußere Magnetfeld stärke als das innere. Daher werden paramagnetische Stoffe in ein
außen anliegendes Magentfeld hineingezogen. $\mu_{\text{r}}$ liegt bei paramagnetischen Stoffen $ < 1 $. Beispiele für paramagnetische Stoffe sind sogenannte 
\textit{seltene Erden}. Diese Stoffe werden in diesem Versuch verwendet. 

\subsection{Zusammenhang von Paramagnetismus und atomaren Drehimpuls}
\label{subsec:drehimpuls}
Damit Paramagnetismus auftreten kann, dürfen die Atome und Moleküle keine verschwindenden Drehimpulse haben. Wie in \autoref{subsec:paramagnetismus} bereits erwähnt, ist 
Paramagnetismus temperaturabhängig. Diese Abhängigkeit entsteht, wenn sich die magnetischen Momente der Moleküle ausrichten. Die magnetischen Momente sind mit den Drehimpulsen 
gekoppelt. Steigt nun die Temperatur eines paramagnetischen Stoffes, so steigt die kinetische Energie der Atome. Daher bewegen sich diese Stärker und stören so die Ausrichtung
der Momente. 

\subsection{Vom Drehimpuls zur Suszeptibilität}
\label{subsec:drehsus}
Es gibt drei Drehimpulse eines Teilchens die den Gesamtdrehimpuls festlegen. Der Banhdrehimpuls der Elektronenhülle, der Spin der Elektronen und Kerndrehimpuls. In schwachen 
Magnetfeldern hat der Kerndrehimpuls einen vernachlässigbar kleinen Effekt auf den Paramagnetismus. Der Gesamtdrehimpuls ergibt sich daher durch 
\begin{equation*}
    \vec{J} = \vec{L} + \vec{S}.
\end{equation*}
Dabei beschreiben $\vec{L}$ und $\vec{S}$ die Summe aller einzelnen Drehmomente der jeweiligen Teilchen. Durch die Quantenmechanik kann den Drehmomenten $\vec{L}$ und $\vec{S}$
ein magnetisches Moment zugeordnet werden. 
Die potentielle Energie der Einrichtungen der magnetischen Momente kann mittels des \textit{Landé-Faktors} $g_{\text{J}}$ und der Orientierungsquantenzahl $m$ durch 
\begin{equation}
    E_{\text{m}} = \mu_{\text{B}} g_{\text{J}} m
\end{equation} 
berechnet werden. Dabei ist $\mu_{\text{B}}$ das \textit{Bohrsche Mageton}. Nach Aufspaltung der Enrgieniveaus tritt der \textit{Zeeman-Effekt} auf. Um nun die Magnetisierung
berechnen zukönnen, muss die Häfugkeit bestimmter Orientierungen der magnetischen Moment herausgefunden werden. Diese verteilen sich gemäß der \textit{Boltzmann-Verteilung}
auf die  Energieunterniveaus. Nach Summation über die Enrgieniveaus kann ein mittleres magnetisches Moment berechnet werden. Diese lässt sich gemäß der \textit{Brillouin-Funktion}
berechnen. Diese Formel lässt sich allerdings im allgemeinen nicht analytisch lösen. Daher wird eine Näherung verwendet. Das Problem wird dabei in Raumtemperatur betrachtet und
es werden lediglich schwache Felder betrachtet. In dieser Näherung kann dann die Magnetisierung bestimmt werden.
\begin{equation*}
    M = \frac{1}{3}\mu_0 N g^2_{\text{J}} \mu^2_{\text{B}} \frac{J(J+1)B}{kT}
\end{equation*}
Daraus lässt sich dann gemäß Formel \eqref{eqn:Mchi} berechnen. 
\begin{equation}
    \label{eqn:chi1}
    \chi = \frac{1}{3}\mu_0 N g^2_{\text{J}} \mu^2_{\text{B}} \frac{J(J+1)}{kT}
\end{equation}   
Daraus folgt eine $\frac{1}{T}$-Abhängigkeit für $\chi$. Man nennt die Formel \eqref{eqn:chi1} \textit{Curiesches Gesetz}.

\subsection{Suszeptibilität paramagnetischer Substanzen}
\label{subsec:Berechnung}

Zunächst wird der Zusammenhang zwischen der Magnetfeldstärke $\vec{H}$ und der Suszeptibilität $\chi$ untersucht. Dabei hängt die Magnetfeldstärke $\vec{H}$ über 
\begin{equation*}
    \vec{B} = \mu_0 \vec{H}
\end{equation*}
zusammen. Dies gilt allerdings nur für ein homogenes Magnetfeld. Befindet sich Materie in einem Magnetfeld ändert sich der Zusammenhang um einen Summanden $\vec{M}$, welcher 
Magnetisierung gennant wird.
\begin{equation}
    \label{eqn:magnetfeld}
    \vec{B} = \mu_0 \vec{H} + \vec{M}
\end{equation}
Die Magnetisierung entsteht durch magnetische Momente der Atome in einer Substanz. Mittels der Suszeptibilität kann die Magnetisierung in Abhängigkeit von $\vec{H}$
ausgedrückt werden. 
\begin{equation}
    \label{eqn:Mchi}
    \vec{M} = \mu_0 \chi \vec{H}
\end{equation}
Hierbei besteht kein lineare zusammenhang zwischen $\vec{M}$ und $\vec{H}$, da $\chi$ sowohl von der Temperatur der Probe, sowie von $\vec{H}$ selbst abhängt.

\subsection{Suszeptibilität Selterner-Erd-Verbindungen}
\label{subsec:suzepseltenererden}
Ionen seltener Erden weisen einen starken Paramagnetismus auf. Wie in \autoref{subsec:drehimpuls} erklärt, folgt aus dieser Eigenschaft, dass die Elektronenhüllen dieser Stoffe
große Drehimpulse haben. Diese entstehen durch die 4f-Elektronen in der Elektronenhülle. 4f-Elektronen treten erst ab Ordnungszahlen $\geq 58$. Die 4f-Elektronen liegen weit
innerhalb der 6s-Schale, wodurch auch die Paramagnetische Eigenschaft seltener Erden erklärt werden kann. Mit den Hundschen Regeln kann der Gesamtdrehimpuls der 4f-Schale
bestimmt werden. Ist die Schale weniger als zur Hälfte gefüllt lässt sich der Gesamtdrehimpuls gemäß $\vec{J} = \vec{L} - \vec{S}$ berechnet werden. Ist die Schale dahingegen 
mehr als zur Hälfte gefüllt berechnet sich der Gesamtdrehimpuls nach $\vec{J} = \vec{L} - \vec{S}$. Dabei gilt Jeweils $\vec{L} = \sum \vec{s}_i$ und $\vec{S} = \sum \vec{l}_i$. 

\subsection{Beschreibung einer Messapparatur der Suszeptibilität}
\label{subsec:Messapparatur}
Für eine Messapparatur der Suszeptibilität ist es nötig ein Magnetfeld zu erzeugen. Dies geschieht häufig über eine Lange Spule, da das Magnetfeld im inneren der Spule sehr 
homogen ist. Um mit einer solchen Spule zu rechnen ist die Induktivität eine relevante Größe.
Für eine Spule in einem Vakuum gilt für die Induktivität die Formel $L = \mu_0 \frac{n^2}{l}F$. Dabei beschreibt $l$ die Länge der Spule und $F$ den Querschnitt der Spule.
Um die Suszeptibilität eines Materials zu bestimmen wird ein Stoff nun in das Magnetfeld der Spule eingeführt. Geschieht dies, so ändert sich die Induktivität der Spule. 
Da in der Realität eine Spule nicht vollständig mit einer Probe gefüllt wird, kann man die Formel der Induktivität anpassen.
\begin{equation}
    \label{eqn:L_M}
    L_{\text{M}} = \mu_0 \frac{n^2 Q}{l} + \chi \mu_0 \frac{n^2 Q}{l}
\end{equation} 
