\section{Auswertung}
\label{sec:Auswertung}
Die in diesem Kapitel genannten Standardfehler des Mittelwertes ergeben sich nach
\begin{equation*}
  \label{eqn:MW-Fehler}
  \sigma(x) = \sqrt{\frac{1}{n(n-1)} \sum_i (x_i - \overline{x})^2}.
\end{equation*}
Daraus resultierende Unsicherheiten genügen der Gaußschen Fehlerfortpflanzung.
Die Fehlerrechnung wird in \textit{Python} unter Verwendung des Paketes \textit{uncertainties} \cite{uncertainties} durchgeführt.

\subsection{Bestimmung der Wellenlänge des Lasers}
\label{subsec:A_Wellenlaenge}
Zur experimentellen Bestimmung der Wellenlänge des Lasers werden die in \autoref{tab:Messwerte} aufgeführten Messwerte der Zählraten verwendet.
Aus diesen ergibt sich der Mittelwert $z_1 = \num{2999+-24}$. 
Die Stellschraube des Spiegels wurde um jeweils $\qty{5}{\milli\metre}$ gedreht. Durch Multiplikation mit dem Untersetzungsverhältnis $u = 1:5,046$ ergibt sich die
Armlängenänderung $\symup{\Delta}d = \qty{0.991}{\milli\metre}$.
Nach Einsetzen der Werte in \autoref{eqn:lambda} folgt die Wellenlänge $\lambda_\text{exp} = \qty{661(5)}{\nano\metre}$. Die reale Wellenlänge des Lasers ist mit 
$\lambda = \qty{635}{\nano\metre}$ angegeben.
\begin{table}
  \centering
  \caption{Messwerte der Zählraten zur Bestimmung der Laserwellenlänge ($N_\lambda$) und des Brechungsindex ($N_n$).}
  \label{tab:Messwerte}
  \begin{tabular}{c c}
    \toprule
      {$N_\lambda$} & {$N_n$} \\
      \midrule
      3019 & 27 \\
      3011 & 25 \\
      2956 & 27 \\
      3007 & 25 \\
      3013 & 32 \\
      2948 & 27 \\
      3004 & 40 \\
      3013 & 24 \\
      3008 & 29 \\
      3014 & 25 \\
           & 31 \\
           & 26 \\
    \bottomrule
  \end{tabular}
\end{table}

\subsection{Bestimmung des Brechungsindex von Luft bei Normalbedingungen}
\label{subsec:A_Index_Luft}
Die Messwerte zur Bestimmung des Brechungsindex von Luft sind ebenfalls in \autoref{tab:Messwerte} aufgeführt. Der Mittelwert dieser ergibt sich zu $z_2 = \num{28(4)}$.
Die Länge der Messzelle, in welcher der Unterdruck erzeugt wird, lautet $b = \qty{5}{\centi\metre}$. Mit der Laserwellenlänge $\lambda = \qty{635}{\nano\metre}$ folgt aus
\autoref{eqn:Delta_n} die Änderung des Brechungsindex $\symup{\Delta}n = \num{0.000179+-0.000027}$. Diese kann wiederum in \autoref{eqn:Brechungsindex} eingesetzt werden um den
Brechungsindex bei Normalbedingungen ($p_0 = \qty{1013.25}{\hecto\pascal}$, $T_0 = \qty{273.15}{\kelvin}$) zu erhalten. Die Messtemperatur $T$ wird als 
$T = \qty{295.15}{\kelvin} = \qty{22}{\celsius}$ angenommen, ebenso wird $p = p_0$ für den Luftdruck während der Messung verwendet. Die Druckänderung wurde im Experiment so 
gewählt, dass immer der gleiche Druck $p' = \qty{41325}{\pascal}$ erreicht wurde (Unterdruck von $-\qty{0.6}{\bar}$). Mit diesen Werten lässt sich aus \autoref{eqn:Brechungsindex} 
der Brechungsindex $n_\text{Luft} = \num{1.000326+-0.00005}$ ermitteln.
