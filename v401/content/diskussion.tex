\section{Diskussion}
\label{sec:Diskussion}
Im ersten Teil des Versuches wurde mit Hilfe eines Michelson-Interferometers die Wellenlänge eines Laser bestimmt. Dabei ergaben sich die Schwierigkeiten, dass die Photokathode 
nicht von der Umgebung abgeschirmt gewesen ist. Daher können äußere Schatten zu einer erhöhten Zählrate führen. Außerdem kann es geschehen, dass aufgrund der 
Motorgeschwindigkeit die Photokathode nicht alle Maxima registriert und somit die Zählrate leicht abweicht. 
Die Wellenlänge des Lasers wurde zu $\lambda_\text{exp} = \qty{661 +- 5}{\nano\metre}$ bestimmt. Nach Angabe des Herstellers soll der Laser eine Wellenlänge von
$\lambda_\text{theo} = \qty{635}{\nano\metre}$ haben.
Daher ergibt sich gemäß 
\begin{equation*}
    \Delta\lambda = 100\frac{\lvert \lambda_\text{exp} - \lambda_\text{theo}\rvert}{\lambda_\text{theo}}
\end{equation*}
eine Abweichung der Wellenlänge von $\Delta\lambda = 4.05\%$.

Im zweiten Teil des Versuches sollte mit dem selben Aufbau der Brechungsindex von Luft bestimmt werden. Bei dieser Messung liegen mögliche Quellen für Abweichungen in der Dichte
der Apparatur. Außerdem kann durch händisches Pumpen nicht bei jeder Messung der gleiche Druck eingestellt werden. Ebenfalls gelten die Formeln zur Berechnung des Brechungsindex
nur genähert für ideale Gase unter Normalbedingungen, weshalb eine Abweichung auftreten muss. Hinzu kommen ebenfalls die selben Fehlerquellen des ersten 
Teils des Versuches.
Der Brechungsindex von Luft wurde mit diesem Experiment zu $n_\text{Luft} = 1.000474 \pm 0.000072$ bestimmt. Da kein Brechungsindex mit den verwendeten Bedinungen in der 
Literatur gefunden werden kann, lässt sich lediglich sagen, dass der experimentell bestimmte Wert in eine realistischen Größenordnung liegt und somit als qualitativ
angenommen werden kann.

Die geringe Abweichung im ersten Versuch deutet auf die hohe Präzesion des Michelson-Interferometers hin. Die Abweichung liegen vermutlich eher an der Photokathode und den 
äußeren Einflüssen. 