\section{Zielsetzung}
\label{sec:Ziel}
In diesem Versuch soll die Wellenlänge eines Lasers durch ein \textit{Michelson-Interferometer} bestimmt werden. Dazu wird die Interferenzeigenschaft von monochromatischem 
Licht verwendet. Außerdem soll mit dem selben Laser der Brechungsindex von Luft bestimmt werden.

\section{Theorie}
\label{sec:Theorie}
Um das Michelson-Interferometer zu verstehen, werden zunächst grundlegende Eigenschaften von Licht erläutert. Dabei wird hauptsächlich auf die Eigenschaft der Interferenz und
der Kohärenz eingegangen. 
\subsection{Interferenz}
\label{subsec:Interferenz}
Licht ist eine elektromagnetische Welle. Daher kann die Ausbreitung von Licht durch die Maxwell-Gleichungen beschrieben werden. Aus dieser Beschreibung folgt auch, dass Licht
zu einem Teil durch ein $\vec{E}$-Feld beschrieben werden kann. Da dieses bei Licht allerdings so schwach ist, dass es nicht gemessen werden kann, muss eine andere Größe für die
Beschreibung von Licht gefunden werden. Dazu wird die Intensität verwendet. Die Intensität folgt aus den Maxwell-Gleichungen und ist durch 
\begin{equation*}
    I = \text{const}\lvert \vec{E} \rvert^2
\end{equation*}
gegeben. 
Nun ist Licht im Allgemeinen allerdings keine einzelne Welle, sondern eine mehrfach überlagerte Wellenkombination. Überlagerung bei elektromagnetischen Wellen bedeutet, dass 
zwei Wellen im gleichen Raum übereinander liegen. Da die Beschreibung von Licht aus den linearen Maxwell-Differentialgleichungen folgt, gilt für Lichtwellen das Prinzip der
Superposition. Das bedeutet, dass die $\vec{E}$-Felder der einzelnen Wellen addiert das $\vec{E}$-Feld der gesamten überlagerten Welle ergibt. Die Intensität einer 
überlagerten Welle kann dann gemäß 
\begin{equation}
    I_\text{Ges} = 2 \text{const}\vec{E}_0^2(1 + \cos(\delta_2 - \delta_1))
\end{equation}
berechnet werden. Der hier auftretende Cosinus-Term wird Interferenzterm genannt. Dieser Term beschreibt das Interferenzverhalten der überlagerten Wellen. Aus den Eigenschaften
der Cosinus-Funktion folgt, dass der Interferenzterm einen Beitrag liefern kann, welcher größer als der Mittelwert selbst ist. Außerdem kann der Interferenzterm an den 
Nullstellen des Cosinus auch verschwinden. 

\subsection{Kohärenz}
\label{subsec:Kohärenz}