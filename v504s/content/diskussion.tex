\section{Diskussion}
\label{sec:Diskussion}
Im folgendem werden ale Abweichungen gemäß 
\begin{equation*}
    \mathrm{\Delta}(x) = \frac{\lvert x - x^*\rvert}{x^*}
\end{equation*}
berechnet.

Zunächst wurden die Kennlinien zur Hochvakuumdiode aufgenommen. Diese haben bei niederen Strömen des untersuchten Stromstärkeintervalls eine zu erwartende Form, weshalb die 
Messung eine genügende Qualität hat. Allerdings gilt dies nicht mehr für eine Stromstärke von $\qty{2.4}{\ampere}$. Dazu wurde kein Sättigungsstrom erreicht, weshalb alle 
folgenden Rechnungen nur eine verminderte Genauigkeit haben, da nur der maximal gemessene Strom verwendet werden kann. 

Dann wurde der Gültigkeitsbereich des Langmuir-Schottkyschen Raumladungsgesetzes untersucht. Dabei konnte mittels linearer Regression ein Exponentialfaktor von 
$m = \num{1.33 +- 0.01}$ bestimmt werden. Der Formel \eqref{eqn:langmuirraumladung} kann der Theoriewert $m = \frac{3}{2}$ entnommen werden. Es ergibt sich eine Abweichung von 
$\mathrm\Delta m = \qty{11.3 +- 0.7}{\percent}$. Im Rahmen der Messunsicherheit ist diese Abweichung akzeptabel. Ein Grund für diese Abweichung ist, dass die Grenzen des 
Raumladungsgebietes nicht genau bestimmt werden können. 

Danach wurde das Anlaufstromgebiet mit einer Heizstromstärke von $\qty{2.4}{\ampere}$ untersucht. Draus ergab sich eine Kathodentemperatur von $T = \qty{2433.06 +- 237.35}{\kelvin}$.
Da diese sehr von den umliegenden Bedingungen abhängt gibt es keinen anerkannten Literaturwert. Im Vergleich zur Bestimmung über das Leistungsverhältnis, wobei sich eine
Temperatur von $T = \qty{2156.73}{\kelvin}$, ergibt sich eine Abweichung von $\qty{13 +- 11}{\percent}$. Aus dieser Abweichung lässt sich schließen, dass beide Methoden ähnlicher Qualität 
sind. Allerdings folgt aus der großen Unsicherheit, dass der Fehler des Sättigungsstromes einen großen Anteil hat, weshalb nicht von einer idealen Temperaturbestimmung 
gesprochen werden kann. 

Zuletzt wurde die Austrittsarbeit von Wolfram untersucht. Dabei ergab sich eine mittlere Austrittsarbeit von $\overline{\Phi} = \qty{4.68+- 0.10}{\electronvolt}$. 
In der Literatur wird eine Austrittsarbeit für Wolfram von $\Phi_\text{lit} = \qty{4.55}{\electronvolt}$\cite{Ingenieurwissen} verwendet. Der experimentell bestimmte Wert weicht um 
$\mathrm{\Delta}\Phi = \num{2.9}\%$ von dem Literaturwert ab. Diese Abweichung ist sehr gering. Trotzdem trägt zu der Abweichung bei, dass die Austrittsarbeit von den
lokal vorliegenden Bedingungen und der Qualität der Wolframprobe abhängt.

Zusammenfassend konnte das Ziel des Versuches erfüllt werden und die Messungen mit eine hinreichenden Genauigkeit durchgeführt werden. 
