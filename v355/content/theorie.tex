\section{Ziel}
\label{sec:Ziel}
In diesem Versuch sollen gekoppelte Schwingkreise auf ihren Energieübergang untersucht werden. Dazu werden die Frequenzabhängigkeiten des Stroms, sowie die Fundamentalschwingungen mittles der Lissajous-Figuren,
betrachtet.
\section{Theorie}
\label{sec:Theorie}
Zu Beginn wird disskutiert wann ein System gekoppelt genannt wird, da ein solches System in diesem Versuch untersucht wird. Von einem gekoppelten System wird gesprochen, wenn es aus zwei Untersystemen
besteht, welche im gegenseitigem Energieaustausch stehen. Ein leichtes Beispiel für ein gekoppeltes System sind zwei gekoppelte Pendel. Die beiden Einzelpendel entsprechen dabei den Untersystemen.
Diese werden dann durch eine Feder gekoppelt, wodurch sie im gegenseitigen Energieaustausch stehen. In diesem Versuch werden anstatt gekoppelter Pendel kapazitiv gekoppelte Schwingkreise betrachtet.
Bei solchen Systemen ist es häufig von Interesse das Verhalten unter äußerer Anregung zu untersuchen.
\subsection{Kapazitiv gekoppelte Schwingkreise}
\label{T_KgS}
Nun wird ein Schwingkreise wie in \autoref{fig:T_skgS} betrachtet. Die Kopplung findet über den Kondensator $C_k$ statt, wodurch sich die beiden Schwingkreise das elektrische Feld dieses Kondensators
teilen. Durch die kirchhoffschen Regeln lassen sich zwei unabhängige Differentialgleichungen aufstellen. Die Lösung dieser beschreiben den überlagerten Stromverlauf, sowie den Verlauf der Differenz.
Die Summe der beiden Ströme $I_1$ und $I_2$ wird durch 
\begin{equation}
    \label{eqn:T_Iplus}
    \left(I_1 + I_2\right)\left(t\right) = \left(I_{1_0} + I_{2_0}\right)\frac{\text{cos}\left(t\right)}{\sqrt{LC}}
\end{equation}
beschrieben. Die Lösung der Differenz lautet
\begin{equation}
    \label{eqn:T_Iminus}
    \left(I_1 - I_2\right)\left(t\right) = \left(I_{1_0} - I_{2_0}\right)\text{cos}\left(\frac{t}{\sqrt{L\left(\frac{1}{C}+\frac{2}{C_k}\right)^{-1}}}\right)
\end{equation}.
Aus \autoref{eqn:T_Iplus} und \autoref{eqn:T_Iminus} lassen sich die Schwinungsfrequenzen $\nu^+$ und $\nu^-$ aufstellen.
\begin{equation}
    \label{eqn:T_nup}
    \nu^+ = \frac{1}{2\pi\sqrt{LC}}
\end{equation}
\begin{equation}
    \label{T_num}
    \nu^- =  \frac{1}{2\pi\sqrt{L\left(\frac{1}{C}+\frac{2}{C_k}\right)^{-1}}}
\end{equation}
Des Weiteren lassen sich aus \autoref{eqn:T_Iplus} und \autoref{eqn:T_Iminus} die Stomverläufe $I_1(t)$ und $I_2(t)$ aufstellen. Durch Addition und Subtraktion der beiden genannten Gleichungen 
folgt 
\begin{equation}
    \label{T_I}
    I_{1,2}(t) = \frac{1}{2}\left(I_{1_0} + I_{2_0}\right)\text{cos}\left(2\pi\nu^+ t\right) - \frac{1}{2}\left(I_{1_0} - I_{2_0}\right)\text{cos}\left(2\pi\nu^- t\right)
\end{equation}
Betrachtet man nun den Fall der gleichsinnigen Schwingung $\left(I_{1_0} = I_{2_0}\right)$ fällt der Differenzteil der \autoref{T_I} weg und die Oszillatoren schwingen gleichphasig mit der Frequenz 
$\nu^+$.
Für den gegensinnigen Schwingfall $\left(I_{1_0} = -I_{2_0}\right)$ ergibt sich, dass die Oszillatoren gegenphasig mit der Frequenz $\nu^-$ schwingen.
Beim Schwebungsfall gilt zur Zeit $t = 0$ $I_{1_0}\neq 0$ und $I_{2_0} = 0$. Daraus ergit sich für \autoref{T_I} 
\begin{equation*}
    I_1(t) = \frac{1}{2} I_{1_0}\text{cos}\left(\frac{1}{2}(\omega^+ + \omega^-)t\right) \text{cos}\left(\frac{1}{2}(\omega^+ - \omega^-)t\right)
\end{equation*}
und
\begin{equation*}
    I_2(t) = \frac{1}{2} I_{1_0}\text{sin}\left(\frac{1}{2}(\omega^+ + \omega^-)t\right) \text{sin}\left(\frac{1}{2}(\omega^+ - \omega^-)t\right)
\end{equation*}
.
Die dazugehörige Schwebungsfrequenz $\nu^s$ ist durch 
\begin{equation}
    \label{T_Schwebung}
    \nu^s = \nu^- -\nu^+
\end{equation}
gegeben. Der Strom der Oszillatoren schwingt dann zwischen einer Anfangsamplitude $I_{{1,2}_0}$ und der $0$.
Die Anzahl der maximalen positiven oder negativen Auslenkungen innerhalb einer Schwebung kann durch
\begin{equation}
    \label{T_n}
    n = \frac{\nu^+ + \nu^-}{2\left(\nu^- - \nu^+\right)} 
\end{equation}
bestimmt werden.  
\begin{figure}
    \centering
    \caption{In dieser Abbildung ist die Skizze eines kapazitiv gekoppelten Schwingkreises zu sehen. \cite{v355}}
    \label{fig:T_skgS}
    \includegraphics[width = 0.6\textwidth]{content/SkizzegekoppelterSchwingkreis.PNG}
\end{figure}
