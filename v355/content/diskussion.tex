\section{Diskussion}
\label{sec:Diskussion}
Zunächst fällt auf, dass bei der Abstimmung der beiden Schwingkreise auf ihre Resonanzfrequenz die Abbweichung von $0.46\%$ sehr gering ist. Diese gute Einstellung ist ein wichtiges Qualitätsmerkmal
für Messungen, welche im weiteren noch diskutiert werden. Nach dieser Einstellung wurde die Schwebungsfrequenz untersucht. Bei dieser Messung ist zu erkennen, dass die Abweichungen mit $0.8\%$ für
den größten Widerstand noch sehr genau ist. Allerdings werden die Abweichungen mit sinkendem Widerstand immer größer. Die Abweichungen steigen bis auf $30.3\%$ für den geringsten Widerstand. Dies 
kann zum einen an den in der Versuchsaleitung \cite{v355} beschriebenen systematischen Abweichungen liegen. Allerdings ist so ein experimenteller Verlauf zu erwarten, da mit geringerem Widerstand 
auch die Qualität des Schwingvorgangs abnimmt. Zunächst wurden die Fundamentalfrequenzen über die Lissajous-Figuren bestimmt. Bei der Frequenz $\nu^+$ fällt sofort auf, dass der experimentelle Wert
bei fallendem $C_k$ ansteigt. Nach Theorie sollte dieser allerdings Konstant seien. Dies könnte an einem systematischen Fehler liegen, da sich eine Gerade zwischen der Resonanzfrequenz und der Frequenz
$\nu^-$ gebildet hat. Der Grund für das Auftreten dieser ist ungewiss. Eine weitere Fehlerquelle ist das händische Einstellen der Gerade. In einem kleinen Frequenzbereich um diese Gerade kann nur geraten werden 
welche die beste Frequenz ist. Außerdem schwankt die Anzeigen des Oszilloskops dauerhaft, weshalb man die Frequenz nicht mit voller Sicherheit ablesen kann. Trotzdem liegen alle Werte
für $\nu^+$, außer der mit dem kleinsten Widerstand, in eine akzeptabelen Bereich von Abweichungen zwischen $2.29\%$ und $9.37\%$. Die experimentellen Werte für die Frequenz $\nu^-$ sind sehr 
qualitativ, da alle Abweichungen kleiner als $0.42\%$ sind. Zuletzt wurden die Fundamentalfrequenzen mittels der Sweep-Funktion bestimmt. Durch diese Methode werde die Frequenz $\nu^+$ mit einer 
Abweichungen von $3.34\%$ bestimmt. Es existiert nur eine Abweichung, da ${t^+}_{exp}$ Konstant war. Dieser Wert ist im Mittel besser als die Methode der Lissajous-Figuren, allerdings ist für die gorßen 
$C_k$ die Lissajousmethode genauer. Für die Frequenz $\nu^-$ liegen die Abweichungen in einem Bereich von $1.64\%$ bis $4.46\%$. Die Abweichungen liegen alle in einem relativ kleinem Intervall. 
Dies spricht für die Qualität der Werte. Für den Widerstand $1.01\unit{\ohm}$ war die Auflösung des Oszilloskops nicht ausreichend um den Wert abzulesen. Allerdings gilt auch hier, dass die Lissajousmethode 
genauer ist. Die Abweichungen sind unter anderem durch die Sweep-Funktion zu erklären. Diese wurde ursprünglich auf $20 \unit{\kilo\hertz}$ und $50 \unit{\kilo\hertz}$ eingestellt. 
Diese Einstellung änderte sich allerdings im Verlauf der Messungen. Außerdem ist das optische Ablesen am Bildschrim des Oszilloskops grundlegend ungenau.


Zusammenfassend bestätigen dennoch beide Methoden die Theorie, da die relativen Abweichungen alle im Rahmen der Messunsichherheit des Expermientes liegen. 