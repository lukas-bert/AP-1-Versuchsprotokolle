\section{Diskussion}
\label{sec:Diskussion}
Im ersten Teil des Versuches wurde die Absorptionskurve von $\gamma$\,-Strahlung für Blei und Zink als Aborberschicht aufgenommen.
Zu Blei wurde aus den Messwerten die Absorptionskonstante 
$\mu_\text{Pb} = \qty{82.7 +- 3.1}{\metre^{-1}}$ bestimmt. Im Vegleich mit der Comptonabsorptionskonstante $\mu_\text{com, Pb} = \qty{69.09}{\metre^{-1}}$ fällt ein Differenz der Werte auf.
Dies bedeutet, dass der Comptoneffekt zwar eine übergeordnete Rolle in der vorliegenden Absorption vertritt, er aber nicht der einzige auftretende Absorptionseffekt ist.
Die aus den Messwerten bestimmte Nullzählrate lautet $N_{0, \text{Pb}} = \qty{105 +- 6}{\second^{-1}}$. Diese weicht um $\symup{\Delta N_{0, \text{Pb}}} = \num{2 +- 6}\%$
von der Nullmessung ab, welche in \autoref{sec:Durchführung} beschrieben wurde.

Aus der Messreihe zu Zink ergibt sich die Absorptionskonstante $\mu_{\text{Zn}} = \qty{37.8 +- 1.9}{\metre^{-1}}$. Hierzu lautet die Comptonabsorptionskonstante 
$\mu_{\text{com, Zn}} = \qty{50.61}{\metre^{-1}}$. Die experimentelle Absorptionskonstante liegt unter dem Wert der Comptonabsorptionskonstante zu Zink. Daraus wird gedeutet, 
dass der Comptoneffekt der überwiegende Absorptionseffekt ist. Allerdings dürfte in diesem Fall die gesamte Absorptionskonstante nicht kleiner als die Comptonabsorptionskonstante 
sein, weshalb auf eine statistische und oder systematische Abweichung der Messwerte geschlossen werden muss. 
Die Nullzählrate liegt bei $N_{0, \text{Zn}} = \qty{111.1 +- 2.5}{\second^{-1}}$ und weicht somit um $\symup{\Delta N_{0, \text{Zn}}} = \num{8.2 +- 3.1}\%$ von der Nullmessung ab.

Die Abweichungen der Zählraten sind keine Abweichung von Literaturwerten, sondern von einer statistisch fehlerbehafteten Messung, weshalb von dieser nicht direkt auf die
Qualität der Messung geschlossen werden kann. Dennoch kann aus dem Fit eine relativ kleine Unsicherheit gewonnen werden. Ebenso unterliegen die Absorptionskonstanten 
statistischen Fehlern. Jedoch kann angemerkt werden, dass der verwendete Cs-137-Strahler lediglich $\gamma$\,-Quanten abstrahlt, welche eine Energie von 
$E_\gamma = \qty{662}{\kilo\electronvolt}$ \cite{physikalischesGrundpraktikum} haben. Daher ist der Paarbildungseffekt in der Deutung der Absorptionskonstanten auszuschließen.

Im zweiten Teil dieses Versuches wurde die Absorptionskurve von $\beta$\,-Strahlung, aus einer Technezium-99 Strahlungsquelle aufgenommen. Aus der durchgeführten 
Regression wurde die maximale Strahlungsreichweite in einer Aluminiumabsorberschicht bestimmt. Diese lautet $R_\text{max} = \qty{0.092 +- 0.004}{\gram\per\centi\metre\squared}$. 
Die dazugehörige maximale Energie $E_\text{max}$ der abgestrahlten $\beta$\,-Teilchen ergibt sich zu $E_\text{max} = \qty{0.325 +- 0.009}{\mega\electronvolt}$. Der Literaturwert
der maximalen Energie der $\beta$\,-Telchen bei diesem Übergang lautet $E_{\text{max, Lit}} = \qty{0.294}{\mega\electronvolt}$. Daher ergibt sich eine Abweichung von
$\Delta E_\text{max} = \num{10.7 +- 2.9}\%$. 

Zusammenfassend genügen alle Abweichungen der statistischen Unsicherheit, welche bei diesem Versuchsaufbau gegeben ist. Die Präzision der Messung ist ausreichend, um 
von einer qualitativen Messung zu sprechen.