\section{Diskussion}
\label{sec:Diskussion}
Im ersten Teil dieses Versuches wurde die Absorptionskurve von $\gamma$\,-Strahlung für Blei und Zink als Aborberschicht aufgenommen. Aus einer linaren Ausgleichsrechung der
aufgenommenen Absorptionskurven wird für beide Materialien die Absorptionskonstante und die Nullzählrate bestimmt. Zu Blei wurde aus den Messwerten die Absorptionskonstante 
$\mu_\text{Pb} = \qty{82.7 +- 3.1}{\per\metre}$ bestimmt. Im Vegleich mit der Comptonabsorptionskonstante $\mu_\text{com, Pb} = \qty{69.09}{\per\metre}$ fällt ein Differenz der Werte auf.
Daraus ist zu deuten, dass der Comptoneffekt zwar eine entscheidene Rolle in vorliegenden Absorption spielt, er aber nicht der einzige auftretende Absorptionseffekt ist.
Die aus den Messwerten bestimmte Nullzälrate wurde zu $N_{0, \text{Pb}} = \qty{105 +- 6}{\per\second}$ bestimmt. Diese weicht um $\symup{\Delta N_{0, \text{Pb}}} = \num{2 +- 6}\%$
von der Nullmessung ab, welche in der \autoref{sec:Durchführung} beschrieben wurde.

Aus der Messreihe zu Zink ergibt sich die Absorptionskonstante $\mu_{\text{Zn}} = \qty{37.8 +- 1.9}{\per\metre}$. Hierzu lautet die Comptonabsorptionskonstante 
$\mu_{\text{com, Zn}} = \qty{50.61}{\per\metre}$. Die experimentelle Absorptionskonstante liegt unter dem Wert der Comptonabsorptionskonstante zu Zink. Daraus wird gedeutet, 
dass der Comptoneffekt die überwiegende Absorptionseffekt ist. Allerdings müsste in diesem Fall die gesamte Absorptionskonstante ungefähr gleich der Comptonabsorptionskonstante 
sein, weshalb auf einen statistische Abweichung der Messwerte geschlossen werden kann. 
Die Nullzählrate liegt bei $N_{0, \text{Zn}} = \qty{111.1 +- 2.5}{\per\second}$ und weicht somit um $\symup{\Delta N_{0, \text{Zn}}} = \num{8.2 +- 3.1}\%$ von der Nullmessung ab.

Die Abweichungen der Zählraten sind keine Abweichung von Literaturwerten, sonder von einer statistisch fehlerbehafteten Messung, weshalb von dieser nicht auf die Qualität der
Messung geschlossen werden kann. Ebenso unterliegen die Absorptionskonstanten statistischen Fehlern. 
Jedoch kann angemerkt werden, dass der verwendete Cs-137-Strahler lediglich $\gamma$\,-Quanten abstrahlt, welche eine Energie von $E_\gamma = \qty{662}{\kilo\electronvolt}$
\cite{physikalischesGrundpraktikum} haben. Daher ist der Paarbildungseffekt in der Deutung der Absorptionskonstanten auszuschließen.

Im zweiten Teil dieses Versuches wurde dann die Absorptionskurve von $\beta$\,-Strahlung, aus einer Technezium-99 Strahlungsquelle, aufgenommen. Aus der durchgeführten 
Regression wurde die maximale Strahlungsreichweite in einer Aluminiumabsorberschicht bestimmt. Diese lautet $R_\text{max} = \qty{0.092 +- 0.004}{\gram\per\centi\metre\squared}$. 
Die dazugehörige maximale Energie $E_\text{max}$ der abgestrahlten $\beta$\,-Telchen ergibt sich zu $E_\text{max} = \qty{0.325 +- 0.009}{\mega\electronvolt}$. Der Literaturwert
der maximalen Energie der $\beta$\,-Telchen bei diesem Übergang lautet $E_{\text{max, Lit}} = \qty{0.294}{\mega\electronvolt}$. Daher ergibt sich eine Abweichung von
$\Delta E_\text{max} = \num{10.7 +- 2.9}\%$. 

Zusammenfassend genügen alle Abweichungen der statistischen Unsicherheit, welche bei diesem Versuchsaufbau gegeben ist. Die Präzesion der Messung ist ausreichend um 
von einer qualitativen Messung zu sprechen.