\section{Auswertung}
\label{sec:Auswertung}
Die Fehlerrechnung dieses Kapitels genügt der gaußschen Fehlerfortpflanzung
\begin{equation*}
  \label{eqn:Gauss}
  \Delta F = \sqrt{\sum_i\left(\frac{\symup{d}F}{\symup{d}y_i}\Delta y_i \right)^2}.
\end{equation*}
Die Standardfehler des Mittelwertes ergeben sich nach
\begin{equation*}
  \label{eqn:MW-Fehler}
  \sigma(x) = \sqrt{\frac{1}{n(n-1)} \sum_i (x_i - \overline{x})^2}.
\end{equation*}
Die Fehlerrechnung wird in \textit{Python} unter Verwendung des Paketes \textit{uncertainties} \cite{uncertainties} durchgeführt.

\subsection{\texorpdfstring{Bestimmung der Absorptionskoeffizienten verschiedener Stoffe bei $\gamma$-Strahlung}
{Bestimmung der Absorptionskoeffizienten verschiedener Stoffe bei Gamma-Strahlung}}
\label{subsec:A_Koeffizienten}

\begin{table}
  \centering
  \caption{Messwerte der Absorption von $\gamma$-Strahlung eines Cäsium-137-Strahlers. Es werden Blei und Zink als Absorber verwendet. $d$ beschreibt die Dicke der Absorberschicht und
  $N_t$ die Zählraten des \textit{GMZ} während der Zeit $t$.}
  \label{tab:Mess_gamma}
  \begin{tabular}{S[table-format = 2.1] S[table-format = 5.0] S[table-format = 3.0] | S[table-format = 2.0] S[table-format = 5.0] S[table-format = 3.0]}
    \toprule
    \multicolumn{3}{c}{Blei} & \multicolumn{3}{c}{Zink} \\
      \midrule
      {$d \mathbin{/} \unit{\milli\metre}$} & {$N_t$} & {$t \mathbin{/} \unit{\second}$} & {$d \mathbin{/} \unit{\milli\metre}$} & {$N_t$} & {$t \mathbin{/} \unit{\second}$} \\
      \midrule
       0   &  3081 &  30 &  0 &  3081 &  30 \\
       1.2 & 10470 & 100 &  2 & 10602 & 100 \\
       2.4 &  9246 & 100 &  4 &  9766 & 100 \\
       4.7 &  7415 & 100 &  6 &  8812 & 100 \\
       7.4 &  6229 & 100 &  8 &  8462 & 100 \\
      10.3 &  6297 & 150 & 10 &  8034 & 100 \\
      13.8 &  4481 & 150 & 12 &  7018 & 100 \\
      17.7 &  3532 & 150 & 14 &  6549 & 100 \\
      20.0 &  2460 & 150 & 16 &  5925 & 100 \\
      30.3 &  1141 & 150 & 18 &  5727 & 100 \\
      40.8 &   892 & 200 & 20 &  5020 & 100 \\
    \bottomrule
  \end{tabular}
\end{table}

Zur Bestimmung der Absorptionskoeffizienten von Blei und Zink werden die Messwerte aus \autoref{tab:Mess_gamma} verwendet. Die Zählraten des \textit{GMZ} folgen einer
Poissoin-Verteilung, weshalb sich die Unsicherheit der Messwerte $N$ als $\symup{\Delta}N = \sqrt{N}$ annehmen lässt. 
Die Nullmessung zur Bestimmung des Hintergrund-/Störsignals ergibt $N_{00} = \qty{1.44}{\second^{-1}}$ (Es wurden $1295$ Signale in $\qty{900}{\second}$ gemessen). 
Die Zählraten werden zeitlich gemittelt. Anschließend wird die Nullmessung (Hintergrund) subtrahiert und es 
wird der Logarithmus der dimensionslosen Größe $N(d) \cdot  \mathrm{s} = N_t/t \cdot  \mathrm{s}$ gegen die Dicke der Absorberschicht aufgetragen. 
Die so entstehenden Diagramme sind für beide Absorbermaterialien in 
\autoref{fig:gamma} abgebildet. 
Da die Absorption von Strahlung nach \autoref{eqn:eabsorbgesetz} einem exponentiellen Gesetz der Form 
\begin{equation}
  \label{eqn:Absorption}
  N(d) = N_0 \cdot \mathrm{e}^{-\mu \cdot d}
\end{equation}
folgt, gilt für den Logarithmus
\begin{equation*}
  \mathrm{log}(N(d)\cdot\mathrm{s}) = -\mu \cdot d + \mathrm{log(N_0) \cdot  \mathrm{s}},
\end{equation*}
was eine Geradengleichung $f(x) = -ax + b$ beschreibt. 

\begin{figure}
  \centering
  \includegraphics{gamma.pdf}
  \caption{Absorptionskurven der beiden Absorbermaterialien und lineare Regression. Erstellt mit \textit{matplotlib} \cite{matplotlib}.}
  \label{fig:gamma}
\end{figure}

Durch eine lineare Regression mittels \textit{scipy} \cite{scipy} ergeben sich die Parameter
\begin{align*}
    \text{Blei:} \\
    a &= \qty{82.7(3.1)e-3}{\per\milli\metre}, & b &= \num{4.65+-0.06} \\
    \text{Zink:} \\
    a &= \qty{37.8(1.9)e-3}{\per\milli\metre}, & b &= \num{4.71+-0.02} \\
\end{align*}
der Geradengleichungen für die beiden Absorber. Da $b = \mathrm{log}(N_0 \cdot  \mathrm{s})$ und $\mu = a$ gilt, ergeben sich die Werte
\begin{align*}
  \mu_\text{Blei} &= \qty{82.7(3.1)}{\per\metre}, & N_{0 \text{,Blei}} &= \qty{105+-6}{\second^{-1}} \\
  \mu_\text{Zink} &= \qty{37.8(1.9)}{\per\metre}, &  N_{0 \text{,Zink}} &= \qty{111.1+-2.5}{\second^{-1}} \\
\end{align*}
für die Absorptionskoeffizienten $\mu$ und die Größe $N_0 = N(0)$. Der Messwert ohne Absorber ergab sich zu $N_{0, gemessen} = \qty{102.7(1.9)}{\second^{-1}}$.

Mithilfe von \autoref{eqn:sigma_com} kann der Wirkungsquerschnitt des Comptoneffektes berechnet werden. Dazu wird das Energieverhältnis $\varepsilon = 1.295$ und
der klassische Elektronenradius $r_e = \qty{2.82e-15}{\metre}$ \cite{v704} verwendet. Der Wirkungsquerschnitt ergibt sich zu 
$\sigma_\text{com} = \qty{2.57e-29}{\metre^2}$. Mit \autoref{eqn:mu_com} kann die Comptonabsorptionskonstante für die beiden Absorbermaterialien berechnet werden.
Dazu werden die in \autoref{tab:Material} aufgeführten Materialkonstanten verwendet.

\begin{table}[H]
  \centering
  \caption{Materialkonstanten zu Blei und Zink. $Z$: Ordnungszahl, $\rho$: Dichte, $M$: Molmasse \cite{Gestis}.}
  \label{tab:Material}
  \begin{tabular}{c c c}
    \toprule
      {} & {Blei} & {Zink} \\
      \midrule
      {$Z$}                                                    & 82    & 30    \\
      {$\rho \mathbin{/} \unit{\gram\per\cubic\centi\metre}$} & 11.3  & 7.14  \\
      {$M \mathbin{/} \unit{\gram\per\mol}$}                  & 207.2 & 65.39 \\
    \bottomrule
  \end{tabular}
\end{table}

Es ergeben sich die Theoriewerte des Compton-Absorptionskoeffizienten 
\begin{align*}
  \mu_\text{com, Pb} &= \qty{69.09}{\per\metre} & \mu_\text{com, Zn} &= \qty{50.61}{\per\metre}
\end{align*}
für die beiden Absorberstoffe.

\subsection{\texorpdfstring{Absorptionskurve des $\beta$-Strahlers}{Absorptionskurve des Beta-Strahlers}}
\label{subsec:A_beta}
Als $\beta$-Strahlungsquelle wird Technezium-99 verwendet. Das Absorbermaterial ist Aluminium. Zur Nullmessung der Hintergrundstrahlung werden $553$ Impulse In
$\qty{900}{\second}$ gemessen. Dies ergibt ein Hintergrundsignal von $\qty{ 0.614+-0.026}{\second^{-1}}$. Die Messwerte der Absorptionskurve sind in 
\autoref{tab:Mess_beta} aufgeführt.

\begin{table}
  \centering
  \caption{Messwerte der Absorption von $\beta$-Strahlung eines Technezium-99-Strahlers. Das Absorbermaterial ist Aluminium. $d$ beschreibt die Dicke der Absorberschicht und
  $N_t$ die Zählraten des \textit{GMZ} während der Zeit $t$.}
  \label{tab:Mess_beta}
  \begin{tabular}{S[table-format = 3.0] @{${}\pm{}$} S[table-format = 1.1] S[table-format = 4.0] S[table-format = 3.0]}
    \toprule
      \multicolumn{2}{c}{$d \mathbin{/} \unit{\micro\metre}$} & {$N_t$} & {$t \mathbin{/} \unit{\second}$} \\
      \midrule
      125 & 0   & 2470 & 200 \\
      100 & 0   & 7508 & 200 \\
      153 & 0.5 & 1964 & 220 \\
      160 & 1   & 1258 & 240 \\
      200 & 1   &  550 & 280 \\
      253 & 1   &  275 & 300 \\
      302 & 1   &  220 & 320 \\
      338 & 5   &  228 & 360 \\
      400 & 1   &  252 & 400 \\
      444 & 1   &  249 & 400 \\
      482 & 1   &  280 & 400 \\
    \bottomrule
  \end{tabular}
\end{table}

Wieder wird der Logarithmus der zeitlich gemittelten Zählraten $N$ gegen die Dicke der Absorberschicht aufgetragen. Das Hintergrundsignal wird wie zuvor von den
Messwerten subtrahiert.
Anhand der \autoref{fig:beta} lässt sich ein Bereich
linearer Steigung (blaue Marker) und ein Bereich einer konstanten Gerade (rote Marker) feststellen. Zweiterer Bereich ist ein Untergrundrauschen, welches nicht in Verbindung
mit der eigentlichen Absorptionskurve steht. 

\begin{figure}
  \centering
  \includegraphics{beta.pdf}
  \caption{Absorptionskurve des $\beta$-Strahlers und Regressionsgeraden.}
  \label{fig:beta}
\end{figure}

Mit den jeweiligen Messwerten wird eine Regression beziehungsweise Mittelung durchgeführt. 
Für die blaue Gerade ergeben sich die Parameter
\begin{align*}
  a &= \qty{-30.4(1.1)}{\per\milli\metre} & b &= \num{6.5(0.21)}.
\end{align*}
Der Wert der konstanten (orangen) Gerade ist $y = const = \num{-3.85(0.86)}$.
Der $x$-Wert des Schnittpunktes dieser beiden Geraden ist die maximale Reichweite der Strahlung im Absorbermaterial (Aluminium). Er kann über
\begin{equation*}
  D_\text{max} = \frac{const -b}{a} = \qty{341(14)}{\micro\metre}
\end{equation*}
berechnet werden.
Mit diesem Wert lässt sich wiederum die Massenbelegung
\begin{equation*}
    R_\text{max} = \rho \cdot D_\text{max} = \qty{0.092+-0.004}{\gram\per\square\centi\metre}
\end{equation*}
ermitteln. Dazu wird die Dichte von Aluminium $\rho_\text{Alu} = \qty{2.7}{\gram\per\cubic\centi\metre}$ \cite{Gestis} verwendet.
Mit \autoref{eqn:E_max} ergibt sich die maximale Energie der $\beta$-Strahlung zu
\begin{equation*}
  E_\text{max} = \qty{0.325(0.009)}{\mega\electronvolt}.
\end{equation*}
