\section{Diskussion}
\label{sec:Diskussion}
Zunächst wird der Literaturwert \cite{Ingenieurwissen} der Verdampfungswärme mit dem expermientellen Wert verglichen. 
\begin{align*}
    L_\text{exp} &= \qty{40.47 +- 0.12}{\kilo\joule\per\mol} \\
    L_\text{lit} &= \qty{40.66}{\kilo\joule\per\mol}
\end{align*}
Die Abweichung zum Literaturwert liegt bei circa $0.46\%$. Das Experiment bestätigt  im Rahmen der Messunsicherheiten eindeutig den Literaturwert.
Betrachtet man die temperaturabhängige Verdampfungswärme fällt auf, dass sich mathematisch zwei Lösungen ergeben. Allerdings kann lediglich der Graph von $L_+$ als physikalisch sinnvoll angenommen
werden, da die Verdampfungswärme bei steigender Temperatur abnehmen muss. Dies wurde bereits im \autoref{sec:Theorie} erklärt. 
