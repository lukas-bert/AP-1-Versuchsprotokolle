\section{Auswertung}
\label{sec:Auswertung}
Die Fehler- und Ausgleichsrechnung des folgenden Kapitels wird in \textit{Python} unter Verwendung der Pakete \textit{scipy} \cite{scipy} und \textit{uncertainties}
\cite{uncertainties} durchgeführt. Sie genügen der gaußschen Fehlerfortpflanzung und dem Standardfehler des Mittelwerts.  
Vor Beginn der Messung wird der Atmosphärendruck zu $p_0 = \qty{1010}{\milli\bar} = \qty{101}{\kilo\pascal}$ und die Außentemperatur zu 
$T_0 = \qty{19.5}{\degreeCelsius}$ bestimmt.

\subsection{Bestimmung der Verdampfungswärme zur ersten Messung}
\label{subsec:A_L_Bestimmung}
Unter der Annahme, dass die Verdampfungswärme in dem hier untersuchten Druckbereich keine Temperatur- und Druckabhängigkeit aufweist, lässt sich nach *autoref{KommtNochSooOOOOsS}*
feststellen, dass der Logarithmus des Relativdruckes in Abhängigkeit zur reziproken Temperatur eine Geradengleichung der Form $f(x) = mx + b$ beschreibt. 
Der Parameter $m$ der Geradengleichung entspricht dabei dem Faktor $-\sfrac{L}{R}$, woraus sich der Mittelwert der Verdampfungswärme $L$ im betrachteten Druckbereich bestimmen lässt.

\begin{figure}
    \centering
    \includegraphics[width = \textwidth]{build/plot1.pdf}
    \caption{Messdaten zur Bestimmung der Verdampfungswärme im Druckbereich $\qty{25}{\milli\bar} \leq p \leq p_0$ und zugehörige Ausgleichsgerade \cite{matplotlib}.}
    \label{fig:plot1}
\end{figure}

In \autoref{fig:plot1} ist der beschriebene Zusammenhang des Logarithmus des Druckes zum Kehrwert der Temperatur zu sehen. Eine lineare Regression mittels \textit{scipy} 
\cite{scipy} ergibt die Parameter
\begin{align*}
    m &= \qty{-4868 +- 13}{\kelvin} & b &= \num{14.67 +- 0.04}
\end{align*}
der Ausgleichsgeraden. Damit folgt für $L = -mR$ mit der allgemeinen Gaskonstante $R = \qty{8.314}{\joule\per\mol\kelvin}$ \cite{scipy} der Wert 
$L = \qty{40474 +- 122}{\joule\per\mol}$. Die äußere Verdampfungswärme $L_\text{a}$ lässt sich mithilfe autoref{AllgemeineGasgleichung} abschätzen. Für die Temperatur
$T = \qty{373}{\kelvin}$ folgt $L_\text{a} = \qty{3101}{\joule\per\mol}$. Mit der Relation autoref{ZusammenhangDerL's} ergibt sich für die innere Verdampfungswärme
\begin{equation*}
    L_\text{i} = L - L_\text{a} = \qty{37372+-122}{\joule\per\mol}.
\end{equation*}
Durch Division durch die Avoagdro-Konstante $N_\text{A} = 6.022 \cdot 10^23 \unit{\per\mol}$ \cite{scipy} kann der Wert dieser Größe pro Molekül bestimmt werden.
In $\unit{\electronvolt}$ ergibt sich
\begin{equation*}
    L_\text{i, Molekül} = \qty{0.3873+- 0.0013}{\electronvolt}.
\end{equation*}

\subsection{Temperaturabhängigkeit der Verdampfungswärme über Atmosphärendruck}
\label{subsec:A_Temperaturabhäng}
Für einen Druckbereich $p_0 \geq \qty{1}{\bar}$ kann $L$ nicht mehr als konstant angenommen werden. Mit der Clausius-Clapeyronschen Gleichung autoref{IstKlarOder}
lässt sich der Zusammenhang 
\begin{equation}
    \label{eqn:L_Clausius}
    L = (V_\text{D} - V_\text{F}) \frac{\symup{d}p}{\symup{d}T} \cdot T
\end{equation}
für $L(T)$ aufstellen. Hierzu wird jedoch das Volumen $V_\text{D}$ des Dampfes benötigt. Das Volumen $V_\text{F}$ wird als vernachlässigbar
klein gegenüber $V_\text{D}$ angenommen. Da die allgemeine Gasgleichung in diesem Bereich nicht mehr gilt, wird die Näherung 
\begin{align*}
    \left(p + \frac{a}{V^2}\right)V &= RT & a &= \qty{0.9}{\joule\cubic\metre\per\mol\squared}
\end{align*}
verwendet. Mit dieser Näherung ergeben sich die Lösungen $V_+$ und $V_-$ des Volumens zu 
\begin{equation*}
    V_\pm = \frac{RT}{2p} \pm \sqrt{\left(\frac{RT}{2p}\right)^2-\frac{a}{p}}.
\end{equation*}
Durch Einsetzen der Lösungen für das Dampfvolumen $V_\text{D}$ in \autoref{eqn:L_Clausius} ergibt sich
\begin{equation}
    L_\pm(p, T) = \frac{1}{p}\left(\frac{RT}{2} \pm \sqrt{\frac{R^2T^2}{4}- ap} \right) \frac{\symup{d}p}{\symup{d}T}T
\end{equation} 

Anhand der Daten der zweiten Messreihe lässt sich eine Ausgleichsfunktion ermitteln, mit der die Temperaturabhängigkeit des Druckes $p$ modelliert werden kann.
Als Funktion wird ein Polynom dritten Grades der Form $p(x) = ax^3 + bx^2 + cx + d$ gewählt. Die entsprechenden Messdaten und die Ausgleichsfunktion sind
in \autoref{fig:plot3} abgebildet. 

\begin{figure}
    \centering
    \includegraphics[width = \textwidth]{build/plot2.pdf}
    \caption{Messwerte im Druckbereich $p_0 \leq p \leq \qty{15}{\bar}$ und Ausgleichspolynom dritten Grades \cite{scipy} \cite{matplotlib}.}
    \label{fig:plot2}
\end{figure}

Die Regression mittels \textit{scipy} \cite{scipy} ergibt die Parameter
\begin{align*}
    a &= \qty{0.65 +- 0.11}{\pascal\per\cubic\kelvin} & b &= \qty{-669 +- 142}{\pascal\per\kelvin\squared} \\
    c &= (2.29 \pm 0.6) \cdot 10^5 \unit{\pascal\per\kelvin}     & d &= (-2.62 \pm 0.86) \cdot 10^7 \unit{\pascal}. \\ 
\end{align*}
Diese Parameter liefern Gleichungen für den Druck $p(T) = aT^3 + bT^2 + cT + d$ und dessen Ableitung $\frac{\symup{d}p}{\symup{d}T}(T) = 3aT^2 + 2bT + c$, mit denen 
sich die beiden Lösungen $L_+$ und $L_-$ in Abhängigkeit zur Temperatur darstellen lassen. Die so entstehenden Funktionen können den Abbildungen \ref{fig:plot3} und 
\ref{fig:plot4} entnommen werden.

\begin{figure}
    \centering
    \includegraphics[width = \textwidth]{build/plot3.pdf}
    \caption{Ergebnis $L_+$ der Bestimmung der Temperaturabhängigkeit der Verdampfungswärme im Bereich $p \geq \qty{1}{\bar}$ \cite{matplotlib}.}
    \label{fig:plot3}
\end{figure}

\begin{figure}
    \centering
    \includegraphics[width = \textwidth]{build/plot4.pdf}
    \caption{Ergebnis $L_-$ der Bestimmung der Temperaturabhängigkeit der Verdampfungswärme im Bereich $p \geq \qty{1}{\bar}$ \cite{matplotlib}.}
    \label{fig:plot4}
\end{figure}
