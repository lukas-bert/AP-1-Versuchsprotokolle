\section{Auswertung}
\label{sec:Auswertung}
Vorab: Die Messunsicherheiten wurden gemäß der gaußschen Fehlerfortpflanzung
\begin{equation*}
  \label{eqn:Gauss}
  \Delta F = \sqrt{\sum_i\left(\frac{\symup{d}F}{\symup{d}y_i}\Delta y_i \right)^2}
\end{equation*}
mittels Scipy \cite{scipy}  berechnet.\\
\\
Gerätekonstanten zu den Bauteilen:\\
Helmholtzspulenpaar:
\begin{align*}
    &\text{Windungen}& N &= 195 \\
    &\text{Radius}& r_H &= 0.109 \unit{\metre} \\
    &\text{Abstand der Spulen}& d_H &= 0.138 \unit{\metre} \\
\end{align*}
Materialkonstanten:
\begin{align*}
    &\text{Kugeldurchmesser}& d_K &= (5.40 \pm 0.01) \unit{\centi\metre} \\
    &\text{Kugelmasse}& m_K &=  141.76  \unit{\gram} \\
    &\text{Masse des Zylinders am Stab}& m_1 &=  1.39  \unit{\gram} \\
    &\text{Masse des Zylinders am Stab mit Stab}& m_{1,s} &= 1.64 \unit{\gram} \\
    &\text{Höhe des Zylinders}& d_{1,s} &= (0.995 \pm 0.01) \unit{\centi\metre} \\
    &\text{Länge des Alustabs}& l_s &= (10.79 \pm 0.01) \unit{\centi\metre} \\
    &\text{Länge des Kugelstiels}& l_{Ks} &= (1.25 \pm 0.01) \unit{\centi\metre} \\
\end{align*}

\begin{table}
    \centering
    \caption{Messdaten zur Methode 1: Ausnutzung der Gravitation. \\ $r$: gemessener Abstand der verschiebbaren Masse, $r_{\text{ges}}$: reale Hebellänge,\\ 
            $B$: Magnetfeldstärke errechnet aus I}
    \label{tab:Mess1}
    \begin{tabular}{S S[table-format=1.2] S[table-format=2.2] S[table-format=1.2]}
        \toprule
        $r \mathbin{/} \unit{\centi\metre}$ & $I \mathbin{/} \unit{\ampere}$ & $r_{\text{ges}} \mathbin{/} \unit{\centi\metre}$ & $B \mathbin{/} \unit{\milli\tesla}$ \\
        \midrule
        6.03 & 2.65 & 10.48 & 3.59 \\
        5.5  & 2.50 & 9.95  & 3.39 \\
        5.0  & 2.40 & 9.45  & 3.25 \\
        4.5  & 2.30 & 8.95  & 3.12 \\
        4.0  & 2.20 & 8.45  & 2.98 \\
        3.5  & 2.05 & 7.95  & 2.78 \\
        3.0  & 1.90 & 7.45  & 2.58 \\
        2.5  & 1.80 & 6.95  & 2.44 \\
        2.0  & 1.70 & 6.45  & 2.31 \\
        1.5  & 1.60 & 5.95  & 2.17 \\
        1.0  & 1.50 & 5.45  & 2.03 \\
        0.5  & 1.40 & 4.95  & 1.90 \\
        0.0  & 1.25 & 4.45  & 1.70 \\
    \bottomrule 
    \end{tabular}
\end{table}

\subsection{Bestimmung des magnetischen Momentes über die Erdgravitation}
\label{subsec:AuswertungZu1}
In \autoref{fig:Plot1} wird die Magentfeldstärke $B$ gegen den Abstand $r$ der aufgesteckten Masse zum Dipol aufgetragen. Der Abstand r wurde mit einer Ungenauigkeit von $\pm 0.001\unit{\centi\metre}$ gemessen.
Die Magentfeldstärke $B$ in der Mitte des Helmholtzspulenpaares lässt sich gemäß der Gleichung \eqref{eqn:Helmholtz_B} aus der gemessenen Stromstärke $I$  berechnen (vgl.\autoref{tab:Mess1}). 
Die gemessene Stromstärke ist mit einer Messunsicherheit von $\pm 0.05\unit{\ampere}$ behaftet. Zur Berechnung von B werden $R = r_H$ und $x = \frac{d_H}{2}$ verwendet.
\begin{figure}
    \centering
    \includegraphics[width=\textwidth]{build/plot1.pdf}
    \caption{Linearer Fit mit Scipy \cite{scipy} zu den Messwerten der Gravitationsmethode. Der Plot wurde mit dem Paket Matplotlib \cite{matplotlib} erstellt.}
    \label{fig:Plot1}
\end{figure} 
\\
\\
Mittels einer linearen Regression mit dem Paket Scipy \cite{scipy} und der Gleichung
\begin{equation*}
    B = ar_{ges} + b
\end{equation*}
ergeben sich die Parameter 
\begin{align*}
    a &= (3.10 \pm 0.04) 10^{-2}\unit{\tesla\per\metre}\\
    b &= (3.24 \pm 0.34) 10^{-4} \unit{\tesla}\\
\end{align*}
für den Fit. Dieser Fit wird dann in Formel \eqref{eqn:Drehmoment_Gleichgewicht2} eingesetzt. Dabei kann $b$ vernachlässigt werden. Nach umstellen der Formel 
\eqref{eqn:Drehmoment_Gleichgewicht2} kann das magnetische Moment berechnet werden.
\begin{equation}
    \label{eqn:mu1}
    \mu_{\text{Dipol}} = (0.440 \pm 0.006) \unit{\square\ampere\per\metre}
\end{equation}

\subsection{Bestimmung des magnetischen Moments über die Schwingungsdauer}
\label{subsec:AuswertungZu2}

Wie in \autoref{subsec:Methode2} beschrieben, wird beim zweiten Messverfahren das Dipolmoment über die Schwingungsdauer bestimmt.
Dazu wird der Zusammenhang \eqref{eqn:Schwingungsdauer} ausgenutzt. In \autoref{tab:Mess2} finden sich die gemessenen Größen und die Schwingungsdauer $T$, die
sich aus den Messwerten berechnet. 

\begin{table}
    \centering
    \caption{Messdaten zur Methode 2: Schwingungsdauer. \\ Es wurden jeweils 2 unabhängige Werte für die 10-fache Schwingungsdauer gemessen
            und daraus der Mittelwert $T$ einer einzigen Periode gebildet.}
    \label{tab:Mess2}
    \begin{tabular}{S[table-format=1.2] S S[table-format=2.2] S[table-format=1.2]}
        \toprule
        $I \mathbin{/} \unit{\ampere}$ & ${10} T_{1} \mathbin{/} \unit{\second}$ & ${10} T_{2} \mathbin{/} \unit{\second}$ & $T \mathbin{/} \unit{\second}$ \\
        \midrule
        0.5  & 23.66 & 24.07 & 2.39 \\
        1.0  & 16.44 & 16.61 & 1.65 \\
        1.5  & 13.52 & 13.55 & 1.35 \\
        2.0  & 11.53 & 11.61 & 1.16 \\
        2.5  & 10.30 & 10.66 & 1.05 \\
        3.0  &  9.40 &  9.41 & 0.94 \\
        3.25 &  9.04 &  9.12 & 0.91 \\
        3.5  &  8.69 &  8.81 & 0.88 \\
        4.0  &  8.13 &  8.24 & 0.82 \\
        4.1  &  8.10 &  8.14 & 0.81 \\
    \bottomrule 
    \end{tabular}
\end{table}

Wie man an \autoref{eqn:Schwingungsdauer} erkennen kann, besteht ein linearer Zusammenhang 
\begin{equation*}
    T^2 = a \cdot \frac{1}{B}
\end{equation*}
zwischen dem Quadrat der Schwingungsdauer und dem Kehrwert des Betrags der magnetischen
Flussdichte $B$

Zur Ermittlung des Parameters a werden die aus den Messwerten berechneten Größen $B^{-1}$ und $T^2$ gegeneinander aufgetragen, wie es in \autoref{fig:Plot2} zu erkennen ist.

\begin{figure}
    \centering
    \includegraphics[width=\textwidth]{build/plot2.pdf}
    \caption{Linearer Fit mit Scipy \cite{scipy} zu den Messwerten der Schwingungsmethode. (Erstellt mit Matplotlib \cite{matplotlib})}
    \label{fig:Plot2}
\end{figure} 

Wie auch bei der ersten Methode lässt sich dadurch mittels linearer Regression eine Gerade finden, die den Verlauf der Messwerte am geeignetsten modelliert.
Die bestimmten Parameter der Geraden $f(x) = ax + b$  lauten:
\begin{align*}
    a &= (3.84 \pm 0.04) \cdot 10^{-3} \unit{\tesla\second\squared} \\
    b &= (-0.046 \pm 0.014) \unit{\second\squared}
\end{align*}
Das Trägheitsmoment $J_\text{K}$ der Kugel berechnet sich mit den oben genannten Abmessungen und \autoref{eqn:Trägheitsmoment_Kugel}. Es hate den Betrag
\begin{equation*}
    J_\text{K} = \frac{2}{5} m_{\text{K}} r_{\text{K}}^2 = (4.1337 \pm 0.0031) \cdot 10^{-5} \unit{\kilogram\meter\squared}  
\end{equation*}
.
Damit folgt aus Formel \eqref{eqn:Schwingungsdauer}, unter Vernachlässigung von $b$ (da theoretisch $b = 0$ gilt):
\begin{equation*}
    a = \frac{4 \pi^2 J_{\text{K}}}{\mu_{\text{Dipol}}}
\end{equation*}
Umgestellt nach $\mu_{\text{Dipol}}$ ergibt sich:
\begin{equation}
    \label{eqn:mu2}
    \mu_{\text{Dipol}} = \frac{4 \pi^2 J_{\text{K}}}{a} = (0.425 \pm 0.004) \unit{\ampere\squared\per\metre}
\end{equation}
