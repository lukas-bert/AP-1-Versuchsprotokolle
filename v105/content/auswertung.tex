\section{Auswertung}
\label{sec:Auswertung}
Vorab: Die Messunsicherheiten wurden gemäß der gaußschen Fehlerfortpflanzung
\begin{equation}
  \label{eqn:Gauss}
  \Delta F = \sqrt{\sum_i\left(\frac{\symup{d}F}{\symup{d}y_i}\Delta y_i \right)^2}
\end{equation}
mittels Scipy \cite{scipy}  berechnet.\\
\\
Gerätekonstanten zu den Bauteilen:\\
Helmholtzspulenpaar:
\begin{align*}
    &\text{Windungen}& N &= 195 \\
    &\text{Radius}& r_H &= 0.109 \unit{\metre} \\
    &\text{Abstand der Spulen}& d_H &= 0.138 \unit{\metre} \\
\end{align*}
Material:
\begin{align*}
    &\text{Kugeldurchmesser}& d_K &= (5.40 \pm 0.01) \unit{\centi\metre} \\
    &\text{Kugelmasse}& m_K &=  141.76  \unit{\gram} \\
    &\text{Masse des Zylinders am Stab}& m_1 &=  1.39  \unit{\gram} \\
    &\text{Masse des Zylinders am Stab mit Stab}& m_{1,s} &= 1.64 \unit{\gram} \\
    &\text{Höhe des Zylinders}& d_{1,s} &= (0.995 \pm 0.01) \unit{\centi\metre} \\
    &\text{Länge des Alustabs}& l_s &= (10.79 \pm 0.01) \unit{\centi\metre} \\
    &\text{Länge des Kugelstiels}& l_{Ks} &= (1.25 \pm 0.01) \unit{\centi\metre} \\
\end{align*}

\begin{table}
    \centering
    \caption{Messdaten zur Methode 1: Ausnutzung der Gravitation. \\ $r$: gemessener Abstand der verschiebbaren Masse, $r_{\text{ges}}$: reale Hebellänge,\\ 
            $B$: Magnetfeldstärke errechnet aus I}
    \label{tab:Mess1}
    \begin{tabular}{S S[table-format=1.2] S[table-format=2.2] S[table-format=1.2]}
        \toprule
        $r \mathbin{/} \unit{\centi\metre}$ & $I \mathbin{/} \unit{\ampere}$ & $r_{\text{ges}} \mathbin{/} \unit{\centi\metre}$ & $B \mathbin{/} \unit{\milli\tesla}$ \\
        \midrule
        6.03 & 2.65 & 10.48 & 3.59 \\
        5.5  & 2.50 & 9.95  & 3.39 \\
        5.0  & 2.40 & 9.45  & 3.25 \\
        4.5  & 2.30 & 8.95  & 3.12 \\
        4.0  & 2.20 & 8.45  & 2.98 \\
        3.5  & 2.05 & 7.95  & 2.78 \\
        3.0  & 1.90 & 7.45  & 2.58 \\
        2.5  & 1.80 & 6.95  & 2.44 \\
        2.0  & 1.70 & 6.45  & 2.31 \\
        1.5  & 1.60 & 5.95  & 2.17 \\
        1.0  & 1.50 & 5.45  & 2.03 \\
        0.5  & 1.40 & 4.95  & 1.90 \\
        0.0  & 1.25 & 4.45  & 1.70 \\
    \bottomrule 
    \end{tabular}
\end{table}

\subsection{Bestimmung des magnetischen Momentes über die Erdgravitation}
\label{AuswertungZu1}
In \autoref{fig:Plot1} wird die Magentfeldstärke $B$ gegen den Abstand $r$ der aufgesteckten Masse bis zum Dipol aufgetragen. Der Abstand r wurde mit einer Ungenauigkeit von $\pm 0.001\unit{\centi\metre}$ gemessen.
Die Magentfeldstärke $B$ in der Mitte des Helmholtzspulenpaares lässt sich gemäß der Gleichung \eqref{eqn:Helmholtz_B} aus der gemessenen Stromstärke $I$  berechnen(vgl.\autoref{tab:Mess1}). 
Die gemessene Stromstärke ist mit einem Fehler von $\pm 0.05\unit{\ampere}$ behaftet. Zur Berechnung von B werden $R = r_H$ und $x = \frac{d_H}{2}$ verwendet.
\begin{figure}
    \centering
    \includegraphics[width=\textwidth]{build/plot1.pdf}
    \caption{linearer Fit \cite{scipy} zu den Messwerten der Gravitationsmethode}
    \label{fig:Plot1}
\end{figure} 
\\
\\
Mittels einer linearen Regression \cite{scipy} 
\begin{equation*}
    B = ar_{ges} + b
\end{equation*}
ergeben sich die Parameter 
\begin{align*}
    a &= (3.10 \pm 0.04) 10^{-2}\unit{\tesla\per\metre}\\
    b &= (3.24 \pm 0.34) 10^{-4} \unit{\tesla}\\
\end{align*}
für den Fit. Dieser Fit wird dann in Formel \eqref{eqn:Drehmoment_Gleichgewicht2} eingesetzt. Dabei kann $b$ vernachlässigt. Nach umstellen der Formel 
\eqref{eqn:Drehmoment_Gleichgewicht2} kann das magnetische Moment berechnet werden.
\begin{align*}
    \mu_{\text{Dipol}} = (0.440 \pm 0.006) \unit{\square\ampere\per\metre}\\
\end{align*}
