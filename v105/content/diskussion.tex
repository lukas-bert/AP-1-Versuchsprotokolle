\section{Diskussion}
\label{sec:Diskussion}
Zunächst wird die Gravitationsmethode diskutiert. Durch diese Methode wurde das magnetische Dipolmoment zu
\begin{align*}
    \mu_{\text{Dipol, Gravitation}} = (0.440 \pm 0.006)\unit{\ampere\squared\per\metre}
\end{align*}
bestimmt. Da es zu diesem Versuch keinen Litaraturwert gibt, kann man keine genaue Aussage über die Qualität des 
Wertes machen. Lediglich der sehr lineare Verlauf der Messwerte, wie in \autoref{fig:Plot1} zu sehen ist, deutet auf die Qualität des bestimmten Wertes hin. 
Dennoch lassen sich auch hier mögliche Fehlerquellen feststellen. So ist es aus reiner Betrachtung nicht möglich das exakte Kräftegleichgewicht zu erkennen, wodurch sich Fehler in der 
eingestellten Stromstärke für das Magnetfeld ergeben. Dazu kommt der Energieverlust des Systems durch die Reibung des Luftkissens, sowie eine leicht ungleichmäßige und unregelmäßige
Verteilung der Luftkissenströmung. Einer weitere Fehlerquelle liegt in dem nicht komplett ebenen Aufbau.
Abgesehen von diesen nicht konkret bestimmbaren Messunsicherheiten ergibt sich nur eine Unsicherheit von $\pm 0.006 \unit{\ampere\squared\per\metre}$ durch 
diverse Ableseungenauigkeiten. Diese fällt in Relation zu der Größe des bestimmten Wertes nur sehr klein aus.

Das Dipolmoment der Schwingungsmethode ergab sich zu
\begin{align*} 
    \mu_{Dipol, Schwingung} = (0.425 \pm 0.005)\unit{\ampere\squared\per\metre}
\end{align*}   
Auch hier liegen ebenfalls potenzielle, aufbaubedingte Fehlerquellen vor. Dazu zählen vor allem durch die Schwingung verursachte Unsicherheiten.
So gilt zum Beispiel die zur Bestimmung verwendete Formel lediglich für kleine Winkel. Diese sind aber durch 
händisches Auslenken nicht unbedingt eingehalten. Dazu kommt noch, dass das Pendel nicht perfekt in einer Ebene schwingt und man auch nicht den genauen Zeitpunkt der 
maximalen Auslenkung bestimmen kann. Auch in dieser Methode ist das einzige Qualitätsmaß wieder die Linearität der in \autoref{fig:Plot2} aufgetragenen Messwerte. 
Die bestimmte Messunsicherheit ist mit $\pm 0.005\unit{\ampere\squared\per\metre}$ ebenfalls relativ klein. 


Verlgeicht man nun die beiden Methoden erkennt man, dass die beiden errechneten Werte jeweils nicht in der Fehlertoleranz des anderen liegen. Dies ist durch die teilweise nicht 
bestimmbaren Fehler, wie z.B das Augenmaß beim  Ermitteln des Gleichgewichtzustandes oder die händische Auslenkung bei der Schwingung, zu erklären.


Betrachtet man die relative Abweichung der magnetischen Dipolmomente 
\begin{align*}
    \Delta_{\text{Abweichung}} = (3.5 \pm 0.02)\%
\end{align*}
kann man sagen, dass die beiden Methoden von ähnlicher Qualität sind.  