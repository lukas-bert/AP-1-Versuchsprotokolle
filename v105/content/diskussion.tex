\section{Diskussion}
\label{sec:Diskussion}
Zunächst wird die Gravitationsmethode diskutiert. Durch dise Methode wurde magnetische Dipolmoment 
\begin{align*}
    \mu_{Dipol, Gravitation} = (0.440 \pm 0.006)\unit{\ampere\squared\per\metre}
\end{align*}
bestimmt. Da es zu diesem Versuch keinen Litaraturwert gibt, weil das magnetische Dipolmoment immer vom Versuchsaufbau abhängt, kann man keine gute Aussage über die Qualität des 
Wertes machen. Lediglich der sehr lineare Verlauf der Messwerte, wie in \autoref{fig:Plot1} zu sehen ist, deutet auf die Qualität unseres bestimmten Wertes hin. 
Dennoch lassen sich auch hier Fehlerquellen feststellen. So ist es aus reiner Betrachtung nicht möglich das exakte Gleichgewicht zu erkennen, wodurch sich Fehler in der 
eingestellten Stromstärke für das Magnetfeld ergeben. Dazu kommt der Energieverlust des Systems durch die Reibungsverlust des Luftkissens, sowie eine leicht ungleichmäßige 
verteilung des Luftkissens. Einer weitere Fehlerquelle liegt in dem nicht koplett ebenen Aufbau. Trotz all dieser möglichen Fehlerquellen ist ein Fehler 
von $\pm 0.006 \unit{\ampere\squared\per\metre}$ eine gute Messung.


Ebenso gibt es auch zur Methode des Schwingungsdauer keinen Litaraturwert, mit welchem man den gemessenen Wert für das magnetische Dipolmoment
\begin{align*} 
    \mu_{Dipol, Schwingung} = (0.425 \pm 0.005)\unit{\ampere\squared\per\metre}
\end{align*}   
vergleichen könnte. Außerdem liegen hier ebenfalls die aufbaubedingten Fehlermöglichkeiten vor. 
Bei dieser Methode kommen noch ein mögliche Fehler durch die Schwingung hinzu. So gilt die zur Bestimmung verwendete Formel lediglich für kleine Winkel. Diese sind aber durch 
händisches Auslenken nicht unbedingt eingehalten. Dazu kommt noch, dass das Pendel nicht perfekt in einer Ebene geschwungen hat und man auch nicht den genauen Zeitpunkt der 
maximalen Auslenkung bestimmen kann. Auch in dieser Methode ist das einzige Qualitätsmaß wieder die Linearität der in \autoref{fig:Plot2} aufgetragenen Messwerte. 
Auch in dieser Methode ist der Verlauf sehr linear und der Fehler $\pm 0.005\unit{\ampere\squared\per\metre}$ relativ klein. 


Verlgeicht man nun die beiden Methoden erkennt man, dass die beiden errechneten Werte jeweils nicht in der Fehlertoleranz des anderen liegen. Dies ist durch die teilweise nicht 
bestimmbaren Fehler, wie z.B das Augenmaß beim Gleichgewicht oder die händische Auslenkung beim Schwingen, zu erklären.


Betrachtet man die prozentuale Abweichung der magnetischen Dipolmomente 
\begin{align*}
    \Delta_{\text{Abweichung}} = (3.5 \pm 0.02)\%
\end{align*}
kann man sagen, dass beiden Methoden von ähnlicher Qualität sind.  