\floatplacement{figure}{hbtp}
\section{Durchführung}
\label{sec:Durchfuehrung}
\subsection{Erste Problemstellung}
Zuerst soll die Zeitabhängigkeit der Spannungsamplitude eines gedämpften RLC-SChwingkreises untersucht werden. Daraus soll der effektive Dämpfungswiderstand bestimmt
werden. Mit der Schaltung aus \autoref{fig:SchaltungZuA} können die notwendigen Daten gemessen werden. Es soll aber anstatt einer Sägezahnspannung eine Rechtecksspannung 
angelegt werden. Diese wird durch einen Frequenzgenerator, an dem die Art der Schwingung und die dazugehörige Frequenz eingestellt werden kann, in die den Schwingkreis 
eingespeist. Das Oszilloskop zeigt den Schwingungsverlauf der eingespeisten Spannung gegen die Zeitachse an. Am Oszilloskop kann die Amplitude und die Zeit abgelesen werden. Es soll 
gemessen werden bis die Amplitude um den Faktor 3-8 abgenommen hat. 
\begin{figure}
    \includegraphics[width=0.7\textwidth]{content/SchaltungZuA.pdf}
    \centering
    \caption{Schaltung zur Bestimmung der Zeitabhängigkeit der Amplitude und des Dämpfungswiderstandes eines gedämpften RLC-SChwingkreises \cite{v354}.}
    \label{fig:SchaltungZuA}
\end{figure}
\subsection{Zweite Problemstellung}
Nun soll der Dämpfungswiderstand, bei dem der aperiodischen Grenzfall eintritt, bestimmt werden. Dieser soll mit der Schaltung aus \autoref{fig:SchaltungZuB} bestimmt werden. 
Die Schaltung ist ähnlich zu der Schaltung aus \autoref{fig:SchaltungZuA}, allerdings wird der feste Widerstand durch einen regelbaren Widerstand ausgetauscht. Zur Messung 
soll der regelbare Widerstand zunächst auf seinen Maximalwert(10\unit{\ohm}) eingestellt werden. Auf dem Oszilloskop wird dann ein Überdämpfungsverlauf der Spannung angezeigt.
Nun soll der Wiederstand solange runtergeregelt werden bis sich ein "Überschwingen" zeigt. Dann muss man den Widerstand nohc leicht erhöhen, bis die Überschwingung gerade 
verschwindet. Der Widerstand zum aperiodischen Grenzfall ist nun am regelbare Widerstand abzulesen.
\begin{figure}
    \centering
    \includegraphics[width=0.7\textwidth]{content/SchaltungZuB.pdf}
    \caption{Schaltung zur Bestimmung des Widerstandes zum aperiodischen Grenzfalls \cite{v354}.}    
    \label{fig:SchaltungZuB}
\end{figure}
\subsection{Dritte Problemstellung}
Im weiteren Messverlauf wird eine sinusförmige Schwingung betrachtet. Zunächst soll die Frequenzabhängigkeit der Kondensatorspannung $U_{C}$ zur anregenden Frequenz bestimmt 
werden. Hierzu muss beachtet werden, dass der Tastkopf(siehe \autoref{fig:SchaltungZuC}) selbst einen Frequenzgang besitzt. Daher wird auch die Erregerspannung $U$ gemessen.
Aus diesen beiden Messungen soll dann der Quotient $\sfrac{U_{C}}{U}$ bestimmt werden. Den Spannungsverlauf von  $U_{C}$ kann man mit der Schaltung aus \autoref{fig:SchaltungZuC}
aufgenommen werden. Anstatt eines AC-Milli-Voltmeters wird ein Oszilloskopangeschlossen an dem man die Spannung ablesen kann. Bei dieser Messung beginnt man bei einer angepasst
niedrigen Frequenz, welche man am Sinusgenerator einstellen kann, und regelt diese dann in konstanten Abständen hoch. Dabei wird bei jeder eingestellten Frequenz $U_{C}$ und 
$U$ gemessen. 
\begin{figure}
    \centering
    \includegraphics[width=0.7\textwidth]{content/SchaltungZuC.pdf}
    \caption{Schaltung zur Bestimmung der Frequenzabhängigkeit der Kondensatorspannung \cite{v354}.}    
    \label{fig:SchaltungZuC}
\end{figure}
\subsection{Vierte Problemstellung}
Zuletzt soll die Frequenzabhängigkeit der Phase $\phi$ zwischen der Erreger- und der Kondensatorspannung bestimmt werden. Dazu wird der Aufbau gemäß \autoref{fig:SchaltungZuD} verwendet.
Er wird ein Zweikanal-Oszilloskop an den RLC-SChwingkreis angeschlossen. Durch den Frequenzgenerator wird eine sinusförmige Schwingung in den RLC-Kreis eingespeist.
Auf dem Oszilloskop werden die Spannungsverläufe von der Kondensatorspannung $U_{C}$ und der Erregerspannung $U$ angezeigt. Die Frequenzabhängigkeit der Phase $\phi$ kann, wie 
in \autoref{fig:Phasenbestimmung} gegeben, bestimmt werden. Daher ließt man den Abstand der Nulldurchgänge $a$ und die Periodendauer $b$ der Kondensatorspannung $U_{C}$ auf dem 
Oszilloskop ab. Man beginnt mit einer angepasst niedrigen Frequenz und regelt diese dann in konstanten Abständen hoch. Die Phase $\phi$ soll dann gemäß der Formel 
\begin{equation}
    \label{eqn:Phase}
    \phi = \frac{a}{b}2\pi 
\end{equation}
bestimmt werden.
\begin{figure}
    \centering
    \includegraphics[width=0.7\textwidth]{content/SchaltungZuD.pdf}
    \caption{Schaltung zur Bestimmung der Frequenzabhängigkeit der Phase \cite{v354}.}    
    \label{fig:SchaltungZuD}
\end{figure}
\begin{figure}
    \centering
    \includegraphics[width=0.7\textwidth]{content/Phasenbestimmung.pdf}
    \caption{Hilfsskizze zur Bestimmung der Phase zweier Schwingungen \cite{v353}.}    
    \label{fig:Phasenbestimmung}
\end{figure}