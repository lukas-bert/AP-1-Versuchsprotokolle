\section{Auswertung}
\label{sec:Auswertung}

Zuerst wurden alle Gerätekonstanten festgestellt und notiert. Verwendet wurde das Oszilloskop Nummer 4, sowie
die Schaltung Nummer 1. Die Bauteile des Verwendeten Gerätes haben folgende Werte:

\begin{align*}
  &R_1 = (67.2 \pm 0.1) \unit{\ohm} \\
  &R_2 = (682 \pm 0.5) \unit{\ohm}  \\
  &L   = (16.87 \pm 0.05) \unit{\milli\henry} \\
  &C   = (2.060 \pm 0.003) \unit{\nano\farad} \\
\end{align*}

\subsection{Verwertung der Messwerte zur zeitabhängigkeit der Amplitude}
\label{subsec:AuswertungA}

Um den effektiven Dämpfungswiderstand zu bestimmen verwende man den Zusammenhang \eqref{eqn:def_mu_nu}.
Durch Umstellen auf $R_{\text{eff}}$ ergibt sich die Berechnungsformel:
\begin{equation}
  \label{Abklingdauer1}
  \Longleftrightarrow R_{\text{eff}} = 4\pi\mu L
\end{equation}
Um nun $R_{\text{eff}}$ zu berechnen wird zunächst eine Fit-Kurve durch die Messwerte gelegt (siehe \autoref{fig:PlotZuA}).

\begin{figure}
  \centering
  \includegraphics[width=\textwidth]{build/PlotZuA.pdf}
  \caption{Exponentieller Fit zu den Messwerten mit linearer Skala (links) und logarithmischer Skala für $U$ (rechts)}
  \label{fig:PlotZuA}
\end{figure}

Man benötigt nun noch den Faktor $2\pi\mu$. Diesen kann man durch die erstellte exponentielle Regression 
bestimmen, da er dem positiven Faktor des Exponenten der Exponentialfunktion entspricht. 
So ist $2\pi\mu = 6.98\,\unit{\second}$ gegeben.
$R_{\text{eff}}$ ergibt sich durch einsetzen zu:

\begin{equation*}
  R_{\text{eff}} = (2\cdot 6.98\cdot \num{16.87\pm 0.05}) \unit{\ohm} = (\num{117.75 \pm 0.05})\,\unit{\ohm}
\end{equation*}

Auffällig ist, dass $R_{\text{eff}}$ genau $R_1 + 50\unit{\ohm}$ entspricht. Dies hat den Grund, dass der Innenwiderstand des Generators
$50 \unit{ohm}$ beträgt. Aus diesem Grund wird dieser in den Folgenden Auswertungen berücksichtigt.
Die Abklingdauer lässt sich nach Gleichung \eqref{eqn:T_ex} berechnen.

\subsection{Aperiodischer Grenzfall}
\label{subsec:AuswertungB}

Der Theoriewert des Widerstands, bei dem der aperiodische Grenzfall eintritt, lässt sich mit Formel \eqref{eqn:R_ap} berechnen.
Der Fehler dieses Wertes ergibt sich nach der gaußschen Fehlerfortpflanzung zu:

\begin{align*}
  \label{eqn:err_R_ap}
  \Delta R_{\text{ap}} &= \sqrt{\Bigl(\frac{\symup{d}R_{\text{ap}}}{\symup{d}L}\Delta L \Bigr)^2+\Bigl(\frac{\symup{d}R_{\text{ap}}}{\symup{d}C}\Delta C\Bigr)^2} \\
  &= \sqrt{\Bigl(\frac{1}{\sqrt{LC}}\Delta L\Bigr)^2+\Bigl(\frac{\sqrt{LC}}{C^2}\Delta C\Bigr)} \\
  &\approx 43 \unit{\ohm}
\end{align*}

Insgesamt erhält man $R_{\text{ap, Theorie}} = (5,72 \pm 0,043) \unit{\kilo\ohm}$. Der experimentell ermittelte Wert wurde 
zu $R_{\text{ap, exp}} = (4,49 \pm 0,01) \unit{\kilo\ohm}$ bestimmt.

\subsection{frequenzabhängigkeit der Kondensatorspannung}
\label{subsec:AuswertungC}

