\section{Theorie}
\label{sec:Theorie}

\subsection{Gedämpfter Schwingkreis}
\label{sec:Theorie_1}

Zuerst wird der gedämpfte Schwingkreis ohne externe Anregung betrachtet. 
Bei Diesem oszilliert zuvor hinzugefügte Energie zwischen der Induktivität L und der Kapazität C,
wobei über den Ohmschen Widerstand R Enrgie in Form von Wärme aus dem System geführt wird.

\begin{figure}
	\centering
    \includegraphics[width=0.4\textwidth]{content/RLC.pdf}
	\caption{Schaltskizze des gedämpften Schwingkreises \cite{v354}.}
	\label{fig:RLC}
\end{figure}

Mit Hilfe der zweiten Kirchhoffschen Regel lässt sich die Gleichung:

\begin{equation}-
    \label{eqn:RLC1}
    U_R(t) + U_C(t) + U_L(t) = 0
\end{equation}

Mit den Bedingungen:

\begin{align}
    \label{eqn:RLC_relations}
        U_R(t) &= R*I(t) & U_L(t) &= L \frac{\symup{d}I}{\symup{d}t} \nonumber \\
        U_C(t) &= \frac{Q(t)}{C} & I(t) &= \frac{\symup{d}Q}{\symup{d}t} 
\end{align}

folgt nach einmaligen differenzieren nach der Zeit und durch Division von L:

\begin{equation}
    \label{eqn:RLC_dgl}
    \frac{\symup{d}^{2}I}{\symup{d}t^2} + \frac{R}{L}\frac{\symup{d}I}{\symup{d}t} + \frac{1}{LC}I = 0   
\end{equation}

Diese lineare, homogene Differentialgleichung zweiter Ordnung wird durch den Exponentialansatz 
$I(t) = A e^{i\tilde{\omega} t} \text{ mit } A, \tilde{\omega} \in \mathbb{C}$ gelöst. Die Lösung lautet dann wie folgt:

\begin{align*}
    \tilde{\omega} &= i \frac{R}{2L} \pm \sqrt{\frac{1}{LC}- \frac{R^2}{4L^2}} & ,\tilde{\omega} &= i2 \pi \mu \pm 2\pi \nu
\end{align*}
\begin{equation}
    \label{eqn:RLC_lsg}
    I(t) = e^{-2 \pi \mu t} (A_1e^{i2 \pi \nu t} + A_2e^{-i2 \pi \nu t})
\end{equation}
$A_1$ und $A_2$ sind wieder komplexe Zahlen und es gilt die Konvention 
\begin{equation*}
    2\pi \mu := \frac{R}{2L} \qquad , 2\pi \nu := \sqrt{\frac{1}{LC}- \frac{R^2}{4L^2}}
\end{equation*}
Man unterscheidet zwei allgemeine Arten von Lösungen:\\
\\
\textbf{1. Fall:} Schwingfall (gedämpfte Schwingung)

Wenn $\nu$ reell ist, also $\frac{1}{LC} > \frac{R^2}{LC}$ gilt, lässt sich die Lösung 
in reeller Form ausdrücken, wenn $A_1 = \overline{A_2}$ gestzt wird und lautet dann:

\begin{equation}
    I(t) = A_0e^{-2\pi \mu t}\cos(2\pi \nu t + \phi) \qquad , \phi, A_0 \in \mathbb{R}
\end{equation}

Die Gleichung beschreibt eine gedämpfte Schwingung mit Schwingungsdauer $T = \frac{1}{\nu}$.
Die sogenannte Abklingdauer beschreibt die Zeit, in der die Amplitude um den Faktor $\frac{1}{e}$ vermindert
wird und ist über die Relation 
\begin{equation}
    \label{eqn:T_ex}
    T_\text{ex} = \frac{1}{2\pi\mu} = \frac{2L}{R}
\end{equation}
gegeben. \\
\\
\textbf{2. Fall:} Aperiodische Dämpfung

Wenn $\nu$ imaginär ist, also $\frac{1}{LC} < \frac{R^2}{LC}$ gilt, wird der Exponent in der Lösung reell
und es liegt keine Schwingung mehr vor. $I(t)$ ist dann proportional zu reellen e-Funktionen:

\begin{equation}
    I(t) \propto e^{(-2\pi\mu-i2\pi\nu)t}
\end{equation}

Der Sonderfall $\nu = 0$ wird aperiodischer Grenzfall genannt. Er tritt ein, wenn der Widerstand $R_\text{ap}$
erreicht wird. Der Wert dieses Widerstands lautet dann:

\begin{equation}
    \label{eqn:R_ap}
    R_\text{ap} =2L \sqrt{\frac{1}{LC}} = 2L \omega_0
\end{equation}
$\omega_0$ ist die Kreisfrequenz des ungedämpften Schwingkreises.



\subsection{Erzwungene Schwingung}
\label{sec:Theorie_2}

Im zweiten Teil des Versuchs wird ein angeregter RLC-Schwingkreis behandelt. Dieser unterscheidet sich zu dem zuvor
beschriebenen Schwingkreis darin, dass eine Wechselspannung $U(t)$ eingeschaltet wird. Dadurch wird kontinuirlich Energie in das 
System eingeführt. Dies bewirkt eine oszillierende Spannung $U_C$ am Kondensator, die in Phase zu der Erregerspannung ist.

\begin{figure}
	\centering
    \includegraphics[width=0.4\textwidth]{content/RLC_angeregt.pdf}
	\caption{Schaltskizze des angeregten Schwingkreises \cite{v354}.}
	\label{fig:RLC_angeregt}
\end{figure}

Gleichung \eqref{eqn:RLC1} ergänzt sich dann mit $U(t) = U_0e^{i\omega t}$ und den Beziehungen \eqref{eqn:RLC_relations} zu

\begin{gather}  
    R I(t) + U_C(t) + L \frac{\symup{d}I(t)}{\symup{d}t} = U_0e^{i\omega t} \nonumber \\
    \Leftrightarrow R \frac{\symup{d}Q(t)}{\symup{d}t} + U_C(t)    + L \frac{\symup{d^2}Q(t)}{\symup{d}t^2} = U_0e^{i\omega t} \nonumber \\
    \Leftrightarrow LC \frac{\symup{d^2}U_C(t)}{\symup{d}t^2} + RC \frac{\symup{d}U_C(t)}{\symup{d}t} + U_C(t) = U_0e^{i\omega t}
    \label{eqn:RLC_angeregt}
\end{gather}

Gleichung \eqref{eqn:RLC_angeregt} kann wieder über den Exponentialansatz $U(t) = U_0(w)e^{i\omega t}$ gelöst werden,
wobei die Amplitude frequenzabhängig ist. Das Einsetzen dieses Ansatzes in \eqref{eqn:RLC_angeregt} führt zu der Gleichung:
\begin{equation*}
    -LC\omega^2 U(\omega) + i\omega RC U(\omega) + U(\omega) = U_0
\end{equation*}
woraus 
\begin{equation}
    \label{eqn:RLC_Amplitude}
    U(\omega) = \frac{U_0 (1-LC\omega^2-i\omega RC)}{(1-LC\omega^2)^2+\omega^2 R^2 C^2}
\end{equation}
für die frequenzabhängige Amplitude folgt. Aus \eqref{eqn:RLC_Amplitude} lässt sich wiederum mit den Rechenregeln für
komplexe Zahlen der Betrag von U und die Phase ($\phi$) berechnen. Es gilt für den Betrag:

\begin{equation}
    \label{eqn:RLC_Betrag}
    |U(\omega)| = |U_C(\omega)| = \frac{U_0}{\sqrt{(1-LC\omega^2)^2+\omega^2R^2C^2}}
\end{equation}
und für die Phase gegenüber der Erregerspannung:

\begin{equation}
    \label{eqn:RLC_Phase}
    \phi(\omega) = \symup{arctan}\left(\frac{-\omega RC}{1-LC\omega^2}\right)
\end{equation}

Aus \eqref{eqn:RLC_Betrag} lässt sich erkennen, dass $U_C$ für den Grenzwert $\omega \to 0$ gegen die Erregerspannung $U_0$
strebt, was insofern sinnvoll ist, da dann ein Gleichstrom anliegt. Für $\omega \to \infty$ geht $U_C$ gegen 0.
Des Weiteren findet sich ein Maximum für $U_C$, welches an der Stelle $\omega_{\text{res}}$ liegt.
$\omega_{\text{res}}$ heißt Resonanzfrequenz und hat den Wert:
\begin{equation}
    \label{eqn:Resonanzfrequenz}
    \omega_{\text{res}} = \sqrt{\frac{1}{LC}-\frac{R^2}{2L^2}}
\end{equation}
Ein interessanter Zusammenhang findet sich im Fall der schwachen Dämpfung: $\frac{R^2}{2L^2} \ll \frac{1}{LC}$. Dann geht $\omega_{\text{res}}$ 
in die Eigenfrequenz $\omega_0$ eines des Systems über und die Kondensatorspannung erreicht den Wert: 
\begin{equation}
    \label{eqn:Guete_Resonanzkurve}
    U_{C, \text{max}} = \frac{1}{\omega_0 RC} U_0 := q U_0
\end{equation}
Der Faktor $q$ wird als Güte der Resonanzkurve bezeichnet. Bei $R \to 0$ geht $q \to \infty$ und somit auch $U_{C, \text{max}} \to \infty$.
Ein weiteres Kennzeichen der Resonanzkurve ist ihre Breite. Diese wird über die Frequenzen $\omega_+$ und $\omega_-$ beschrieben, bei denen
die Kondensatorspannung jeweils den Wert $\sfrac{1}{\sqrt{2}}\; U_{C,\text{max}}$ annimmt. Die gesuchten Frequenzen lassen sich mit Formel
\eqref{eqn:RLC_Betrag} bestimmen. Unter der Annahme, dass eine schwache Dämpfung vorliegt gilt für die Breite der Resonanzkurve:
\begin{gather}
    \Delta\omega := \omega_+ - \omega_- \approx \frac{R}{L} \label{eqn:DeltaOmega} \\
    \text{Für die Güte gilt die Gleichung:} \nonumber \\
    q = \frac{\omega_0}{\Delta\omega}                       \label{eqn:Guete_Resonanzkurve2}
\end{gather}