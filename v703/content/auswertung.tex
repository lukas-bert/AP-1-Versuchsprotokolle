\section{Auswertung}
\label{sec:Auswertung}
Die Fehlerrechnung dieses Kapitels genügt der gaußschen Fehlerfortpflanzung
\begin{equation}
  \label{eqn:Gauss}
  \Delta F = \sqrt{\sum_i\left(\frac{\symup{d}F}{\symup{d}y_i}\Delta y_i \right)^2}.
\end{equation}
%Die Standardfehler des Mittelwertes ergeben sich nach
%\begin{equation*}
%  \label{eqn:MW-Fehler}
%  \sigma(x) = \sqrt{\frac{1}{n(n-1)} \sum_i (x_i - \overline{x})^2}.
%\end{equation*}
Die Fehlerrechnung wird in \textit{Python} unter Verwendung des Paketes \textit{uncertainties} \cite{uncertainties} durchgeführt, jedoch werden die entsprechenden Fehlerformeln
an den jeweiligen Stellen angegeben.

\subsection{Bestimmung der Charakteristik des \textit{GMZ}}
\label{subsec:A_Charakteristik}
Um die Charakteristika des verwendeten Geiger-Müller-Zählrohres zu bestimmen werden die Messwerte aus \autoref{tab:Messwerte} verwendet.
Die Zählraten des \textit{GMZ} folgen einer Poissoin-Verteilung, weshalb sich die Unsicherheit der Messwerte $Z$ als $\symup{\Delta}Z = \sqrt{Z}$ annehmen lässt.
Nach Division durch die Messzeit $t = \qty{120}{\second}$ folgt die anschauliche und vergleichbare Größe $N$, die die Zählraten pro Zeiteinheit beschreibt.
Ihr Fehler ergibt sich nach \eqref{eqn:Gauss} zu 
\begin{equation*}
  \symup{\Delta}N = \frac{\sqrt{Z}}{\qty{120}{\second}}.
\end{equation*} 
Das Auftragen der zeitlich gemittelten Zählraten $N$ gegen die Spannung ergibt die charakteristische Kurve des \textit{GMZ}. Diese ist in \autoref{fig:plot}
dargestellt. Es lässt sich ein Bereich feststellen, in welchem die Kurve annähernd linear verläuft. Dieser ist mit grünen Messpunkten markiert. 

\begin{figure}
  \centering
  \includegraphics{plot.pdf}
  \caption{Messwerte der Charakteristik-Kurve des verwendeten Geiger-Müller-Zählrohres und lineare Ausgleichsgerade. Erstellt mit \textit{matplotlib} \cite{matplotlib}.}
  \label{fig:plot}
\end{figure}

Durch eine lineare Regression mittels \textit{scipy} \cite{scipy} folgen die Parameter
\begin{align*}
  a &= \qty{0.0125+-0.0022}{\volt^{-1}} & b &= \qty{104.4+-1.1}{\volt}
\end{align*}
der Ausgleichsgeraden $f(x) = ax + b$. Dies bedeutet eine Plateau-Steigung von $\qty{1.25+-0.22}{\percent}$. Die Länge des Plateaus ist die Differenz der äußersten Messpunkte,
welche zum Plateau gezählt werden. Sie lautet $L = \qty{270}{\volt}$

\subsection{Bestimmung der Totzeit des Zählrohres}
\label{subsec:A_Totzeit}


\begin{table}
  \centering
  \caption{Messwerte der charakteristischen Kurve des \textit{GMZ} und zur Bestimmung der freigesetzten Ladung. Es wurde jeweils für $t = \qty{120}{\second}$ gemessen.
          $U$ beschreibt die Spannung, $Z$ die Zählraten und $I$ den mittleren Strom. $N$ sind die Zählraten pro Sekunde mit Fehlerangabe.
          Die Unsicherheit der Strommesswerte beträgt $\qty{0.1}{\micro\ampere}$}
  \label{tab:Messwerte}
  \begin{tabular}{S[table-format = 3.0] S[table-format = 5.0] S[table-format = 3.1] @{${}\pm{}$} S[table-format = 1.2] S[table-format = 1.1]}
    \toprule{$U \mathbin{/} \unit{\volt}$} & {$Z$} & \multicolumn{2}{c}{$N \mathbin{/} \unit{\second^{-1}}$} & {$I \mathbin{/} \unit{\micro\ampere}$} \\
      \midrule
      320 & 12692 & 105.8 & 0.94 & 0.1 \\
      330 & 12881 & 107.3 & 0.95 & 0.2 \\
      340 & 12815 & 106.8 & 0.94 & 0.2 \\
      350 & 13066 & 108.9 & 0.95 & 0.2 \\
      360 & 13013 & 108.4 & 0.95 & 0.2 \\
      370 & 13008 & 108.4 & 0.95 & 0.2 \\
      380 & 13054 & 108.8 & 0.95 & 0.2 \\
      390 & 13055 & 108.8 & 0.95 & 0.2 \\
      400 & 13154 & 109.6 & 0.96 & 0.2 \\
      410 & 13094 & 109.1 & 0.95 & 0.2 \\
      420 & 13326 & 111.1 & 0.96 & 0.3 \\
      430 & 13082 & 109.0 & 0.95 & 0.3 \\
      440 & 13252 & 110.4 & 0.96 & 0.3 \\
      450 & 13144 & 109.5 & 0.96 & 0.3 \\
      460 & 13223 & 110.2 & 0.96 & 0.3 \\
      470 & 13247 & 110.4 & 0.96 & 0.4 \\
      480 & 13547 & 112.9 & 0.97 & 0.4 \\
      490 & 13222 & 110.2 & 0.96 & 0.4 \\
      500 & 13214 & 110.1 & 0.96 & 0.4 \\
      510 & 13297 & 110.8 & 0.96 & 0.4 \\
      520 & 13362 & 111.4 & 0.96 & 0.4 \\
      530 & 13292 & 110.8 & 0.96 & 0.4 \\
      540 & 13248 & 110.4 & 0.96 & 0.4 \\
      550 & 13481 & 112.3 & 0.97 & 0.5 \\
      560 & 13411 & 111.8 & 0.97 & 0.5 \\
      570 & 13111 & 109.3 & 0.95 & 0.5 \\
      580 & 13614 & 113.5 & 0.97 & 0.6 \\
      590 & 13223 & 110.2 & 0.96 & 0.6 \\
      600 & 13398 & 111.7 & 0.96 & 0.6 \\
      610 & 13468 & 112.2 & 0.97 & 0.6 \\
      620 & 13458 & 112.2 & 0.97 & 0.6 \\
      630 & 13768 & 114.7 & 0.98 & 0.6 \\
      640 & 13759 & 114.7 & 0.98 & 0.7 \\
      650 & 13494 & 112.5 & 0.97 & 0.7 \\
      660 & 13828 & 115.2 & 0.98 & 0.6 \\
      670 & 13811 & 115.1 & 0.98 & 0.7 \\
      680 & 14186 & 118.2 & 0.99 & 0.8 \\
      690 & 14039 & 117.0 & 0.99 & 0.8 \\
      700 & 14178 & 118.2 & 0.99 & 0.8 \\
    \bottomrule
  \end{tabular}
\end{table}
