\section{Diskussion}
\label{sec:Diskussion}
Im ersten Teil dieses Versuches wurde versucht das Reflexionsgestz \eqref{eqn:Reflexionsgesetz} nachzuweisen. Dazu wurde die absolute Differenz von Einfalls- und 
Ausfallswinkel gebildet. Dabei ergab sich eine mittlere Differenz von $\mathrm{\Delta}\alpha = \qty{2.64}{\degree}$. Aufgrund dieser großen Differenz wurde das 
Reflexionsgesetz \eqref{eqn:Reflexionsgesetz} nicht nachgewiesen. Da die Differenz keinen großen Schwankungen unterliegt ist nicht von einem statistischen Fehler auszugehen.
Diese Konstanz deutet also auf einen systematischen Fehler hin. Dieser liegt vermutlich in den ungenügenden Mitteln des Aufbaus. Durch händische Unterlage einer Schablone kann 
nicht für eine genau Ausrichtung garantirert werden. Außerdem hat die Schablone nicht den ganzen Aufbau unterlegt, weshalb die Winkelablesung nicht gut möglich war. Eine 
ungenaue Einstellung der Messapparatur, wie zum Beispiel eine leichte Neigung oder Versetzung des Spiegels, kann ebenfalls für einen konstanten Fehler sorgen. 

Danach wurde das Brechungsgesetz \eqref{eqn:Brechung} untersucht. Dabei ergab sich experimentell bestimmter Brechungsindex $n_{\text{exp}} = \num{1.486 +- 0.026}$. Dieser
weißt zum Literaturwert $n_{\text{Lit}} = \num{1.489}$ eine Abweichung von $\mathrm{\Delta}n = \qty{0.2}{\percent}$. Aufgrund der sehr geringen Abweichung ist die Messung 
lediglich durch Ablesefehler behaftet. Daher ist diese Messung als qualitativ anzunehmen. Aus dem experimentellen Brechungsindex wurde dann die Lichtgeschwindigkeit in 
Acryl errechnet. Diese ergab sich zu $c_{\text{Acryl}} = \qty{2.018(0.036)e8}{\metre\per\second}$ und kann aufgrund der zugrunde liegenden kleinen Abweichung ebenfalls als 
qualitativ angenommenw werden. 

Nun wurde der Strahlenversatz an einem Prisma untersucht. Dabei wurde der Strahlungsversatz über zwei Methoden berechnet. Zwischen den beiden Methoden ergab sich eine 
maximale Abweichung von $\mathrm{\Delta}s = \qty{2.2}{\percent}$. Daher können beide Methoden als Gleichwertig angenommen werden.

Weiter wurde an einem Prisma der Gesamtreflexionswinkel $\delta$ untersucht. Der \autoref{tab:Prisma} ist zu entnehmen, dass beide verwendeten Methoden eine ähnlich gute 
qualität haben und somit keine eindeutige Präferenz für eine der Methoden ausgesprochen werden kann. 

Zuletzt wurde die Beugung an einem Gitter untersucht. Dabei wurde ein roter Laser der Wellenlänge $\lambda_{\text{theo, rot}} = \qty{635}{\nano\metre}$ und ein grüner Laser
der Wellenlänge $\lambda_{\text{theo, grün}} = \qty{532}{\nano\metre}$ verwendet. Aus der Messung ergaben sich für die beiden Laser experimentell bestimmte Mittelwerte von 
$\overline{\lambda}_{\text{rot}} = \qty{650.18 +- 17.81}{\nano\metre}$ und $\overline{\lambda}_{\text{grün}} = \qty{543.51 +- 14.21}{\nano\metre}$. Für den roten Laser ergibt 
sich daher eine Abweichung von $\mathrm{\Delta}\lambda_{\text{rot}} = \qty{2.4}{\percent}$. Der experimentelle Wert des grünen Laser weicht um 
$\mathrm{\Delta}\lambda_{\text{grün}} = \qty{2.2}{\percent}$ vom angegebenen Wert ab. Da dies beides sehr kleine Abweichungen sind ist die Bestimmung der Wellenlänge 
qualitativ gelungen und unterliegt lediglich kleinen statistischen Fehlern.

Insgesamt wurde die geforderten Gesetze mit zufriedenstellender Genauigkeit bestätigt. Lediglich des Reflexionsgesetz konnte nicht eindeutig bewiesen werden. 