\section{Auswertung}
\label{sec:Auswertung}
Die Fehlerrechnung dieses Kapitels genügt der Gauß'schen Fehlerfortpflanzung
\begin{equation*}
  \label{eqn:Gauss}
  \Delta F = \sqrt{\sum_i\left(\frac{\symup{d}F}{\symup{d}y_i}\Delta y_i \right)^2}.
\end{equation*}
Die Standardfehler des Mittelwertes ergeben sich nach
\begin{equation*}
  \label{eqn:MW-Fehler}
  \sigma(x) = \sqrt{\frac{1}{n(n-1)} \sum_i (x_i - \overline{x})^2}.
\end{equation*}
Die Fehlerrechnung wird in \textit{Python} unter Verwendung des Paketes \textit{uncertainties} \cite{uncertainties} durchgeführt.

\subsection{Reflexionsgesetz}
\label{subsec:A_Reflexion}
Zur Überprüfung des Reflexionsgesetzes werden die in \autoref{tab:Reflexion} aufgeführten Messwerte betrachtet. 
Die Differenz zwischen Einfallswinkel $\alpha_1$ und Ausfallswinkel $\alpha_2$ sollte nach der Theorie $\qty{0}{\degree}$ ergeben.

\begin{table}
  \centering
  \caption{Messwerte zur Überprüfung des Reflexionsgesetzes. $\alpha_1$: Einfallswinkel, $\alpha_2$: Ausfallswinkel.}
  \label{tab:Reflexion}
  \begin{tabular}{S[table-format = 2.0] S[table-format = 2.1] S[table-format = 1.1]}
    \toprule
      {$\alpha_1 \mathbin{/} \unit{\degree}$} & {$\alpha_2 \mathbin{/} \unit{\degree}$} & {$|\alpha_1 - \alpha_2| \mathbin{/} \unit{\degree}$} \\
      \midrule
      25 & 22.5 & 2.5 \\
      30 & 28   & 2   \\
      35 & 32.5 & 2.5 \\
      40 & 37   & 3   \\
      45 & 43   & 2   \\
      50 & 47   & 3   \\
      55 & 51.5 & 3.5 \\
    \bottomrule
  \end{tabular}
\end{table}

Zu den Messwerten kann eine Unsicherheit von jeweils $\qty{1}{\degree}$ angenommen werden, die auf Ableseungenauigkeiten und Abweichungen durch den Aufbau der Apparatur beruht.
Auch in Anbetracht dieser Unsicherheit von insgesamt $\pm \qty{2}{\degree}$ lässt sich das Reflexionsgesetz nicht bestätigen, die Ursachen für dieses Ergebnis werden in
\autoref{sec:Diskussion} erörtert.

\subsection{Brechungsgesetz}
\label{subsec:A_Brechung}
Um das Brechungsgesetz \eqref{eqn:Brechung} zu überprüfen, werden Messwerte der Einfallswinkel $\alpha$ und zugehörige Brechungswinkel $\beta$ aus \autoref{tab:Brechung} verwendet.
Durch Umstellen der \autoref{eqn:Brechung} und Einsetzen des Brechungsindizes von Luft $n_\text{Luft} \approx 1$ ergibt sich der Brechungsindex 
\begin{equation*}
  n = \frac{\sin{\alpha}}{\sin{\beta}}
\end{equation*}
des verwendeten Acrylglases (Plexiglas). Dieser ist ebenfalls zu den jeweiligen Messwerten in \autoref{tab:Brechung} eingetragen. 

\begin{table}
  \centering
  \caption{Messwerte zur Überprüfung des Brechungsgesetzes. $\alpha$: Einfallswinkel, $\beta$: Brechungswinkel.
          Zu Jedem Messwert wird wieder eine Unsicherheit von $\qty{1}{\degree}$ angenommen.
          Die Theoriewerte $\beta_\text{Theo}$ ergeben sich durch das Brechungsgesetz unter Verwendung des im Folgenden berechneten Brechungsindizes.}
  \label{tab:Brechung}
  \begin{tabular}{S[table-format = 2.0] S[table-format = 2.1] S[table-format = 1.2] @{${}\pm{}$} S S[table-format = 2.1] @{${}\pm{}$} S}
    \toprule
      {$\alpha \mathbin{/} \unit{\degree}$} & {$\beta \mathbin{/} \unit{\degree}$} &%
      \multicolumn{2}{c}{$n$} & \multicolumn{2}{c}{$\beta_\text{Theo} \mathbin{/} \unit{\degree} $} \\
      \midrule
      25 & 16.5 & 1.49 & 0.10 & 16.5 & 0.6\\
      30 & 19.5 & 1.50 & 0.09 & 19.7 & 0.6\\
      35 & 23   & 1.47 & 0.07 & 22.7 & 0.7\\
      40 & 25.5 & 1.49 & 0.06 & 25.6 & 0.7\\
      45 & 28.5 & 1.48 & 0.05 & 28.4 & 0.7\\
      50 & 31   & 1.49 & 0.05 & 31.0 & 0.7\\
      55 & 33.5 & 1.48 & 0.04 & 33.5 & 0.8\\
    \bottomrule
  \end{tabular}
\end{table}
Als Mittelwert ergibt sich $n_\text{exp} = \num{1.486+-0.026}$. Der Literaturwert des Brechungsindizes von Acrylglas lautet $n_\text{Lit} = \num{1.489}$ \cite{czichos}.
Aus dem experimentell ermittelten Wert lässt sich die Ausbreitungsgeschwindigkeit von Licht in Acrylglas zu 
\begin{equation*}
  c_\text{Acryl} = \frac{c_\text{Vakuum}}{n_\text{Acryl}} = \qty{2.018(0.036)e8}{\metre\per\second}
\end{equation*}
bestimmen. Dazu wird die Vakuum-Lichtgeschwindigkeit $c_\text{Vakuum} = \qty{2.998}{\metre\per\second}$ \cite{scipy} verwendet.

Der Strahlenversatz $s$ des gebrochenen Strahles zu einem Strahl, welcher den Acrylblock nicht durchläuft kann über \autoref{eqn:Strahlenversatz} berechnet werden. 
Dazu wird die Dicke $d = \qty{5.85}{\centi\metre}$ und der Brechungswinkel $\beta$ benötigt. Für den Brechungwinkel werden zu einer Methode 1 die direkt 
aufgenommenen Messwerte und für eine Methode 2, die aus dem Brechungsgesetz berechneten Werte verwendet. Dabei wird der zuvor bestimmte Mittelwert 
des Brechungsindizes verwendet. Die entsprechenden Werte sind in \autoref{tab:Brechung} eingetragen.
Die Ergebnisse der Rechnung sind in \autoref{tab:Strahlenversatz} aufgeführt.

\begin{table}
  \centering
  \caption{Berechnter Strahlenversatz der zwei genannten Methoden zu den ausgewählten Einfallswinkeln.}
  \label{tab:Strahlenversatz}
  \begin{tabular}{S[table-format = 2.0] S[table-format = 1.2] @{${}\pm{}$} l S @{${}\pm{}$} l}
    \toprule
    {} & \multicolumn{2}{c}{Methode 1} & \multicolumn{2}{c}{Methode 2} \\
      \midrule
      {$\alpha \mathbin{/} \unit{\degree}$} & \multicolumn{2}{c}{$s \mathbin{/} \unit{\centi\metre}$} &% 
      \multicolumn{2}{c}{$s \mathbin{/} \unit{\centi\metre}$} \\
      \midrule
      25 & 0.90 & 0.15 & 0.90 & 0.06 \\
      30 & 1.13 & 0.15 & 1.11 & 0.06 \\
      35 & 1.32 & 0.15 & 1.35 & 0.07 \\
      40 & 1.62 & 0.15 & 1.61 & 0.08 \\
      45 & 1.89 & 0.15 & 1.90 & 0.09 \\
      50 & 2.22 & 0.14 & 2.22 & 0.09 \\
      55 & 2.57 & 0.14 & 2.57 & 0.10 \\
    \bottomrule
  \end{tabular}
\end{table}

\subsection{Brechung im Prisma}
\label{subsec:A_Prisma}
Bei der Untersuchung der Brechung im Prisma, wurden Messwerte der Einfallswinkel $\alpha_1$ und der Austrittswinkel $\alpha_2$ zu einem grünen- und einem roten Laser genommen.
Aus diesen Messwerten lässt sich die Ablenkung $\delta$ gemäß \autoref{eqn:delta} der Strahlen berechnen. Dazu werden die Brechungswinkel $\beta_1$ über das Brechungsgesetz 
\eqref{eqn:Brechung} bestimmt und der Winkel $\beta_2$ über den Zusammenhang $\beta_1 + \beta_2 = \gamma$ berechnet. Der brechende Winkel des Prismas ist 
$\gamma = \qty{60}{\degree}$. Das Prisma besteht aus Kronglas und hat somit einen Brechungsindex von $n_\text{Prisma} =  1.5067$ \cite{czichos}.
Die Messwerte, sowie die Ergebnisse der Rechnung sind \autoref{tab:Prisma} zu entnehmen.

\begin{table}
  \centering
  \caption{Messwerte zur Brechung im Prisma und daraus resultierende Ablenkung $\delta$.}
  \label{tab:Prisma}
  \begin{tabular}{S[table-format = 2.0] | S[table-format = 2.1] S S S}
    \toprule
    {} & \multicolumn{2}{c}{Grüner Laser} & \multicolumn{2}{c}{Roter Laser} \\
      \midrule
      {$\alpha_1 \mathbin{/} \unit{\degree}$} & {$\alpha_2 \mathbin{/} \unit{\degree}$} & {$\delta \mathbin{/} \unit{\degree}$} &%
      {$\alpha_2 \mathbin{/} \unit{\degree}$} & {$\delta \mathbin{/} \unit{\degree}$} \\
      \midrule
      60 & 38.5 & 38.5 & 38   & 38   \\
      50 & 47.5 & 37.5 & 47   & 37   \\
      40 & 60   & 40   & 59   & 39   \\
      30 & 78   & 48   & 76.5 & 46.5 \\
      35 & 66.5 & 41.5 & 65.5 & 40.5 \\
    \bottomrule
  \end{tabular}
\end{table}

\subsection{Beugung am Gitter}
\label{subsec:A_Beugung}
Im letzten Teil des Versuches wird die Beugung eines roten und eines grünen Lasers an einem Gitter untersucht. Dazu werden drei verschiedene Gitter mit den 
Gitterkonstanten $d_1 = \qty{1.67}{\micro\metre}$, $d_2 = \qty{3.3}{\micro\metre}$ und $d_3 = \qty{10}{\micro\metre}$ verwendet. Bei der Messung wurden, wenn dies optisch
möglich war, zwei Beugungswinkel pro Ordnung $k$ gemessen, da theoretisch eine symmetrische Verteilung um das Lot auftreten sollte.
Die augenommenen Messwerte sind in den Tabellen \ref{tab:Beugung600}, \ref{tab:Beugung300} und \ref{tab:Beugung100} dargestellt. Aus den aufgenommenen Beugungswinkeln kann 
für die $k$-te Ordnung mit der zugehörigen Gitterkonstante $d$ gemäß \autoref{eqn:Beugung} eine experimentelle Wellenlänge bestimmt werden. Dies wird zu jedem Beugungswinkel
durchgeführt und die daraus resultierenden Wellenlängen werden in den Tabellen \ref{tab:Beugung600}, \ref{tab:Beugung300} und \ref{tab:Beugung100} dargestellt.

\begin{table}
  \centering
  \caption{In dieser Tabelle sind die Messwerte zur Beugung am Gitter der Gitterkonstante $\qty{1.67}{\micro\metre}$, sowie die daraus resultierenden Wellenlängen aufgeführt.}
  \label{tab:Beugung600}
  \begin{tabular}{c S[table-format = 2.1] S[table-format = 2.1] S[table-format = 3.2] S[table-format = 3.2]}
    \toprule
    {Ordnung $k$} & {$\alpha_{\text{grün}} \mathbin{/}\unit{\degree}$} & {$\alpha_{\text{rot}} \mathbin{/}\unit{\degree}$} & {$\lambda_{\text{grün}} \mathbin{/}\unit{\nano\metre}$} & {$\lambda_{\text{rot}} \mathbin{/}\unit{\nano\metre}$}\\
      \midrule
      1 & 19.5 & 23.5 & 556.34 & 664.58 \\
      1 & 20.0 & 24.0 & 570.03 & 677.89 \\
      \bottomrule
  \end{tabular}
\end{table}

\begin{table}
  \centering
  \caption{In dieser Tabelle sind die Messwerte zur Beugung am Gitter der Gitterkonstante $\qty{3.3}{\micro\metre}$, sowie die daraus resultierenden Wellenlängen aufgeführt.}
  \label{tab:Beugung300}
  \begin{tabular}{c S[table-format = 2.1] S[table-format = 2.1] S[table-format = 3.2] S[table-format = 3.2]}
    \toprule
    {Ordnung $k$} & {$\alpha_{\text{grün}} \mathbin{/}\unit{\degree}$} & {$\alpha_{\text{rot}} \mathbin{/}\unit{\degree}$} & {$\lambda_{\text{grün}} \mathbin{/}\unit{\nano\metre}$} & {$\lambda_{\text{rot}} \mathbin{/}\unit{\nano\metre}$}\\
      \midrule
      
      1 & 9.0  & 11.0 & 521.45 & 636.03 \\
      1 & 9.5  & 11.0 & 550.16 & 636.03 \\
      2 & 19.0 & 22.5 & 542.61 & 637.81 \\
    	2 & 19.0 & 23.0 & 542.61 & 664.58 \\    
    	3 & 29.0 & 35.0 & 538.68 & 637.31 \\
    	3 & 29.0 & 35.0 & 538.68 & 637.31 \\

      \bottomrule
  \end{tabular}
\end{table}

\begin{table}
  \centering
  \caption{In dieser Tabelle sind die Messwerte zur Beugung am Gitter der Gitterkonstante $\qty{10}{\micro\metre}$, sowie die daraus resultierenden Wellenlängen aufgeführt.}
  \label{tab:Beugung100}
  \begin{tabular}{c S[table-format = 2.1] S[table-format = 2.1] S[table-format = 3.2] S[table-format = 3.2]}
    \toprule
    {Ordnung $k$} & {$\alpha_{\text{grün}} \mathbin{/}\unit{\degree}$} & {$\alpha_{\text{rot}} \mathbin{/}\unit{\degree}$} & {$\lambda_{\text{grün}} \mathbin{/}\unit{\nano\metre}$} & {$\lambda_{\text{rot}} \mathbin{/}\unit{\nano\metre}$}\\
      \midrule
      
      1 &  3.0 &  4.0 & 523.36 & 697.56 \\
      1 &  3.0 &  3.5 & 523.36 & 610.49 \\
      2 &  6.0 &  7.5 & 522.64 & 652.63 \\
    	2 &  6.0 &  7.5 & 522.64 & 652.63 \\    
    	3 &  9.5 & 11.0 & 550.16 & 636.03 \\
    	3 &  9.5 & 11.5 & 550.16 & 664.56 \\
      4 & 12.5 & 15.0 & 541.10 & 647.05 \\
      4 & 13.0 & 15.0 & 562.38 & 647.05 \\
      5 & 16.0 & 19.0 & 551.27 & 651.14 \\
      6 & 19.5 & 23.0 & 556.34 & 651.22 \\
      7 & 23.0 & 27.0 & 558.19 & 648.56 \\
      8 & 26.0 & 31.5 & 547.96 & 653.12 \\
      
      \bottomrule
  \end{tabular}
\end{table}

Aus den berechneten Wellenlängen wird nun ein Mittelwert der Wellenlänge gebildet. Für den grünen Laser ergibt sich eine Wellenlänge von 
$\overline{\lambda}_{\text{grün}} = \qty{543.51 +- 14.21}{\nano\metre}$. Der Fehler genügt dem Mittelwertfehler gemäß der Formel \eqref{eqn:MW-Fehler}.
Für den roten Laser wird eine Wellenlänge von $\overline{\lambda}_{\text{rot}} = \qty{650.18 +- 17.81}{\nano\metre}$ bestimmt.
