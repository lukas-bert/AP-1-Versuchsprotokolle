\section{Auswertung}
\label{sec:Auswertung}
In \autoref{tab:Mess1} sind die Messwerte zu den Temperaturen des kalten- und warmen Reservoirs in °C, der Druck in den jeweiligen Leitungen in Bar, sowie die Leistungsaufnahme 
des Kompressors in Watt, in Abhängigkeit zur Zeit dargestellt. Zu den Drücken $p_\text{warm}$ ($p_b$) und $p_\text{kalt}$ ($p_a$) wurde bereits der Atmosphärendruck von 
$1 \unit{\bar}$ addiert.

Die Ausgleichs- und Fehlerrechnungen zu den Auswertungsaufgaben wurden mittels Python (Version 3.9.7) unter Verwendung der Pakete \textit{scipy} \cite{scipy}, \textit{numpy}
\cite{numpy} und \textit{uncertainties} \cite{uncertainties} durchgeführt. Die Fehlerformeln finden sich gleichwohl an den entsprechenden Stellen.

\begin{table}
  \centering
  \caption{Gemessene Werte für Temperatur, Druck und Leistung.}
  \label{tab:Mess1}
  \begin{tabular}{S[table-format=2] S[table-format=2.1] S S[table-format=1.1] S S[table-format=3]}
      \toprule
      {$t \mathbin{/} \text{min}$} & {$T_{\text{kalt}} \mathbin{/} \unit{\degreeCelsius}$} &%
      {$T_{\text{warm}} \mathbin{/} \unit{\degreeCelsius}$} & {$p_{\text{kalt}} \mathbin{/} \unit{\bar}$} &%
      {$p_{\text{warm}} \mathbin{/} \unit{\bar}$} & {$P \mathbin{/} \unit{\watt}$} \\
      \midrule
       1 & 21.7 & 23.0 & 2.6 &  7.0 & 170 \\
       2 & 21.4 & 23.7 & 2.8 &  7.5 & 175 \\
       3 & 20.3 & 25.0 & 3.0 &  7.7 & 190 \\
       4 & 18.9 & 26.4 & 3.1 &  8.0 & 195 \\
       5 & 17.1 & 28.2 & 3.2 &  8.5 & 200 \\
       6 & 15.4 & 29.9 & 3.2 &  8.9 & 205 \\
       7 & 13.6 & 31.7 & 3.2 &  9.2 & 200 \\
       8 & 12.0 & 33.5 & 3.2 &  9.5 & 200 \\
       9 & 10.4 & 35.0 & 3.2 & 10.0 & 205 \\
      10 &  8.8 & 36.8 & 3.2 & 10.3 & 210 \\
      11 &  7.0 & 38.4 & 3.2 & 10.6 & 210 \\
      12 &  5.7 & 40.1 & 3.2 & 11.0 & 210 \\
      13 &  4.3 & 41.6 & 3.2 & 11.4 & 210 \\
      14 &  3.2 & 43.0 & 3.2 & 11.7 & 215 \\
      15 &  2.5 & 44.3 & 3.2 & 12.0 & 210 \\
      16 &  1.8 & 45.7 & 3.2 & 12.4 & 210 \\
      17 &  1.2 & 46.9 & 3.2 & 12.7 & 210 \\
      18 &  0.7 & 48.1 & 3.2 & 13.0 & 210 \\
      19 &  0.4 & 49.2 & 3.2 & 13.4 & 205 \\
      20 &  0.1 & 50.2 & 3.2 & 13.6 & 205 \\
      \bottomrule
      \end{tabular}
  \end{table}

\subsection{Auswertung der Temperaturverläufe mittels Ausgleichsrechnung}
\label{subsec:Temperaturverlauf}
Zuerst sollten die Temperaturverläufe der Messreihe graphisch dargestellt werden. Anhand der Messdaten kann anschließend durch eine nicht-lineare Regression der Kurvenverlauf 
approximiert werden. Als Ansatz wurde ein Polynom zweiten Grades gewählt
\begin{equation*}
  T(t) = At^2 + Bt + C
\end{equation*}
mit den reellen Koeffizienten $A$, $B$ und $C$, die sich aus der Regression ergeben. \autoref{fig:Plot1} zeigt die Temperaturverläufe der beiden Reservoire
und die jeweiligen Ausgleichsfunktionen.
\begin{figure}
  \caption{Temperaturverläufe in Abhängigkeit zur Zeit und quadratischer Fit zu den Werten. (Erstellt mit dem Paket \textit{matplotlib} \cite{matplotlib})}
  \label{fig:Plot1}
  \centering
  \includegraphics[width=0.8\textwidth]{build/plot1.pdf}
\end{figure}

Für die gesuchten Parameter ergeben sich die Werte
\begin{align*}
  A &= (1.035 \pm 0.162)\cdot 10^{-5} \unit{\kelvin\per\second\squared} & B &= (-0.034 \pm 0.002) \unit{\kelvin\per\second} & C &= (298.79 \pm 0.58) \unit{\kelvin} \\
\end{align*}
für das kalte Gefäß und 
\begin{align*}
  A &= (0.373 \pm 0.094)\cdot 10^{-5} \unit{\kelvin\per\second\squared} & B &= (-0.03 \pm 0.001) \unit{\kelvin\per\second} & C &= (293.19 \pm 0.33) \unit{\kelvin} \\
\end{align*}
für das warme Gefäß.

Mit diesen Parametern lassen sich nun die Differentialquotienten $\symup{d}T_1/\symup{d}t$ und $\symup{d}T_2/\symup{d}t$ der Temperaturverläufe punktweise über
die Ableitung der quadratischen Ausgleichsfunktionen bestimmen.
Die Ableitung der Funktion ist durch
\begin{equation*}
  \frac{\symup{d}T}{\symup{d}t} (t) = 2 At + B
\end{equation*}
gegeben. Der Fehler des Differentialquotienten beträgt dann nach der gaußschen Fehlerfortpflanzung 
\begin{equation*}
  \symup{\Delta} \frac{\symup{d}T}{\symup{d}t} = \sqrt{(2t\symup{\Delta}A)^2 + \symup{\Delta}B^2}.
\end{equation*}
Der Differentialquotient beider Temperaturkurven sollte für je 4 verschiedene Temperaturen ermittelt werden. Ausgewählt wurden die Temperaturen zu den Zeitpunkten
$t_1 = 3 \symup{min}$ $(180 \unit{\second})$, $t_2 = 8 \symup{min}$ $(480 \unit{\second})$, $t_3 = 13 \symup{min}$ $(780 \unit{\second})$ und 
$t_4 = 18 \symup{min}$ $(1080 \unit{\second})$. Die Werte können \autoref{tab:dT} entnommen werden.

\begin{table}
  \centering
  \caption{Berechnete Werte der Differentialquotienten zu den ausgewählten Zeitpunkten}
  \label{tab:dT}
  \begin{tabular}{c c @{${}\pm{}$} c c @{${}\pm{}$} c c @{${}\pm{}$} c c @{${}\pm{}$} c}
    \toprule
    & \multicolumn{2}{c}{$\frac{\symup{d}T}{\symup{d}t}(t_1) \mathbin{/} \frac{\symup{K}}{\symup{s}}$} & \multicolumn{2}{c}{$\frac{\symup{d}T}{\symup{d}t}(t_2) \mathbin{/} \frac{\symup{K}}{\symup{s}}$} &%
    \multicolumn{2}{c}{$\frac{\symup{d}T}{\symup{d}t}(t_3) \mathbin{/} \frac{\symup{K}}{\symup{s}}$} & \multicolumn{2}{c}{$\frac{\symup{d}T}{\symup{d}t}(t_4) \mathbin{/} \frac{\symup{K}}{\symup{s}}$}\\
    \midrule
    {kaltes Reservoir} & -0.031 & 0.002 & -0.024 & 0.003 & -0.018 & 0.003 & -0.012 & 0.004\\
    {warmes Reservoir} & 0.029 & 0.001 & 0.026 & 0.002 & 0.024 & 0.002 & 0.022 & 0.002\\
    \bottomrule
  \end{tabular}
\end{table}

\subsection{Vergleich der realen- und idealen Güteziffer}
\label{subsec:Güteziffer}
Aus den Werten des Differentialquotienten kann man mit \autoref{eqn:Güte_Messung} die reale Güteziffer bestimmen. $m_k c_k$ ist die Wärmekapazität der Gefäße und der Leitungen und
hat den Wert $m_k c_k = 750 \, \unit{\joule\per\kelvin}$. $m_1 c_w$ ist die Wärmekapazität des Wassers im warmen Reservoir. Sie berechnet sich mit der Masse $m_1 = 3\unit{\kilogram}$
und der spezifischen Wärme von Wasser $c_w \approx 4181.8 \, \unit{\joule\per\kilogram\kelvin}$ \cite{Ingenieurwissen}. Da der Differentialquotient fehlerbelastet ist, hat die
daraus berechnete Güteziffer einen Fehler, der sich auch nach \autoref{eqn:Güte_real} durch einsetzen des Fehlers von $\symup{d}T\mathbin{/}\symup{d}t$ berechnen lässt.
Die Theoriewerte ergeben sich mit \autoref{eqn:Güte_ideal}. 

\begin{table}
  \centering
  \caption{Vergleich der realen- und idealen Güteziffern an den Zeitpunkten $t_i$}
  \label{tab:Güteziffer}
  \begin{tabular}{c S[table-format = 2.2] S[table-format = 1.2] @{${}\pm{}$} S}
    \toprule
    {$t \mathbin{/} \symup{s}$} & {$\nu_{\text{ideal}}$} & \multicolumn{2}{c}{$\nu_{\text{real}}$} \\
    \midrule
    180  & 63.44 & 2.01 & 0.09 \\ 
    480  & 14.26 & 1.76 & 0.10 \\
    780  &  8.44 & 1.53 & 0.12 \\
    1080 &  6.78 & 1.39 & 0.15 \\
    \bottomrule
  \end{tabular}
\end{table}

\subsection{Bestimmung der Verdampfungswärme und des Massendurchsatzes}
\label{subsec:Massendurchsatz}
Nun soll der Massendurchsatz des Transportgases Dichlordifluormethan ($\text{Cl}_2\text{F}_2\text{C}$) berechnet werden. Dies lässt sich mit \autoref{eqn:Massendurchsatz} bewerkstelligen. 
$\symup{d}T_2/\symup{d}t$ wurde bereits in \autoref{subsec:Temperaturverlauf} bestimmt und $m_2 c_w$ ist gleich dem Wert von $m_1 c_w$ aus \autoref{subsec:Güteziffer}.
Die Verdampfungswärme $L$ ist stoffspezifisch und nicht gegeben. Daher muss sie aus den Messwerten bestimmt werden. Dazu wird die Dampfdruckkurve von Dichlordifluormethan
betrachtet. Sie ergibt sich durch Auftragen des Druckverhätnisses in einem System gegen die Temperatur. Näheres wird in Versuch 203 \cite{v203} beschrieben. 
Da es einfacher ist eine lineare Ausgleichsrechnung durchzuführen, wird in \autoref{fig:Plot2} der Logarithmus des Druckverhätnisses, also
$\symup{log}(\frac{p_{\text{b}}}{p_0})$, an der $y$-Achse aufgetragen. $p_b$ ist gleich zu setzen mit $p_\text{warm}$ und $\frac{p_{\text{b}}}{p_0}$ beschreibt das Druckverhätnis
zum Atmosphärendruck $p_0$.

\begin{figure}
  \centering
  \caption{Dampfdruckkurve von Dichlordifluormethan. Messwerte und linearer Fit in halblogarithmischer Darstellung. (Erstellt mit \textit{matplotlib}\cite{matplotlib})}
  \label{fig:Plot2}
  \includegraphics[width=0.8\textwidth]{build/plot2.pdf}
\end{figure}

In dem Diagramm sind die Wertepaare des warmen Reservoirs eingetragen, aus denen sich mittels linearer Regression die Parameter $m$ und $b$ der Geraden
\begin{equation*}
  f(x) = mx + b
\end{equation*}
bestimmen lassen. $x$ ist in diesem Fall Der Kehrwert der Temperatur und $y = f(x)$ der oben beschriebene Logarithmus. Es gilt der Zusammenhang $m = - \frac{L}{R}$ für 
die Verdampfungswärme, auch dies wird in Versuch 203 \cite{v203} ersichtlich. Die Parameter lassen sich rechnerisch als
\begin{align*}
  m &= \frac{\overline{xy}-\overline{x} \cdot \overline{y}}{\overline{x^2}-\overline{x}^2} & %  
  b = \frac{\overline{y} \cdot \overline{x^2}-\overline{xy} \cdot \overline{x}}{\overline{x^2}-\overline{x}^2}
\end{align*}
bestimmen, wurden jedoch mit dem Paket \textit{scipy} \cite{scipy} berechnet. Der Fehler dieser Größen ergibt sich mit der gaußschen Fehlerfortpflanzung.

Aus den Messwerten konnten die Parameter $m = (-2186 \pm 37) \unit{\kelvin}$ und $b = 9.4 \pm 0.1$ ermittelt werden. Daraus folgt für $L = -m/R$ mit der
allgemeinene Gaskonstante $R \approx 8.314$ \cite{scipy} der Wert der Verdampfungswärme $L = (18.2 \pm 0.3) \unit{\kilo\joule\per\mol}$. Die molare Masse von Dichlordifluormethan
beträgt $M = 120.91 \unit{\gram\per\mol}$ \cite{Dichlordifluormethan}, daraus folgt $L = (150 \pm 2.5) \unit{\kilo\joule\per\kilogram}$.


Nun kann der Massendurchsatz für die verschiedenen Temperaturen gemäß \autoref{eqn:Massendurchsatz} berechnet werden. Diese werden zusammen mit dem jeweiligen Fehler in \autoref{tab:Massendurchsatz}
dargestellt.
\begin{table}
  \centering
  \caption{Massendurchsatz zu den Zeitpunkten $t_i$}
  \label{tab:Massendurchsatz}
  \begin{tabular}{c S[table-format = 1.2] @{${}\pm{}$} S[table-format = 1.2]}
    \toprule
    {$t_i \mathbin{/} \unit{\second}$} &  \multicolumn{2}{c}{$\frac{\text{d}m}{\text{d}t}(t)\cdot {10^{-3}}\unit{\kilogram\per\second}$} \\
    \midrule
    180  & 2.71 & 0.20 \\ 
    480  & 2.16 & 0.23 \\
    780  & 1.61 & 0.29 \\
    1080 & 1.06 & 0.36 \\
    \bottomrule
  \end{tabular}
\end{table}
Die Fehler des Massendurchsatzes ergeben sich gemäß der gaußschen Fehlerfortpflanzung durch die Gleichung
\begin{equation*}
  \label{eqn:Fehlerformeldm}
  \Delta\frac{\text{d}m}{\text{d}t} = \sqrt{\left(-\frac{m_2c_w + m_kc_k}{L^2}\frac{\Delta T_2}{\Delta t}\Delta L\right)^2+\left(\frac{m_2c_w + m_kc_k}{L}\Delta\frac{\Delta T_2}{\Delta t}\right)^2}.
\end{equation*}
\subsection{mechanische Leistung des Kompressors}
\label{subsec:Leistung}
Zuletzt wird nun die mechanische Leistung des Kompressors ausgewertet, welche dieser abgibt, um mit den Drücken $p_a$ und $p_b$ zu arbeiten. Diese Leistung lässt sich gemäß \autoref{eqn:Leistung}
berechnen. Die in \autoref{subsec:Kenngrößen} bereits diskutierten notwendigen Materialkonstanten sind für Dichlordifluormethan bekannt. 
\begin{align}
  \rho_0 = 5.51 \unit{\gram\per\liter} \,\,\text{,} \,\,p_0 = 1 \unit{\Bar}\,\, \text{,}\,\, \kappa = 1,14
\end{align}
$p_0$ entspricht hier dem Atmosphärendruck und die Angabe von $\rho_0$ bezieht sich auf T$= 0 \unit{\degreeCelsius}$.
Die Dichte von Dichlordifluormethan zu den verschiedenen Temperaturen ergibt sich durch die Gleichung
\begin{equation*}
  \rho(T) = \frac{\rho_0 T_0 \cdot p_b}{p_0 T_2}, 
\end{equation*}
welche aus der idealen Gasgleichung folgt. 
Die damit errechneten Werte für die Leistung können nun \autoref{tab:Leistung} entnommen werden.
\begin{table}
  \centering
  \caption{Leistungsabgabe zu den Zeitpunkten $t_i$}
  \label{tab:Leistung}
  \begin{tabular}{c S[table-format = 4.1] @{${}\pm{}$} S[table-format = 1.1]}
    \toprule
    {$t_i \mathbin{/} \unit{\second}$} & \multicolumn{2}{c}{$N_{\text{mech}}\mathrm{/}\unit{\watt}$} \\
    \midrule
    180  & -18.0 & 1.3 \\ 
    480  & -14.1 & 1.5 \\
    780  & -10.1 & 1.8 \\
    1080 &  -6.4 & 2.2 \\
    \bottomrule
  \end{tabular}
\end{table}

Auch hier werden die Fehler gemäß der gaußschen Fehlerfortpflanzung bestimmt. 
\begin{equation*}
  \Delta N_{\text{mech}} = \frac{1}{\kappa -1}\left(p_b\sqrt[\kappa]{\frac{p_a}{p_b}}-p_a\right)\frac{1}{\rho}\Delta\frac{\text{d}m}{\text{d}t}.
\end{equation*}
