\section{Diskussion}
\label{sec:Diskussion}
Über die Differentialquotienten $\frac{\Delta T_1}{\Delta t}$ und $\frac{\Delta T_2}{\Delta t}$, die Verdampfungswärme $L$, sowie den Massendurchsatz $\frac{\text{d}m}{\text{d}t}$
lassen sich keine Vergleiche zu Theoriewerten ziehen. 
Anders ist dies bei der Güteziffer. Diese ist sehr viel kleiner als der Theoriewert $\nu_{\text{ideal}}$ für die ideale Güteziffer. 
Mögliche Ursachen dafür gibt es reichlich. So arbeitet
der Kompressor real nicht adiabatisch, die Reservoire und Rohrleitungen können nicht vollständig isoliert werden, sodass 
ein konstanter Wärmeaustausch mit der Umgebung stattfindet. Außerdem wurden 
Reibungsverluste vernachlässigt, welche beim Gastransport weniger wichtig sind, aber bei den Rührmotoren im Wasser für eine zusätzliche Erwärmung der Reservoire sorgen können. 
Des Weiteren war es unmöglich den Messprozess exakt durchzuführen, da es händisch nicht möglich ist alle relevanten Werte gleichzeitig festzuhalten. Dazu kommen noch die 
Ablesegenauigkeiten der analogen Messgeräte von Druck und Leistung. Weiterhin arbeitet die gesamte Wärmepumpe bei höherer Temperaturdifferenz
immer ineffizienter, wodurch auch die Ungenauigkeiten insgesamt immer größer werden. 

Dennoch steht all dies nicht in Widerspruch zur Theorie, da die reale Güteziffer eben kleiner
als die Ideale ist. Auch die (fallende) Monotonie der aus den Messwerten berechneten Güteziffern ist eigentlich eine Bestätigung der Theorie, da diese den oben genannten
Effizienzverlust bei steigendem Temperaturunterschied vorhersagt. Selbiges gilt für die mechanische Leistung $N$, dessen berechnetete Werte weitaus kleiner als die Messwerte
ausfallen. Der Trend der sinkenden Effizienz ist auch hier zu erkennen. 
Zusammfassend ergibt sich, dass die reale Wärmepumpe, die in diesem Versuch verwendet wurde, sehr stark von der idealen Wärmepumpe abweicht, aber in der Funktionsweise der
Theorie entspricht.