\section{Diskussion}
\label{sec:Diskussion}
Zunächst wird der gemessene Temperaturverlauf diskutiert. Man kann diese Messwerte nicht mit einer Literatur vergleichen, da der Messprozess von sehr vielen Parametern abhängt. 
Über die Differentialquotienten $\frac{\Delta T_1}{\Delta t}$ und $\frac{\Delta T_2}{\Delta t}$, die Verdampfungswärme L, den Massendurchsatz $\frac{\text{d}m}{\text{d}t}$ und 
die vom Kompressor abgegebene Leistung $N_{mechanisch}$ lassen sich keine Vergleiche zu Theoriewerten und ähnlichem ziehen. Lediglich kann angemertk werden das die Ungenauigkeiten 
dieser Werte nicht aus dem Ruder laufen. Anders ist dies bei der Güteziffer. Diese weicht sehr stark Güteziffer von ihrem Theoriewert ab. Mögliche Fehler gibt es reichlich. So arbeitet
der Kompressor real nicht adiabatisch. Die Reservoire waren nicht wirklich gut isoliert so das ein konstanter Wärmeaustausch mit der Umgebung stattfinden konnte. Außerdem wurden 
Reibungsverluste vernachlässigt, welche beim Gastransport weniger wichtig sind, aber beim Rotor im Wasser sorgt dies für zusätzliche Erwärmung der Reservoire. 
Der Messprozess konnte auch nicht perfekt minütlich durchgeführt werden, da es händisch nicht möglich ist alle relevanten Werte in einem Moment festzuhalten. Dazu kommen noch die 
schlechten Messgeräte vom Druck und der Leistung, da diese keine hohe Ablesegenauigkeit geboten haben. Außerdem arbeitet die gesamte Wärmepumpe bei höherer Temperaturdifferenz
immer schlechter wodruch auch die Ungenauigkeiten insgesamt immer größer werden. Zusammfassend ergibt sich eine relativ hohe Ungenauigkeit in diesem Experiment. 