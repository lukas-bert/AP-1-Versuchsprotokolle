\section{Diskussion}
\label{sec:Diskussion}
\begin{table}
    \centering
    \caption{Abweichung der Messwerte zur Theorie}
    \sisetup{table-format=1.3}
    \begin{tabular}{c S @{${}\pm{}$} S c S[table-format=2.3] @{${}\pm{}$} S[table-format=1.3] c}
    \toprule
    & \multicolumn{2}{c}{$l = 28.4\unit{\centi\metre}$} & \multicolumn{1}{c}{$\symup{\Delta}_\text{relativ}$} & \multicolumn{2}{c}{$l = 78.4\unit{\centi\metre}$} & \multicolumn{1}{c}{$\symup{\Delta}_\text{relativ}$}\\
    \cmidrule(lr){2-3}\cmidrule(lr){4-4}\cmidrule(lr){5-6}\cmidrule(lr){7-7}
    {$\omega_+ \mathbin{/} \unit{\hertz}$}                     & 5.124 & 0.019 &  & 3.530 & 0.009 & \\
    {$\omega_{+,\text{Theorie}} \mathbin{/} \unit{\hertz}$}    & 5.876 & 0.01 & 12.8\% & 3.537 & 0.002 & 0.19\% \\
    {$\omega_- \mathbin{/} \unit{\hertz}$}                     & 6.147 & 0.011 & & 3.834 & 0.010 & \\
    {$\omega_{{-}\text{, Theorie}} \mathbin{/} \unit{\hertz}$} & 5.983 & 0.003 & 2.73\% & 3.566 & 0.001 & 7.52\% \\
    {$\omega_s \mathbin{/} \unit{\hertz}$}                     & 1.023 & 0.007 & & 0.315 & 0.002 & \\
    {$\omega_{{s}\text{, Theorie}} \mathbin{/} \unit{\hertz}$} & 0.107 & 0.011 & 857\% & 0.030 & 0.002 & 961\% \\
    \bottomrule
    \end{tabular}
\end{table}


Zunächst wird die gleichsinnige Schwingung diskutiert. Bei diesen Messungen ergibt sich für $l_1$ eine Abweichung 
$\symup{\Delta}\omega_{+} = 12.8\%$ vom Theoriewert. Diese liegt nicht im Fehlerbereich des gemessenen Wertes und kann durch die Überlänge des Pendels entstehen, da die Masse sehr
weit oben aufgehangen ist (siehe \autoref{fig:Aufbau2}).\\ Für die Massenlage $l_2$ ergibt sich lediglich eine prozentuale Abweichung $\symup{\Delta}\omega_{+} = 0.19\%$. Dieser Wert
liegt im Rahmen des errechneten Fehlers und ist somit qualitativ gut. Es war allerdings zu erwarten das dieser Wert genauer ist, da die Masse hier sehr weit unten aufgehangen war
und das Pendel dadurch genauer wird. 


Betrachtet man die Federkonstante fällt auf, dass diese für die unterschiedlichen Längen einen anderen Wert ergibt. Dies ergibt sich, da man die Federkonstante aus den experimentell
bestimmten fehlerbehafteten Werten errechnen soll. In Wahrheit sollte die Federkonstante aber gleich sein, da bei den Messungen jeweils die gleiche Feder genutzt wurde. Daher ist die 
Aussagekraft der Theoriewerte zweifelhaft.


Allerdings ist die Kopplungskonstante der genutzten Feder nicht angegeben und sollte aus gemessenen fehlerbehafteten Werten bestimmt werden, wodurch der Theoriewert nicht korrekt bestimmt 
werden konnte.

Zur gegensinnigen Schwingung der Länge $l_1$ lässt sich die Abweichung $\symup{\Delta}\omega_{-} = 2.73\%$ bestimmen. Diese Abweichung liegt um einen kleinen Wert nicht im Fehlerbereich.
Aufgrund der kleinen Abweichung des Wertes zur Länge $l_1$ kann man die Messung als qualitativ annehmen. Bei der Länge $l_2$ liegt eine Abweichung von $\symup{\Delta}\omega_{-} = 7.53\%$
vor. Diese ist trotz des besseren Pendels schlechter im vergleich zur ersten Messung.

Die Schwebefrequenz weicht für beide Längen um über 800\% ab. Da sich der Theoriewert der Schwebungsfrequenzen aus der Differenz von $\omega_-$ und $\omega_+$ zusammensetzt entsteht 
aus den zunächst im Rahmen der Messungenauigkeit akzeptablen Abweichungen der einzelnen Werte eine große Abweichung der Schwebungsfrequenz. Berechnet man hingegen aus den experimentell
bestimmen Werten $\omega_-$ und $\omega_+$ die Schwebungsfrequenzen, ergeben sich sehr geringe Abweichungen von $\symup{\Delta}\omega_{s} = 0.02\%$ für $l_1$ und $\symup{\Delta}\omega_{s} = 3.31\%$ 
für $l_2$. 


Allgemein liegen in diesem Experiment noch Fehlerquellen vor. Durch händische Auslenkung ist es nicht möglich beide Pendel gleichmäßig auszulenken und sich genau im Rahmen der
Kleinwinkelnäherung aufzuhalten. Dazu muss die Schwingungsdauer durch eine Stoppuhr mit optischen abpassen der maximalen Auslenkung bestimmt werden. Daher kommen Fehler durch 
Reaktionszeit und durch Erkennungsungenauigkeit dazu. Außerdem war das Pendel leicht gekrümmt und ist somit auf nicht perfekt in einer Ebene geschwungen wodurch weitere Abweichungen 
bei der Schwingungs- und Schwebedauer hinzukommen.


Unter betrachtung dieser Fehlerquellen sind, bis auf die bereits diskutierte Abweichung der Schwebefrequenz, alle Abweichungen akzeptabl bestimmt worden.  


