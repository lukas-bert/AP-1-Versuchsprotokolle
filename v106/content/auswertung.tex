\section{Auswertung}
\label{sec:Auswertung}
Im Folgenden werden alle zu untersuchenden Werte für beide Pendellängen bestimmt. In der ersten Messung wurde die Pendellänge auf $(28.4\pm 0.1)\unit{\centi\metre}$ eingestellt. 
Für den zweiten Messdurchgang wurde die Länge auf $(78.4\pm 0.1)\unit{\centi\metre}$ geändert.

\begin{table}
    \centering
    \caption{Messwerte zur Pendellänge $l_1 = {28.4}\unit{\centi\metre}$} 
    \label{tab:Mess1}
    \begin{tabular}{S S S S S S[table-format=1.2]}
        \toprule
        $\unit{{5}T_{1}\per\second}$ & $\unit{{5}T_{2}\per\second}$ & $\unit{{5}T_+\per\second}$ & $\unit{{5}T_-\per\second}$ & $\unit{{3}T\per\second}$ & $\unit{T_s\per\second}$ \\
        \midrule
        6.18 & 5.98 & 6.11 & 5.07 & 3.32 & 6.06 \\
        6.11 & 5.79 & 6.15 & 5.07 & 3.45 & 6.20 \\
        6.11 & 5.99 & 6.04 & 5.13 & 3.23 & 6.20 \\
        5.91 & 5.93 & 5.99 & 5.08 & 3.43 & 6.35 \\
        6.10 & 5.99 & 6.19 & 5.11 & 3.39 & 5.87 \\
        5.85 & 6.18 & 6.16 & 5.11 & 3.36 & 6.07 \\
        5.89 & 5.97 & 6.14 & 5.12 & 3.34 & 6.25 \\
        5.99 & 6.04 & 6.23 & 5.16 & 3.40 & 6.12 \\
        5.94 & 6.09 & 6.16 & 5.14 & 3.44 & 6.19 \\
        5.96 & 6.05 & 6.14 & 5.12 & 3.44 & 6.12 \\
        \bottomrule 
    \end{tabular}
\end{table}

\begin{table}
    \centering
    \caption{Messwerte zur Pendellänge $l_1 = {78.4}\unit{\centi\metre}$}
    \label{tab:Mess2}
    \begin{tabular}{S S S S S[table-format=1.2] S[table-format=2.2]}
        \toprule
        $\unit{{5}T_{1}\per\second}$ & $\unit{{5}T_{2}\per\second}$ & $\unit{{5}T_+\per\second}$ & $\unit{{5}T_-\per\second}$ & $\unit{{5}T\per\second}$ & $\unit{T_s\per\second}$ \\
        \midrule
        8.89 & 8.81 & 8.93 & 8.09 & 8.22 & 20.20 \\
        8.79 & 8.75 & 8.98 & 8.20 & 8.33 & 20.10 \\
        8.93 & 8.86 & 8.98 & 8.13 & 8.52 & 19.49 \\
        8.97 & 8.82 & 8.87 & 8.12 & 8.36 & 19.61 \\
        8.91 & 8.91 & 8.89 & 8.18 & 8.50 & 20.00 \\
        8.75 & 8.84 & 8.75 & 8.27 & 8.39 & 20.31 \\
        8.85 & 8.90 & 8.89 & 8.18 & 8.51 & 20.25 \\
        8.82 & 9.01 & 8.98 & 8.31 & 8.45 & 19.94 \\
        8.85 & 8.91 & 8.84 & 8.22 & 8.43 & 19.53 \\
        8.96 & 8.99 & 8.89 & 8.23 & 8.56 & 20.24 \\
        \bottomrule 
    \end{tabular}
\end{table}

\subsection{Einzelpendel und gleichsinnige Schwingung}
\label{subsec:Einzelpendelugleichsinnig}
Die Einzelpendel und die gleichsinnige Schwingung können zusammen betrachtet werden, da die Kopplungsfeder bei der gleichsinnigen Schwingung keinen Einfluss auf die Dynamik hat und sie
somit auch als Einzelpendel behandelt werden kann. Die gemessenen Schwingungsdauern $T_1, T_2 \text{und} T_+$ können mit jeweiliger Pendellänge den Tabellen \autoref{Mess1} 
und \autoref{Mess2} entnommen werden. Zu den Längen $L_1$ und $l_2$ lassen sich die theoretisch vorhergesagten Frequenzen der Pendel errechnen. Mit \autoref{eqn:omega} berechnet 
ergeben sich die theoretischen Frequenzen $\omega_1 = (5.87\pm 0.01)\unit{\per\second}$ und $\omega_2 = (3.537\pm 0.002)\unit{\per\second}$. Aus den gemessenen Schwebungsdauern wird 
zunächst der Mittelwert gebildet. Dieser bestimmt sich durch 
\begin{equation}
    \label{eqn:Mittelwert}
    \overline{T} = \frac{1}{N}\sum_{k = 1}^N T_k
\end{equation}
zu $\overline{T} = XXX$. Der Mittelwertfehler lässt sich druch 
\begin{equation}
    \label{eqn:MWFehler}
    \symup{\Delta} T = \frac{\sigma}{\sqrt{N}} \qquad \sigma = \sqrt{\frac{1}{N-1}\sum_{i = 1}^N (\overline{T}-T_i)^2}
\end{equation}
berechnen. Daraus folgen die fehlerbehafteten Schwingungsdauern $\overline{T_1} = XXX$, $\overline{T_2} = YYY$ und $\overline{T_+} =ZZZ$. Diese kann man durch 
\begin{equation}
    \label{eqn:DeltaOmega}
    \symup{\Delta}\omega = \frac{2\pi}{T^2}\symup{\Delta}T
\end{equation}
gemäß der Gaußschen Fehlerfortpflantung in fehlerbehaftete Frequenzen umrechnen. 
\begin{align*}
    \omega_1 &= XX \\
    \omega_2 &= YY \\
    \omega_+ &= ZZ \\
\end{align*}
