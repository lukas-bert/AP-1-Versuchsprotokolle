\section{Auswertung}
\label{sec:Auswertung}
Im Folgenden werden alle zu untersuchenden Werte für beide Pendellängen bestimmt. In der ersten Messung wurde die Pendellänge auf $(28.4\pm 0.1)\unit{\centi\metre}$ eingestellt. 
Für den zweiten Messdurchgang wurde die Länge auf $(78.4\pm 0.1)\unit{\centi\metre}$ geändert. Den Tabellen \ref{tab:Mess1} und \ref{tab:Mess2} können sämtliche Messwerte für die
Schwingungs- und Schwebungsdauern entnommen werden. Dabei ist darauf zu achten, dass bis zu fünf Periodendauern gemessen wurden, was bei der Mittelwertbildung berücksichtigt werden muss.

\begin{table}
    \centering
    \caption{Messwerte zur Pendellänge $l_1 = {28.4}\unit{\centi\metre}$} 
    \label{tab:Mess1}
    \begin{tabular}{S S S S S S[table-format=1.2]}
        \toprule
        ${5}T_l \mathbin{/} \symup{s}$ & ${5}T_r \mathbin{/} \symup{s}$ & ${5}T_{+} \mathbin{/} \symup{s}$ & ${5}T_{-} \mathbin{/} \symup{s}$ & 
        ${3}T \mathbin{/} \symup{s}$ & $T_s \mathbin{/} \symup{s}$ \\
        \midrule
        6.18 & 5.98 & 6.11 & 5.07 & 3.32 & 6.06 \\
        6.11 & 5.79 & 6.15 & 5.07 & 3.45 & 6.20 \\
        6.11 & 5.99 & 6.04 & 5.13 & 3.23 & 6.20 \\
        5.91 & 5.93 & 5.99 & 5.08 & 3.43 & 6.35 \\
        6.10 & 5.99 & 6.19 & 5.11 & 3.39 & 5.87 \\
        5.85 & 6.18 & 6.16 & 5.11 & 3.36 & 6.07 \\
        5.89 & 5.97 & 6.14 & 5.12 & 3.34 & 6.25 \\
        5.99 & 6.04 & 6.23 & 5.16 & 3.40 & 6.12 \\
        5.94 & 6.09 & 6.16 & 5.14 & 3.44 & 6.19 \\
        5.96 & 6.05 & 6.14 & 5.12 & 3.44 & 6.12 \\
        \bottomrule 
    \end{tabular}
\end{table}

\begin{table}
    \centering
    \caption{Messwerte zur Pendellänge $l_1 = {78.4}\unit{\centi\metre}$}
    \label{tab:Mess2}
    \begin{tabular}{S S S S S[table-format=1.2] S[table-format=2.2]}
        \toprule
        ${5}T_l \mathbin{/} \symup{s}$ & ${5}T_r \mathbin{/} \symup{s}$ & ${5}T_{+} \mathbin{/} \symup{s}$ & ${5}T_{-} \mathbin{/} \symup{s}$ & 
        ${5}T \mathbin{/} \symup{s}$ & $T_s \mathbin{/} \symup{s}$ \\
        \midrule
        8.89 & 8.81 & 8.93 & 8.09 & 8.22 & 20.20 \\
        8.79 & 8.75 & 8.98 & 8.20 & 8.33 & 20.10 \\
        8.93 & 8.86 & 8.98 & 8.13 & 8.52 & 19.49 \\
        8.97 & 8.82 & 8.87 & 8.12 & 8.36 & 19.61 \\
        8.91 & 8.91 & 8.89 & 8.18 & 8.50 & 20.00 \\
        8.75 & 8.84 & 8.75 & 8.27 & 8.39 & 20.31 \\
        8.85 & 8.90 & 8.89 & 8.18 & 8.51 & 20.25 \\
        8.82 & 9.01 & 8.98 & 8.31 & 8.45 & 19.94 \\
        8.85 & 8.91 & 8.84 & 8.22 & 8.43 & 19.53 \\
        8.96 & 8.99 & 8.89 & 8.23 & 8.56 & 20.24 \\
        \bottomrule 
    \end{tabular}
\end{table}

Alle Mittelwerte für die Schwingungsdauern $T$ berechnen sich nach Division durch die Periodenanzahl zu 
\begin{equation*}
    \label{eqn:Mittelwert}
    \overline{T} = \frac{1}{N}\sum_{k = 1}^N T_k \qquad ,
\end{equation*}
wobei $N$ die Anzahl der Messungen beschreibt und in diesem Experiment immer gleich $10$ ist. Der Mittelwertfehler $\symup{\Delta}T$ lässt sich durch 
\begin{equation}
    \label{eqn:MWFehler}
    \symup{\Delta} \overline{T} = \frac{\sigma}{\sqrt{N}}, \qquad \sigma = \sqrt{\frac{1}{N-1}\sum_{i = 1}^N (\overline{T}-T_i)^2}
\end{equation}
berechnen.

An einigen Stellen werden Schwingungsdauern in Kreisfrequenzen umgerechnet. Dies geschieht nach \autoref{eqn:omega} und führt zu einem Fehler
\begin{equation}
    \label{eqn:DeltaOmega}
    \symup{\Delta}\omega = \frac{2\pi}{T^2}\symup{\Delta}T
\end{equation}
der Kreisfrequenz, der sich aus der Gaußschen Fehlerfortpflantung ergibt.

\subsection{Einzelpendel und gleichsinnige Schwingung}
\label{subsec:Gleichsinnig}
Die Schwingungsdauern der Einzelpendel und der gleichsinnigen Schwingung können zusammen betrachtet werden, da die Kopplungsfeder bei der gleichsinnigen Schwingung 
keinen Einfluss auf die Dynamik hat. Gemessenen wurden die Schwingungsdauern $T_l, T_r$ und $T_+$ aus \autoref{tab:Mess1} und \autoref{tab:Mess2}.
Der Index l beschreibt jeweils das linke Pendel, r bezeichnet das rechte Pendel. Zu den Längen $l_1$ und $l_2$ lassen sich die 
theoretisch vorhergesagten Frequenzen der Pendel errechnen. Mit \autoref{eqn:omega} ergeben sich die theoretischen Frequenzen $\omega_{+1\text{, Theorie}} = (5.87\pm 0.01)\unit{\hertz}$ und 
$\omega_{+2\text{, Theorie}} = (3.537\pm 0.002)\unit{\hertz}$. Aus den gemessenen Werten wird zunächst der Mittelwert gebildet. Die Fehler des Mittelwerts ergeben sich nach \autoref{eqn:MWFehler}.
Daraus folgen die fehlerbehafteten Schwingungsdauern $\overline{T_l}$, $\overline{T_r}$ und $\overline{T_+}$ für beide Pendellängen. Diese kann man durch \autoref{eqn:omega} in 
die zugehörigen Frequenzen überführen.
\begin{table}
    \centering
    \caption{Mittelwerte der Messungen und daraus resultierende Frequenzen}
    \sisetup{table-format=1.3}
    \begin{tabular}{c S @{${}\pm{}$} S S @{${}\pm{}$} S}
    \toprule
    & \multicolumn{2}{c}{$l = 28.4\unit{\centi\metre}$} & \multicolumn{2}{c}{$l = 78.4\unit{\centi\metre}$} \\
    \cmidrule(lr){2-3}\cmidrule(lr){4-5}
    %\midrule
    {$\overline{T}_l \mathbin{/} \symup{s}$}                & 1.201 & 0.007 & 1.774 & 0.005 \\
    {$\overline{T}_r \mathbin{/} \symup{s}$}                & 1.200 & 0.007 & 1.776 & 0.005 \\
    {$\overline{T}_+ \mathbin{/} \symup{s}$}                & 1.226 & 0.004 & 1.780 & 0.005 \\
    {$\omega_+ \mathbin{/} \symup{Hz}$}                     & 5.124 & 0.019 & 3.530 & 0.009 \\
    {$\omega_{+,\text{Theorie}} \mathbin{/} \symup{Hz}$}    & 5.876 & 0.01  & 3.537 & 0.002 \\
    \bottomrule
    \end{tabular}
\end{table}    

\subsection{Gegensinnige Schwingung}
\label{subsec:Gegensinnig}
Die Theoriwerte der Frequenzen $\omega_-$ der gegensinnigen Schwingung lassen sich mittels \autoref{eqn:omega_Gegensinnig} berechnen. Dabei wurde für die Kopplungskonstante $K$ 
der experimentell ermittelte Wert eingestzt. Es folgt mit der gaußschen Fehlerfortpflanzung
\begin{equation*}
    \label{eqn:DeltaOmegaMinus}
    \symup{\Delta}\omega_- = \frac{1}{\sqrt{l} \cdot (g + 2K)} \cdot \sqrt{\left(\frac{1}{2}\symup{\Delta}g\right)^2 + \symup{\Delta}K^2 + \left(\frac{g+2K}{2l}\symup{\Delta}l\right)^2}
\end{equation*}
für die Unsicherheit dieser Größe. $\symup{\Delta}g$ wurde in der Rechnung als $0$ angenommen. Wie oben beschrieben lassen sich Mittelwerte aus den Messwerten bilden,
woraus sich wiederum die Frequenzen berechnen lassen. 

\begin{table}
    \centering
    \caption{Mittelwerte der Messungen für die gegensinnige Schwingung und daraus resultierende Frequenzen}
    \sisetup{table-format=1.3}
    \begin{tabular}{c S @{${}\pm{}$} S S @{${}\pm{}$} S}
    \toprule
    & \multicolumn{2}{c}{$l = 28.4\unit{\centi\metre}$} & \multicolumn{2}{c}{$l = 78.4\unit{\centi\metre}$} \\
    \cmidrule(lr){2-3}\cmidrule(lr){4-5}
    %\midrule
    {$\overline{T}_- \mathbin{/} \symup{s}$}                & 1.022 & 0.002 & 1.639 & 0.004 \\
    {$\omega_- \mathbin{/} \symup{Hz}$}                     & 6.147 & 0.011 & 3.835 & 0.010 \\
    {$\omega_{{-}\text{, Theorie}} \mathbin{/} \symup{Hz}$} & 5.983 & 0.003 & 3.566 & 0.001 \\
    \bottomrule
    \end{tabular}
\end{table}
Die Federkonstante $K$ lässt sich nun mit den experimentell ermittelten Werten für $\omega_+$ und $\omega_-$ nach \autoref{eqn:Federkonstante} berechnen.
Die Unsicherheit des Wertes ergibt sich zu
\begin{equation*}
    \label{eqn:DeltaK}
    \symup{\Delta}K = \frac{4T_+T_-}{(T_+^2 + T_-^2)^2} \cdot \sqrt{T_-^2 \, \symup{\Delta}T_+^2 + T_+^2 \, \symup{\Delta}T_-^2}.
\end{equation*}
Damit folgen die beiden Werte für die Federkonstanten $K_1 = 0.180 \pm 0.004$, $K_2 = 0.083 \pm 0.004$, die bereits zur Berechnung der Theoriewerte für $\omega_-$ verwendet wurden.

\subsection{Gekoppelte Schwingung / Schwebung}
\label{subsec:Schwebung}
Die Theoriewerte der Schwebungsfrequenzen ergeben sich mit \autoref{eqn:Schwebung}. Diese Werte haben einen Fehler von
\begin{equation*}
    \label{eqn:DeltaOmega_s}
    \symup{\Delta}\omega_{\text{S}} = \sqrt{\symup{\Delta}\omega_+^2 + \symup{\Delta}\omega_-^2}.
\end{equation*}
Erneut lassen sich die experimentellen Werte durch Mittelung und Umrechnen der Schwingungsdauern in Frequenzen feststellen.

\begin{table}
    \centering
    \caption{Mittelwerte der Messungen für die Schwebung und daraus resultierende Frequenzen}
    \sisetup{table-format=1.3}
    \begin{tabular}{c S @{${}\pm{}$} S S[table-format=2.3] @{${}\pm{}$} S[table-format=1.3]}
    \toprule
    & \multicolumn{2}{c}{$l = 28.4\unit{\centi\metre}$} & \multicolumn{2}{c}{$l = 78.4\unit{\centi\metre}$} \\
    \cmidrule(lr){2-3}\cmidrule(lr){4-5}
    %\midrule
    {$\overline{T}_s \mathbin{/} \symup{s}$}                & 6.14  & 0.04  & 19.97 & 0.10 \\
    {$\omega_s \mathbin{/} \symup{Hz}$}                     & 1.023 & 0.007 & 0.315 & 0.002 \\
    {$\omega_{{s}\text{, Theorie}} \mathbin{/} \symup{Hz}$} & 0.107 & 0.011 & 0.030 & 0.002 \\
    \bottomrule
    \end{tabular}
\end{table}