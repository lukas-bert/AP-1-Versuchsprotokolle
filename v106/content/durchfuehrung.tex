\section{Durchführung}
\label{sec:Durchführung}
In diesem Versuch werden alle untersuchten Schwing- und Schwebedauern für zwei unterschiedliche Pendellängen bestimmt. Dies kann durch unterschiedliche Aufhängungen der Masse am 
Pendel realiusiert werden, da die angehängte Masse sehr viel Größer ist als die überstehende Masse des Stabes(siehe \autoref{fig:Aufbau1}).
Die Schwing- und Schwebedauern werden für verschiedene Arten von Schwingungen gemessen. Vorbereitent dazu wird zunächst überlegt, ob es sich bei diesem Versuch um eine 
harmonische Schwingung handelt. Eine harmonische Schwingung muss durch  Sinus- und Kosinusfunktion beschrieben werden können und es muss ein lineares Kraftgesetzt gelten.
Außerdem muss die Energie erhalten sein. Die ersten beiden Bedingungen gelten für ein Pendel, wie es imer Versuch genutzt wird. Aufgrund der kurzen Messdauern kann annähernde 
Energieerhaltung angenommen werden. Um nun die in \autoref{sec:Theorie} nutzen zu können dürfen bei der Durchführung lediglich kleine Auslenkungen der Pendel verwendet werden, 
also solche bei denen die Kleinwinkelnäherung noch möglichst genau gilt. Dies ist bis zu einem Auslenkwinkel von circa $5\textdegree$ gegeben. 


Für diesen Versuch benötigt man zwei identische Pendel(siehe \autoref{fig:Aufbau1}), welche durch eine Feder gekoppelt werden können(siehe \autoref{fig:Aufbau2}).
\begin{figure}
    \centering
    \includegraphics[width=0.4\textwidth]{content/Einzelpendel.jpg}
	\caption{Aufnahme des Versuchaufbaus}
	\label{fig:Aufbau1}
\end{figure}
Zuerst wird die Schwingungsdauer eines einfachen Pendels bestimmt. Dazu werden fünf Schwingungsdauern gemessen um den Fehler der Zeitmessung möglichst klein zu halten.
Dies wird zehn mal an beiden Pendeln durchgeführt, um auch die möglichen Unterschiede der beiden Einzelpendel zueinander gemessen zu haben. 


Danach wird die Kopplungsfeder an die Penel wie in \autoref{fig:Aufbau2} angebracht. 
\begin{figure}
    \centering
    \includegraphics[width=0.4\textwidth]{content/Gekoppelt.jpg}
	\caption{Aufnahme des Versuchaufbaus}
	\label{fig:Aufbau2}
\end{figure}
Zuerst wird eine gleichsinnige Schwingung der gekoppelten Pendel untersucht. Dazu werden die Pendel, wie in \autoref{fig:gleichsinnig} zu sehen ist, um den gleichen Winkel ausgelenkt.
Es werden pro Messung fünf Schwingungsdauern gemessen. Die Schwingungsdauer wird von einer maximalen Auslenkung bis zur selben maximalen Auslenkung gemessen. Erneut werden zehn
Messungen durchgeführt. 


Danach wird die gegensinnige Schwingung untersucht. Die Pendel werden gemäß \autoref{fig:gegensinnig} um den negativen Winkel zueinander ausgelenkt. Eine Schwingungsdauer wird von
der äußeren Auslenkung bis zur nächsten äußeren Auslenkung gemessen. Auch hier werden fünf Schwingungsdauern zehn mal gemessen. 


Zuletzt untersucht man die gekoppelte Schwingung beziehungsweise den Schwebungsfall. Eines der Pendel soll sich in Ruhelage befinden während das Andere ausgelenkt wird. 
\autoref{fig:schwebungsfall} stellt die beschrieben Auslenkung dar. Hier wird die Schwebungssdauer von einer Ruhelage eines Pendels bis zu seiner nächsten bestimmt. Dabei muss sehr
genau beobachtet werden wann die Ruhelage eintrifft, da kurz vor der Ruhelage noch sehr kleine Schwingungen ausgeübt werden. Zu dieser Messung reicht eine Schwebungsdauer, da diese
ausreichend lang ist. Zusätzlich soll die Schwingungsdauern gemessen werden. Diese ist beim Schwebungsfall bei einem der Pendel zwisches dessen Ruhelagen zu messen. Je nach länge 
des Pendel können hier nur 3-5 Schwingungsdauer gemessen werden, da sich das Pendel dann schon wieder in Ruhelage begibt. Es sollen ebenfalls 10 Messungen der Schwing- und Schwebedauern
durchgeführt werden. 