\section{Auswertung}
\label{sec:Auswertung}
Für die Auswertung der Messungen sind einige Materialkonstanten relevant. Die Dopplerphantomflüssigkeit hat eine Dichte $\rho = \qty{1.15}{\gram\per\cubic\centi\metre}$,
eine Viskosität von $\eta = \qty{12}{\milli\pascal\second}$ und
die Schallgeschwindigkeit in dieser beträgt $c_\text{L} = \qty{1800}{\metre\per\second}$. Die Geschwindigkeit des Schalls im Dopplerprisma
beträgt $c_\text{P} = \qty{2700}{\metre\per\second}$. Die Länge der Vorlaufsstrecke im Prisma ist $l = \qty{30.7}{\milli\metre}$. Die Senderfrequenz beträgt 
$\nu_0 = \qty{2}{\mega\hertz}$. Die Rohre, durch welche die Dopplerphantomflüssigkeit fließt, haben Innendurchmesser von $\qty{7}{\milli\metre}$, $\qty{10}{\milli\metre}$
und $\qty{16}{\milli\metre}$. Die Außendurchmesser betragen $\qty{10}{\milli\metre}$, $\qty{15}{\milli\metre}$ und $\qty{20}{\milli\metre}$.

\subsection{Bestimmung der Strömungsgeschwindigkeit}
Die Strömungsgeschwindigkeit der Dopplerphantomflüssigkeit kann aus \autoref{eqn:frequenzverschiebung} berechnet werden. Dazu wird diese Formel auf $v$ umgestellt.
\begin{equation*}
  v_\text{Strömung} = \frac{\symup{\Delta}\nu\cdot c}{\nu_0\,\symup{cos}\,\alpha} 
\end{equation*}
Die Frequenzverschiebung $\symup{\Delta}\nu$ wird aus den gemessenen Werten $f_\text{max}$ und $f_\text{mean}$ gemäß $\symup{\Delta}\nu = f_\text{max} - f_\text{mean}$ berechnet.
Diese können dem Anhang entnommen werden.
In den folgenden Abbildungen \ref{fig:plot1_1}, \ref{fig:plot1_2} und \ref{fig:plot1_3} wird die Frequenzverschiebung dividiert durch $\symup{cos}\,\alpha$ in 
Abhängigkeit zur berechneten Strömungsgeschwindigkeit für die verschiedenen Rohrduchmesser dargestellt.
\begin{figure}
  \centering
  \includegraphics{plot1_1.pdf}
  \caption{Messwerte zu verschiedenen Einstrahlwinkeln gegen die berechnete Strömungsgeschwindigkeit zum (Innen-)Rohrduchmesser $d = \qty{7}{\milli\metre}$.}
  \label{fig:plot1_1}
\end{figure}

\begin{figure}
  \centering
  \includegraphics{plot1_2.pdf}
  \caption{Messwerte zu verschiedenen Einstrahlwinkeln gegen die berechnete Strömungsgeschwindigkeit zum (Innen-)Rohrduchmesser $d = \qty{10}{\milli\metre}$.}
  \label{fig:plot1_2}
\end{figure}

\begin{figure}
  \centering
  \includegraphics{plot1_3.pdf}
  \caption{Messwerte zu verschiedenen Einstrahlwinkeln gegen die berechnete Strömungsgeschwindigkeit zum (Innen-)Rohrduchmesser $d = \qty{16}{\milli\metre}$.}
  \label{fig:plot1_3}
\end{figure}

\subsection{Bestimmung des Strömungsprofils}
\label{subsec:strömungsprofil}
Wie in \autoref{subsec:Messaufgaben} beschrieben werden Messwerte zur Frequenzverschiebung, zur Streuintensität und zur momentanen Strömungsgeschwindigkeit zu verschiedenen 
Messtiefen genommen. Die originalen Messdaten können dem Anhang entnommen werden.
In den Abbildungen \ref{fig:plot2_1} und \ref{fig:plot2_2} wird für zwei unterschiedliche Pumpleistungen die Streuintensität $I$ und die Momentangeschwindigkeit $v$ gegen die 
Messtiefe dargestellt. Des Weiteren wurde ein Ausgleichspolynom zweiten Grades eingefügt.

\begin{figure}
  \centering
  \includegraphics[width=.8\textwidth]{plot2_1.pdf}
  \caption{Intensität und Momentangeschwindigkeit der Flüssigkeit in Abhängigkeit zur Messtiefe mit einer Pumpleistung von $\qty{45}{\percent}$ der maximalen Leistung.}
  \label{fig:plot2_1}
\end{figure}

\begin{figure}
  \centering
  \includegraphics[width=.8\textwidth]{plot2_2.pdf}
  \caption{Intensität und Momentangeschwindigkeit der Flüssigkeit in Abhängigkeit zur Messtiefe mit einer Pumpleistung von $\qty{70}{\percent}$ der maximalen Leistung.}
  \label{fig:plot2_2}
\end{figure}
