\section{Auswertung}
\label{sec:Auswertung}
Für die Auswertung der Messungen wird im folgenden die Dopplerphantomflüssigkeit mit einer Dichte $\rho = 1.15 \unit{\gram\per\cubic\centi\metre}$, einer 
Schallgeschwindigkeit $c_\text{L} = 1800 \unit{\metre\per\second}$ und einer Viskosität $\eta = 12 \unit{\milli\pascal\second}$ angenommen. Außerdem hat das Dopplerprisma
eine Schallgeschwindigkeit $c_\text{P} = 2700 \unit{\metre\per\second}$ und eine Länge der Vorlaufsstrecke von $l = 30.7 \unit{\milli\metre}$. Die Senderfrequenz beträgt 
$\nu_0 = 2\unit{\mega\hertz}$. Die Rohre, durch welche die Dopplerphantomflüssigkeit fließt, haben Innendurchmesser von $7 \unit{\milli\metre}$, $10 \unit{\milli\metre}$
und $16 \unit{\milli\metre}$. Die Außendurchmesser betragen $10 \unit{\milli\metre}$, $15 \unit{\milli\metre}$ und $20 \unit{\milli\metre}$.
\subsection{Bestimmung der Strömungsgeschwindigkeit}
Die Strömungsgeschwindigkeit der Dopplerphantomflüssigkeit kann aus der Formel \eqref{eqn:frequenzverschiebung} berechnet werden. Dazu wird diese Formel auf $v$ umgestellt.
\begin{equation*}
  v_\text{Strömung} = \frac{\symup{\Delta}\nu\cdot c}{\nu_0\,\symup{cos}\,\alpha} 
\end{equation*}
Die frequenzverschiebung $\symup{\Delta}\nu$ wurde aus den gemessenen Werten $f_\text{max}$ und $f_\text{mean}$ gemäß $\symup{\Delta}\nu = f_\text{max} - f_\text{mean}$ berechnet.
Diese können dem Anhang entnommen werden.
In den folgenden Abbildungen \ref{fig:plot1_1}, \ref{fig:plot1_2} und \ref{fig:plot1_3} werden die Frequenzverschiebung dividiert durch den $\symup{cos}\,\alpha$ in 
Abhängigkeit von der berechneten Strömungsgeschwindigkeit dargestellt. Dies wird für die drei oberhalb erwähnten Rohrdurchmesser gemacht.
\begin{figure}
  \centering
  \includegraphics{plot1_1.pdf}
  \caption{In dieser Abbildung wird die eben erwähnte Grafik für einen Innenrohrdurchmesser  von $d = 7 \unit{\milli\metre}$ dargestellt.}
  \label{fig:plot1_1}
\end{figure}

\begin{figure}
  \centering
  \includegraphics{plot1_2.pdf}
  \caption{In dieser Abbildung wird die eben erwähnte Grafik für einen Innenrohrdurchmesser  von $d = 10 \unit{\milli\metre}$ dargestellt.}
  \label{fig:plot1_2}
\end{figure}

\begin{figure}
  \centering
  \includegraphics{plot1_3.pdf}
  \caption{In dieser Abbildung wird die eben erwähnte Grafik für einen Innenrohrdurchmesser  von $d = 16 \unit{\milli\metre}$ dargestellt.}
  \label{fig:plot1_3}
\end{figure}
\subsection{Bestimmung des Strömungsprofils}
\label{subsec:strömungsprofil}
Das Strömungsprofil mit einer gleichen Messung aufgenommen werden. Die orginalen Messwerte können dem Anhang einnommen werden. Allerdings ist nun der  Winkel am Prisma
$\theta = 15\unit{\degree}$ konstant und die Messtiefe wird variiert.
In den Abbildungen \ref{fig:plot2_1} und \ref{fig:plot2_2} wird für zwei unterschiedliche Pumpleistungen die Streuintensität $I$ und die Momentangeschwindigkeit $v$ gegen die 
Messtiefe dargestellt. Außerdem wurde mit den Messwerten ein quadratischer Fit erstellt. 

\begin{figure}
  \centering
  \includegraphics[width=.8\textwidth]{plot2_1.pdf}
  \caption{In dieser Abbildung werden die Intensität und die Momentangeschwindigkeit in Abhängigkeit von der Messtiefe zu einer Pumpleistung von 45\% der maximalen Pumpleistung dargestellt.}
  \label{fig:plot2_1}
\end{figure}

\begin{figure}
  \centering
  \includegraphics[width=.8\textwidth]{plot2_2.pdf}
  \caption{In dieser Abbildung werden die Intensität und die Momentangeschwindigkeit in Abhängigkeit von der Messtiefe zu einer Pumpleistung von 70\% der maximalen Pumpleistung dargestellt.}
  \label{fig:plot2_2}
\end{figure}