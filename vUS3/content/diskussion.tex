\section{Diskussion}
\label{sec:Diskussion}
Im ersten Teil des Versuches wird der Zusammenhang zwischen Dopplerwinkel $\alpha$ und Strömungsgeschwindigkeit $v_\text{Fluss}$ untersucht. 
Anhand der \autoref{eqn:frequenzverschiebung} lässt sich erkennen, dass der Quotient $\symup{\Delta}\nu \mathbin{/} \symup{cos}\, \alpha$
nicht mehr von $\alpha$ abhängig ist und sich proportional zu $v_\text{Fluss}$ verhält. Demnach sollten die in den Abbildungen \ref{fig:plot1_1}
bis \ref{fig:plot1_3} dargestellten Messwerte eine Gerade ergeben, auf welcher die Messpunkte zu verschiedenen Winkeln $\theta$ gruppiert sind.
Es fällt jedoch auf, dass besonders bei höheren Fließgeschwindigkeiten eine große Streuung der Messwerte auftritt, obwohl diese die gleichen
Punkte abbilden sollten.
Dies deutet auf die Ungenauigkeit des Verfahrens hin. Eine Ursache dafür ist, dass die Messwertanzeige der mittleren Frequenz $f_\text{mean}$ und
jener der maximalen Frequenz $f_\text{max}$ starken Schwankungen unterliegt, wodurch die daraus bestimmte Frequenzverschiebung eine große Ungenauigkeit
vorweist. Weitere Fehlerquellen sind die manuelle Führung der Messsonde und die Drehzahlschwankungen der Pumpe. 

Bei der Bestimmung des Strömungsprofils ergeben sich die Grafiken \ref{fig:plot2_1} und \ref{fig:plot2_2}. Es lässt sich erkennen, dass der Verlauf der
Messpunkte der Momentangeschwindigkeit parabelähnlich ist, weshalb die Vermutung nahe liegt, dass ein quadratischer Zusammenhang zwischen Messtiefe und
Geschwindigkeit der Flüssigkeit besteht. Auch die Messwerte der Intensität stellen einen ähnlichen Verlauf dar, jedoch könnte hier eine Hutfunktion
eine bessere Approximation liefern. Bei höherer Pumpleistung (Strömungsgeschwindigkeit) können mehr Messwerte genommen werden, da bei geringeren 
Geschwindigkeiten die Streuintensität des Signals an den Rändern der Wasserleitung nicht mehr ausreicht. Wie auch im ersten Teil ist die starke 
Fluktuation der in \textit{Flow View} angezeigten Messwerte eine potenzielle Fehlerquelle. 
