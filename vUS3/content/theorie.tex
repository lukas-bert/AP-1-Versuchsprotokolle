\section{Ziel}
\label{sec:Ziel}
Im folgendem Experiment wird mittels Doppler-Sonographie die Strömung einer Testflüssigkeit untersucht. Die Doppler-Sonographie ist eine Ultraschallmethode, welche es ermöglicht
Strömungsgeschwindigkeiten einer Flüssigkeit zu ermitteln, obgleich diese von einem anderem Medium umschlossen ist.   
\section{Theorie}
\label{sec:Theorie}
Schallwellen sind Druckschwankungen in einem Medium, welche sich gemäß der Wellengleichung
\begin{equation*}
    \label{eqn:Schallwelle}
    p(x,t) = p_0 + v_0 Z \cos \left(\omega t - kx\right)
\end{equation*}
ausbreiten. $Z = c \cdot \rho$ beschreibt dabei die materialabhängige akustische Impedanz, $p_0$ den Normaldruck des Mediums.
Der Frequenzbereich des Schalls von $\qty{20}{\mega\hertz}$ bis $\qty{1}{\giga\hertz}$ befindet sich über dem Hörbaren und wird als \textit{Ultraschall} bezeichnet.
Schall dieses Frequenzbereiches kann zur zerstörungsfreien Werkstoffprüfung genutzt werden und findet so unter anderem Anwendung in der Medizin. \\

\subsection{Erzeugung von Ultraschall}
\label{subsec:Erzeugung}
Um Ultraschall zu erzeugen, kann der \textit{piezo-elektrische Effekt} genutzt werden. Ein piezo-elektrischer Kristall wird in ein elektrisches Wechselfeld gesetzt,
welches bei paralleler Anordnung einer polaren Achse des Kristalls zu den elektrischen Feldlinien diesen in Schwingung versetzt, wodurch Ultraschallwellen erzeugt werden. 
Passt die Anregungsfrequenz zur Eigenfrequenz des Kristalls (Resonanzfall), können große Schwingungsamplituden (Schallenergiedichten) erzeugt werden. 
Umgekehrt kann ein solcher Kristall als Empfänger genutzt werden, da er bei Anregung durch Schallwellen ebenfalls in Schwingung versetzt wird. \\

\subsection{Physikalische Grundlagen der Scanverfahren}
\label{subsec:Grundlagen}
Bei Scanverfahren mit Ultraschall wird das Wellenverhalten des Schalls ausgenutzt. Anhand der Medium-abhängigen Schallgeschwindigkeit $c$ können mithilfe der
Laufzeit $t$ eines Impulssignals und dem Weg-Zeit-Gesetz
\begin{equation}
    \label{eqn:WegZeit}
    s = v \cdot t
\end{equation} 
Abstände bestimmt werden.
In Flüssigkeiten kann die Schallgeschwindigkeit mit der Kompressibilität $\kappa$ über $c_\text{Fl} = \sqrt{\sfrac{1}{\kappa \rho}}$ bestimmt werden. In Feststoffen
lautet die Gleichung $c_\text{Fe} = \sqrt{\sfrac{E}{\rho}}$, mit dem Elastizitätsmodul $E$. Über die Schallgeschwindigkeit $c$ und die Frequenz $f$ der Schallwelle 
kann die Wellenlänge
\begin{equation}
    \label{eqn:lambda}
    \lambda = \frac{c}{f}
\end{equation}
der Schallwelle berechnet werden.
Die Intensität $I$ des Signals nimmt exponentiell mit der Strecke $x$ ab
\begin{equation*}
    I(x) = I_0 \cdot e^{\alpha x},
\end{equation*}
wobei $\alpha$ ein materialabhängiger Absorptionskoeffizient ist. Da die Absorption in Luft sehr stark ist, wird in der Regel ein Kontaktmittel (z.B. Wasser) zwischen
Sender und Material verwendet.
An Grenzflächen verschiedener Stoffe wird ein Teil der Schallwelle reflektiert. Der reflektierte Anteil $R$ kann mit den akustischen Impedanzen $Z = \rho \cdot c$ der 
angrenzenden Materialien über
\begin{equation*}
    \label{eqn:Reflektion}
    R = \left(\frac{Z_1 - Z_2}{Z_1 + Z_2}\right)^2
\end{equation*}
bestimmt werden. $T = 1 - R$ ist der transmittierte Anteil.

\subsection{Scanverfahren}
\label{subsec:Scanverfahren}
Grundlegend können zwei verschiedene Verfahren angewendet werden. 

Bei dem \textit{Durchschallungsverfahren} wird ein Schallimpuls an einer Seite des zu analysierenden Stückes ausgesendet und an der Anderen von einem Empfänger 
aufgenommen. Das Vorhandensein von Störstellen im Material lässt sich über abweichende Intensitäten des Empfangsimpulses feststellen. Über die Größe dieser Störstelle
kann keine Aussage getroffen werden.

Bei dem \textit{Impuls-Echo-Verfahren} wird die Reflektion der Schallwelle an Grenzflächen ausgenutzt, indem der Sender gleichzeitig als Empfänger dient und den 
reflektierten Teil des Signals an Grenzflächen von Stoffen detektiert. Bei bekannter Schallgeschwindigkeit lässt sich so eine Aussage über die Tiefe der Störstelle 
mit dem Weg-Zeit-Gesetz \eqref{eqn:WegZeit} und der Signallaufzeit treffen.

Aufgenommene Messdaten der Signallaufzeit können in einem \textit{A-Scan}, \textit{B-Scan} oder \textit{TM-Scan} dargestellt werden.
\begin{itemize}
    \item{Beim \textbf{A}mplituden-Scan werden lediglich die empfangenen Echoamplituden als Funktion der Laufzeit (oder als Funktion der Tiefe, unter Angabe von $c$)
    in einem Diagramm dargestellt.}
    \item{Beim \textbf{B}rightness-Scan kann der Sender entlang einer Linie bewegt werden, wobei ein zweidimensionales Bild des Querschnitts des untersuchten Materials
    erstellt wird. Die gemessenen Amplituden werden zu jeder Tiefe (Laufzeit) in Helligkeitsstufen (oder wahlweise Farbstufen) dargestellt.}
    \item{Beim \textbf{T}ime-\textbf{M}otion-Scan wird durch Aussenden mehrerer Signale eine Bildfolge aufgenommen. Dies macht es möglich Bewegungen, wie beispielsweise 
    die eines Organs, in dem untersuchten Objekt/Körper sichtbar zu machen.}
\end{itemize}
\subsection{Doppler-Effekt}
\label{subsec:doppler}
Der Doppler-Effekt beschreibt das Phänomen der Änderung der Wellenlänge, wenn sich Sender und Empfänger relativ zueinander bewegen. Dabei können drei verschiedene Fälle auftreten.
Im ersten Fall wird angenommen, dass sich der Sender bewegt und sich der Empfänger in Ruhe befindet. Wenn sich der Sender auf den Empfänger zu bewegt, steigt die Frequenz $\nu_{\text{kl}}$. Bewegt 
sich der Sender jedoch weg von dem Empfänger sinkt die vom Empfänger aufgenommene Frequenz $\nu_{\text{gr}}$.
\begin{equation*}
    \nu_{\text{kl/gr}} = \frac{\nu_0}{1 \mp \frac{v}{c}} 
\end{equation*}
Im zweiten Fall bewegt sich der Empfänger und der Sender soll sich in Ruhe befinden. Bewegt sich der Empfänger auf den Sender zu steigt die Frequenz $\nu_{\text{h}}$.
Wenn er sich entfernt, dann sinkt die aufgenommene Frequenz $\nu_{\text{n}}$. 
\begin{equation*}
    \nu_{\text{h/n}} = \nu_0 \left(1 \pm \frac{v}{c}\right)
\end{equation*}
Im dritten Fall ändert sich die Frequenz in Abhängigkeit beider Bewegungen.
\subsection{Bestimmung der Strömungsgeschwindigkeit}
\label{subsec:strömung}
Wird der Doppler-Effekt auf eine Flüssigkeit angewendet, welche mit Ultraschall untersucht wird, so trifft die Schallwelle unter einem Winkel $\alpha$ auf die Flüssigkeit
und wird in einem Winkel $\beta$ reflektiert. Aufgrund des Doppler-Effekts ändert sich dabei die Frequenz. Die Frequenzänderung kann mittels
\begin{equation*}
    \symup{\Delta}\nu = \nu_0\frac{v}{c}\left(\symup{cos}\,\alpha + \symup{cos}\,\beta\right)
\end{equation*}
berechnet werden. $\alpha$ und $\beta$ beschreiben den Winkel zwischen der Wellengeschwindigkeit und der Wellennormalen. Wird das Ultraschalloskop 
im Impuls-Echo-Verfahren verwendet gilt $\alpha = \beta$, sodass die Frequenzänderung durch 
\begin{equation}
    \label{eqn:frequenzverschiebung}
    \symup{\Delta}\nu = 2\nu_0\frac{v}{c}\symup{cos}\,\alpha
\end{equation}
berechnet werden kann. Die größe des Winkel $\alpha$ folgt aus dem Brechungsgesetz. Mit den Schallgeschwindigkeiten der beiden Grenzmedien $c_\text{L}$ und $c_\text{P}$
ergibt sich  
\begin{equation}
    \label{eqn:alpha}
    \alpha = \frac{\pi}{2} - \symup{arcsin}\left(\symup{sin}\left(\theta\right)\,\frac{c_\text{L}}{c_\text{P}}\right)
\end{equation}
für einen Einstrahlwinkel $\theta$.
