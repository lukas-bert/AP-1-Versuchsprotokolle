\section{Auswertung}
\label{sec:Auswertung}
Zur Auswertung des Versuches werden Mittelwerte gebildet. Diese führen zu einem Standardfehler des Mittelwertes, welcher gemäß
\begin{equation*}
  \label{eqn:MW-Fehler}
  \sigma(x) = \sqrt{\frac{1}{n(n-1)} \sum_i (x_i - \overline{x})^2}.
\end{equation*}
berechnet werden kann. Die Fehlerrechnung der Messwerte genügt der gaußschen Fehlerfortpflanzung und wird in \textit{Python} mittels \textit{Scipy} \cite{scipy} durchgeführt.

\subsection{Bestimmung der Brechungsindizes über die Fresnelschen Formeln}
\label{subsec:Index_Fresnel}
Die Messwerte der gemessenen Ströme des Photoelements sind in \autoref{tab:Messwerte} aufgeführt. 

\begin{table}
  \centering
  \caption{Messwerte der Ströme des Photoelements und daraus resultierende Brechungsindizes. Der Strom ohne Reflexion beträgt $I_0 = \qty{180}{\micro\ampere}$, der
  Dunkelstrom $I_\text{D} = \qty{62}{\nano\ampere}$. Die mit '*' markierten Werte werden nicht in nachfolgende Rechnungen einbezogen.}
  \label{tab:Messwerte}
  \begin{tabular}{S[table-format = 2.0] | c S[table-format = 3.2] | c S[table-format = 1.2]}
      {} & \multicolumn{2}{c}{p-polarisiert} & \multicolumn{2}{c}{s-polarisiert} \\
      \toprule
        {$\alpha \mathbin{/} \unit{\degree}$} & {$I_\text{r} \mathbin{/} \unit{\micro\ampere}$} & {$n_\text{p}(\alpha)$} &%
        {$I_\text{r} \mathbin{/} \unit{\micro\ampere}$} & {$n_\text{p}(\alpha)$} \\
        \midrule
         6 & 66  &   4.09  & 64  & 3.93  \\  
         8 & 66  &   4.11  & 66  & 4.03  \\
        10 & 64  &   4.01  & 62  & 3.79  \\
        12 & 64  &   4.04  & 64  & 3.87  \\
        14 & 66  &   4.19  & 66  & 3.96  \\
        16 & 66  &   4.23  & 68  & 4.04  \\
        18 & 64  &   4.15  & 68  & 4.00  \\
        20 & 62  &   4.07  & 68  & 3.95  \\
        22 & 56  &   3.78  & 62  & 3.58  \\
        24 & 60  &   4.06  & 70  & 3.96  \\
        26 & 56  &   3.89  & 68  & 3.79  \\
        28 & 60  &   4.20  & 76  & 4.19  \\
        30 & 52  &   3.81  & 68  & 3.66  \\
        32 & 52  &   3.88  & 72  & 3.80  \\
        34 & 54  &   4.09  & 80  & 4.18  \\
        36 & 52  &   4.07  & 80  & 4.09  \\
        38 & 50  &   4.05  & 82  & 4.11  \\
        40 & 48  &   4.04  & 86  & 4.24  \\
        42 & 44  &   3.92  & 86  & 4.12  \\
        44 & 42  &   3.93  & 90  & 4.25  \\
        46 & 38  &   3.82  & 90  & 4.11  \\
        48 & 36  &   3.84  & 88  & 3.85  \\
        50 & 32  &   3.74  & 92  & 3.94  \\
        52 & 28  &   3.65  & 90  & 3.67  \\
        54 & 28  &   3.83  & 100 & 4.11  \\
        56 & 26  &   3.89  & 100 & 3.92  \\
        58 & 24  &   3.96  & 105 & 4.05  \\
        60 & 20  &   3.90  & 110 & 4.17  \\
        62 & 16  &   3.83  & 110 & 3.93  \\
        64 & 14  &   3.94  & 115 & 4.03  \\
        66 & 10  &   3.86  & 110 & 3.44  \\
        68 & 7.6 &   3.94  & 120 & 3.82  \\
        70 & 4.8 &   3.95  & 120 & 3.51  \\
        72 & 2.1 &   3.90  & 120 & 3.20  \\
        73 & 1.8 &   4.06  &     &       \\
        74 & 1.2 &   4.16  & 125 & 3.18  \\
        75 & 0.92&   4.35  &     &       \\
        76 & 1.0 &   4.70* & 135 & 3.51  \\
        77 & 1.4 &   5.21* &     &       \\
        78 & 2.6 &   6.04* & 140 & 3.45  \\
        79 & 4.4 &   7.11* &     &       \\
        80 & 7.0 &   8.53* & 145 & 3.36  \\ 
        82 & 16  &  13.25* & 150 & 3.21  \\     
        84 & 30  &  22.75* & 150 & 2.50* \\    
        86 & 52  &  47.64* & 140 & 1.49* \\    
        88 & 82  & 147.65* & 130 & 1.09* \\    
    \bottomrule
  \end{tabular}
\end{table}

\begin{figure}
  \centering
  \includegraphics{plot.pdf}
  \caption{Plot.}
  \label{fig:plot}
\end{figure}

