\section{Auswertung}
\label{sec:Auswertung}
Zur Auswertung des Versuches werden Mittelwerte gebildet. Diese führen zu einem Standardfehler des Mittelwertes, welcher gemäß
\begin{equation*}
  \label{eqn:MW-Fehler}
  \sigma(x) = \sqrt{\frac{1}{n(n-1)} \sum_i (x_i - \overline{x})^2}.
\end{equation*}
berechnet werden kann. Die Fehlerrechnung der Messwerte genügt der gaußschen Fehlerfortpflanzung und wird in \textit{Python} mittels \textit{Scipy} \cite{scipy} durchgeführt.

\subsection{Bestimmung der Brechungsindizes über die Fresnelschen Formeln}
\label{subsec:Index_Fresnel}
Die Messwerte der gemessenen Ströme des Photoelements sind in \autoref{tab:Messwerte} aufgeführt.
Von den aufgeführten Strömen wird der gemessene Dunkelstrom von $I_\text{D} = \qty{62}{\nano\ampere}$ subtrahiert, auch wenn dies die Ergebnisse kaum beeinflusst, da 
der Dunkelstrom um Größenordnungen unter den gemessenen Strömen liegt.

Es werden die Brechungsindizes $n_p$ und $n_s$ aus den Messreihen zum parallel- und senkrecht polarisierten Licht bestimmt.
Dazu werden die Fresnelschen Formeln für p-polarisiertes Licht \eqref{eqn:Fresnel_parallel} und für s-polarisiertes Licht \eqref{eqn:Fresnel_senkrecht} jeweils auf den 
Brechungsindex umgestellt. Dabei wird ausgenutzt, dass die Intensität des Lichtes proportional zum Quadrat der Amplitude des $\vec{E}$-Feldes ist. 
Daher gilt $\frac{E_\text{r}}{E_\text{e}} = \sqrt{\frac{I_r}{I_0}} = E$. Mit dieser Konvention folgt nach Auflösen der Gleichung \eqref{eqn:Fresnel_parallel} nach $n$
\begin{equation}
  \label{eqn:n_p}
  n_\text{p}(\alpha, E) = \left(\frac{E + 1}{E - 1}\right)^2 \frac{1}{2\cos^2(\alpha)} + \sqrt{\frac{1}{4\cos^2(\alpha)}\left(\frac{E + 1}{E - 1}\right)^4 - \left(\frac{E + 1}{E - 1}\right)^2\tan^2(\alpha)}
\end{equation}
für den Brechungsindex in Abhänigkeit zum Winkel $\alpha$ und dem Quotienten der Intensitäten. 
Nun wird verwendet, dass die Intensität $I_\text{r}, I_0$ auch proportional zur gemessenen Stromsärke ist. Daher gilt $E = \sqrt{\frac{I_\text{p}}{I_0}}$ für p-polarisiertes
Licht, wobei $I_\text{p}$ die gemessene Stromsärke beschreibt. Diese Bedingung gilt gleichermaßen für s-polarisiertes Licht.
Aus den Messwertepaaren der Winkel und Stromstärken $I_\text{p}$ können 
Brechungsindizes $n_\text{p}$ berechnet werden. Diese
werden in \autoref{tab:Messwerte} dargestllt. Der Mittelwert der Brechungsindizes ergibt sich zu $\overline{n_\text{p}} = \num{3.98+-0.15}$, wobei offensichtliche Ausreißer
(hier: $n > \num{4.5}$) verworfen werden.

Die Fresnelsche Formel für s-polarisiertes Licht \eqref{eqn:Fresnel_senkrecht} wird ebenfalls auf den Brechungsindex $n_\text{s}$ umgestellt.
Hierbei ergibt sich 
\begin{equation}
  \label{eqn:n_s}
  n_\text{s}(\alpha, E) = \sqrt{\frac{1 + E^2 + 2E\cos(2\alpha)}{1 - 2E + E^2}}
\end{equation}
für den Brechungsindex bei s-polarisiertem Licht. 
Die durch \autoref{eqn:n_s} berechneten Brechungsindizes werden ebenfalls in \autoref{tab:Messwerte} dargestellt. Der Mittelwert lautet
$\overline{n_\text{s}} = \num{3.85+-0.30}$. Wieder werden Ausreißer (hier: $n < 3$) nicht berücksichtigt.

\begin{table}
  \centering
  \caption{Messwerte der Ströme des Photoelements und daraus resultierende Brechungsindizes. Der Strom des einfallenden Lichtbündels beträgt $I_0 = \qty{180}{\micro\ampere}$, der
  Dunkelstrom $I_\text{D} = \qty{62}{\nano\ampere}$. Die mit '*' markierten Werte werden in der Auswertung dieses Versuches nicht einbezogen.}
  \label{tab:Messwerte}
  \begin{longtable}{S[table-format = 2.0] | c S[table-format = 3.2] | c S[table-format = 1.2]}
      {} & \multicolumn{2}{c}{p-polarisiert} & \multicolumn{2}{c}{s-polarisiert} \\
      \toprule
        {$\alpha \mathbin{/} \unit{\degree}$} & {$I_\text{r} \mathbin{/} \unit{\micro\ampere}$} & {$n_\text{p}(\alpha)$} &%
        {$I_\text{r} \mathbin{/} \unit{\micro\ampere}$} & {$n_\text{p}(\alpha)$} \\
        \midrule
         6 & 66  &   4.09  & 64  & 3.93  \\  
         8 & 66  &   4.11  & 66  & 4.03  \\
        10 & 64  &   4.01  & 62  & 3.79  \\
        12 & 64  &   4.04  & 64  & 3.87  \\
        14 & 66  &   4.19  & 66  & 3.96  \\
        16 & 66  &   4.23  & 68  & 4.04  \\
        18 & 64  &   4.15  & 68  & 4.00  \\
        20 & 62  &   4.07  & 68  & 3.95  \\
        22 & 56  &   3.78  & 62  & 3.58  \\
        24 & 60  &   4.06  & 70  & 3.96  \\
        26 & 56  &   3.89  & 68  & 3.79  \\
        28 & 60  &   4.20  & 76  & 4.19  \\
        30 & 52  &   3.81  & 68  & 3.66  \\
        32 & 52  &   3.88  & 72  & 3.80  \\
        34 & 54  &   4.09  & 80  & 4.18  \\
        36 & 52  &   4.07  & 80  & 4.09  \\
        38 & 50  &   4.05  & 82  & 4.11  \\
        40 & 48  &   4.04  & 86  & 4.24  \\
        42 & 44  &   3.92  & 86  & 4.12  \\
        44 & 42  &   3.93  & 90  & 4.25  \\
        46 & 38  &   3.82  & 90  & 4.11  \\
        48 & 36  &   3.84  & 88  & 3.85  \\
        50 & 32  &   3.74  & 92  & 3.94  \\
        52 & 28  &   3.65  & 90  & 3.67  \\
        54 & 28  &   3.83  & 100 & 4.11  \\
        56 & 26  &   3.89  & 100 & 3.92  \\
        58 & 24  &   3.96  & 105 & 4.05  \\
        60 & 20  &   3.90  & 110 & 4.17  \\
        62 & 16  &   3.83  & 110 & 3.93  \\
        64 & 14  &   3.94  & 115 & 4.03  \\
        66 & 10  &   3.86  & 110 & 3.44  \\
        68 & 7.6 &   3.94  & 120 & 3.82  \\
        70 & 4.8 &   3.95  & 120 & 3.51  \\
        72 & 2.1 &   3.90  & 120 & 3.20  \\
        73 & 1.8 &   4.06  &     &       \\
        74 & 1.2 &   4.16  & 125 & 3.18  \\
        75 & 0.92&   4.35  &     &       \\
        76 & 1.0 &   4.70* & 135 & 3.51  \\
        77 & 1.4 &   5.21* &     &       \\
        78 & 2.6 &   6.04* & 140 & 3.45  \\
        79 & 4.4 &   7.11* &     &       \\
        80 & 7.0 &   8.53* & 145 & 3.36  \\ 
        82 & 16  &  13.25* & 150 & 3.21  \\     
        84 & 30  &  22.75* & 150 & 2.50* \\    
        86 & 52  &  47.64* & 140 & 1.49* \\    
        88 & 82  & 147.65* & 130 & 1.09* \\    
    \bottomrule
  \end{longtable}
\end{table}

\subsection{Bestimmung des Brechungsindizes über den Brewsterwinkel}
\label{subsec:Brewsterwinkel}
Der Brewsterwinkel stellt im Idealfall den Winkel der gemessenen Intensität $I_\text{B} = 0$ dar. Da immer ein gewisser Reststrom gemessen wurde,
wird das Minimum der Messwerte $I_\text{p}$ aus \autoref{tab:Messwerte} festgestellt.
Es folgt der experimentelle Wert des Brewsterwinkels $\alpha_\text{B} = \qty{75}{\degree}$.
Mithilfe von \autoref{eqn:Brewster} kann ein hierraus resultierender Brechungsindex berechnet werden. Dieser ergibt sich zu $n_{\alpha_{\text{p}}} = \num{3.73}$.

\autoref{fig:plot} zeigt den Verlauf der Messwerte und die zugehörigen Theoriekurven, welche durch Einsetzen der Mittelwerte der Brechungsindizes für p- und 
s-polarisiertes Licht in die Gleichungen \eqref{eqn:Fresnel_parallel} und \eqref{eqn:Fresnel_senkrecht} folgen.

\begin{figure}
  \centering
  \includegraphics{plot.pdf}
  \caption{Messpunkte und zugehörige, aus $\overline{n}$ folgende Theoriekurven. Erstellt mit \textit{matplotlib} \cite{matplotlib}.}
  \label{fig:plot}
\end{figure}

