\section{Diskussion}
\label{sec:Diskussion}

In diesem Versuch wurde wurde der Brechungsindex eines Silizium-Spiegels mit einem Aufbau, welcher im Abschnitt \ref{sec:Durchführung} beschrieben wurde, bestimmt. 
Dies ist über drei Methode geschehen. Die errechneten Brechungsindizes wreden in \autoref{tab:Indizes_dis} erneut dargestellt. Außerdem wird ein Literaturwert für einen idealen
Silizium-Spiegel angegeben. Nach Betrachtung der Tabelle fällt auf, dass die berechneten Brechungsindizes alle in der Größenordnung des Literaturwertes liegen, allerdings 
weichen sie jeweils um einen nicht vernachlässigbaren Wert vom Literaturwert ab. Dies liegt vermutlich an dem verwendetem Spiegel. Der Literaturwert gilt lediglich für 
einen perfekten Silizium-Spiegel, was der verwendete Spiegel nicht ist. Durch Verunreinigung, Abnutzung und weitere äußere Effekte liegen keine idealen Bedingung vor, 
weshalb eine Abweichung vom Literaturwert zu erwarten war. Aus diesem Grund wird auf die Berechnung der Abweichungen verzichtet, denn diese träfen keine Aussagen über die
Qualität der Messmethode, sondern demonstrieren lediglich die im imperfekten Bedingung der Messung. Allerdings fließt in die Abweichung der Werte ebenfalls mit ein, dass alle
Einstellungen der Messapparatur händisch vorgenommen wurden, weshalb gleiche Bedingungen bei jeder Messung nicht gewährleistet werden können. 

Wie bereits diskutiert ist der Literaturwert des Brechungsindizes nicht aussagekräftig für die realen Materialien des Versuches. Daher ist anzunehmen, dass die berechneten 
Wert die realen Bedingungen, im Rahmen der allgemeinen Messunsicherheit, relativ gut beschreiben.

\begin{table}
    \centering
    \caption{In dieser Tabelle werden die einzelnen berechneten Brechungsindizes dargestellt.}
    \label{tab:Indizes_dis}
    \begin{tabular}{c c c c}
      \toprule
        {$n_\text{p}$} & {$n_\text{s}$} & {$n_{\alpha_\text{p}}$} & {$n_\text{Lit}$} \\
        \midrule
        $3.98 \pm 0.15$ & $3.85 \pm 0.30$ & 3.73 & 3.35 \\
      \bottomrule
    \end{tabular}
  \end{table}