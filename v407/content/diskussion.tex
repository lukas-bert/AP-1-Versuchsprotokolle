\section{Diskussion}
\label{sec:Diskussion}
Ziel des Versuches war die Bestimmung des Brechungsindizes eines Silizium-Spiegels bei der Wellelänge des verwendeten (roten) Lasers. 
Der gesuchte Brechungsindex wurde aus Messreihen zu parallel- und senkrecht polarisiertem Licht und über Bestimmung des Brewsterwinkels ermittelt.
Die Ergebnisse der errechneten Brechungsindizes werden in \autoref{tab:Indizes_dis} dargestellt. 

\begin{table}
  \centering
  \caption{Experimentell ermittelte Brechungsindizes von Silizium.}
  \label{tab:Indizes_dis}
  \begin{tabular}{c c c}
    \toprule
      {$\overline{n}_\text{p}$} & {$\overline{n}_\text{s}$} & {$n_{\alpha_\text{B}}$} \\
      \midrule
      $3.98 \pm 0.15$ & $3.85 \pm 0.30$ & 3.73 \\
    \bottomrule
  \end{tabular}
\end{table}

Die aus den verschiedenen Methoden resultierenden Brechungsindizes liegen alle in einem relativ kleinen Bereich von $\num{3.7}$ bis ca. $4$. 
Auf Grund der aus dem Standardfehler des Mittelwertes
resultierenden Unsicherheiten stehen die verschiedenen Werte nicht im Widerspruch zueinander, was zeigt, dass die verschiedenen Bestimmungsmethoden
zu gleichwertigen Ergebnissen führen.
Beim Vergleich der Werte mit Literaturwerten des Brechungsindizes von Silizium bei Wellenlängen im roten Bereich, bestätigt sich die Qualität dieser. Auch der Literaturwert 
ist bei Wellenlängen zwischen $\num{600}$ und $\qty{700}{\nano\metre}$ in einem Bereich von ca. $\num{3.7}$ bis $4$ zu verorten.

Ungenauigkeiten bei der Einstellung der Messapparatur und beim Ablesen der Messwerte führen zu den relativ großen Messunsicherheiten der Werte. Beispielsweise war eine exakte
Ausrichtung des Photoelements sehr schwierig umzusetzen. Des Weiteren können sich äußere Einflüsse wie Licht- und Schattenänderungen auf die Messwerte auswirken.
Dies erklärt die in \autoref{fig:plot} sichtbaren Fluktuationen der Messpunkte um die Theoriekurven.

Insgesamt ist das Ziel, die Fresenelschen Formeln experimentell zu Verifizieren und den Brechungsindex von Silizium zu bestimmen, mit einer im Rahmen der Messunsicherheiten
guten Präzision erreicht worden
