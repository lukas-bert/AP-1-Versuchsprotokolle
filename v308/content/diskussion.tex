\section{Diskussion}
\label{sec:Diskussion}
Zuerst werden die Messergebnisse zur langen Spule diskutiert. Der theoretische Wert für die magnetische Flussdichte innerhalb der Spule beträgt 
$B_{\text{Theorie}} = 2.299\:\unit{\milli\tesla}$, der experimentelle Wert wurde zu $B_{\text{exp}} = 2.335\:\unit{\milli\tesla}$ bestimmt. 
Dies entspricht einer relativen Abweichung von $\symup{\Delta_{\text{rel}}} = 1.57 \%$, was in Anbetracht der aufbaubedingten Messungenauigkeiten 
eine sehr geringe Abweichung ist. Auffällig ist, dass das gemessene Magnetfeld im Inneren der Spule nicht homogen verläuft, sondern bis zu einer Eindringtiefe von
ca. $2 \: \unit{\centi\metre}$ noch deutlich ansteigt und auch bis $5 \unit{\centi\metre}$ nicht eindeutig als homogen betrachtet werden kann.
Dies zeigt, dass die theoretischen Näherungen einer "langen"\: Spule nicht vollständig den realen Verhältnissen entsprechen, da die Randeffekte in einem größeren
Bereich zu Abweichungen führen. Eine Ursache dafür ist, dass die verwendete Spule mit einer Länge von etwa $l = 16.4\: \unit{\centi\metre}$ und einem Radius von 
$r = 2.05 \: \unit{\centi\metre}$ nicht den Idealbedingungen einer langen Spule entspricht.


Bei der Messung zum Helmholtzspulenpaar wurden die drei Abstände $d_1 = 10\: \unit{\centi\metre}$, $d_2~=~15\: \unit{\centi\metre}$
und $d_3 = 20\: \unit{\centi\metre}$ betrachtet. Die experimentell bestimmten Werte im Mittelpunkt der Spulen haben die relativen Abweichungen 
$\symup{\Delta_{\text{rel}1}} = 0.29 \%$, $\symup{\Delta_{\text{rel}2}} = 5.31\%$ und $\symup{\Delta_{\text{rel}3}} = 6.35 \%$ zu den Theoriewerten, wobei in allen 
drei Messungen keine experimentellen Werte vorhanden waren, die exakt in der Spulenmitte lagen. Des Weiteren sind die Skalen, mit denen der Spulenabstand und die 
Position der Hallsonde bestimmt wurden um ca. $2.3\: \unit{\centi\metre}$ zueinander verschoben, weshalb der Mittelpunkt der Spulen nicht genau ermittelt werden konnte.
Da alle experimentellen Werte über den Theoriewerten liegen, ist es möglich, dass die Ursache dafür eine zu hohe Stromstärke ist. Diese wurde an einem
analogen Messgerät auf $4 \: \unit{\ampere}$ geregelt, wobei eine Ungenauigkeit von $0.1 \: \unit{\ampere}$ durchaus denkbar ist. Selbiges gilt für die Bestimmung des
Spulenabstandes über die Längenskala. Insgesamt stimmen die Messwerte mit dem Verlauf der Theoriekurven weitgehend überein, was auf die Qualität dieser hindeutet.


Die Hysteresekurve des Kernmaterials der Ringspule lässt sich nicht mit Theoriewerten vergleichen. Jedoch stimmt sie mit dem allgemeinen Verlauf einer Hysteresekurve gut
überein und weist eine hohe Symmetrie auf, was ebenfalls ein Indikator für die Qualität der Messung ist. So weichen beispielsweise die Werte der positiven- und negativen 
Sättigung $+B_S = 709.5 \: \unit{\milli\tesla}$ und $-B_S = -709.6 \: \unit{\milli\tesla}$ nur um $0.1 \unit{\milli\tesla}$ voneinander ab. 


Zusammenfassend lässt sich festellen, dass die Messergebnisse im Rahmen der Messgenauigkeit mit der Theorie übereinstimmen. Die berechneten
Abweichungen liegen trotz der möglichen Fehlerquellen in einem akzeptablen Bereich. 
