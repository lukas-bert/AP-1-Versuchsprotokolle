\section{Auswertung}
\label{sec:Auswertung}
Die Material- und Gerätekonstanten befinden sich im Anhang. Ebenso werden alle Messtabellen nur im Anhang aufgeführt, da die Daten auch den Plots entnommen werden können.
\subsection{Magnetfeld einer langen Spule}
\label{subsec:A_langeSpule}
Die Messwerte dieser Spule sind in \autoref{fig:PlotLangeSpule} grafisch dargestellt. Der Spuleneingang befindet sich in der Grafik bei ca. $0 \: \unit{\centi\metre}$.
Innerhalb der Spule scheinen die Messwerte ein Plateau konstanter Feldstärke zu erreichen, was an der eingezeichneten Theoriegeraden deutlich wird. 
Für den Verlauf des Magnetfeldes außerhalb der Spule, ist ein Abfallen der Feldstärke zu erkennen. Da der letzte Messwert in der Mitte der Spule liegt, sollte dieser auch den geeignetsten experimentellen Wert
für das homogene Magnetfeld innerhalb der Spule liefern. Daher lautet $B_{\text{exp}} = 2.335\:\unit{\milli\tesla}$. 
Für den Theoriewert des Magentfeldes folgt mit \autoref{eqn:LangeSpule} $B_{\text{Theorie}} = 2.299\:\unit{\milli\tesla}$
\begin{figure}
    \centering
    \caption{Die Messwerte der langen Spule. Erstellt mit \textit{matplotlib} \cite{matplotlib}}
    \label{fig:PlotLangeSpule}
    \includegraphics[width=0.85\textwidth]{build/plotLangeSpule.pdf}
\end{figure}
\subsection{Magnetfeld des Helmholtzspulenpaares}
\label{subsec:A_Helmholtz}
Die gemessenen Magnetfelder des Helmholtzspulenpaares sind neben den theoretisch zu erwartenden Kurven, für die drei verwendeten Abstände in den Abbildungen \ref{fig:PlotHH1}, \ref{fig:PlotHH2} und \ref{fig:PlotHH3} 
grafisch dargestellt. Die experimentellen Werte des Magnetfeldes in der Mitte des Spulenpaares lassen sich den Abbildungen entnehmen. Da die Skalen am Helmholtzspulenpaar um
$2.3\:\unit{\centi\metre}$ zueinander versetzt sind, muss der Wert
genommen werden, welcher am nächsten bei $\frac{d}{2} - 2.3\:\unit{\centi\metre}$ liegt. Für den Abstand von $d = 10\:\unit{\centi\metre}$ beträgt der experimentelle Wert
$B_{\text{exp,1}} = 3.840\:\unit{\milli\tesla}$.
Der Theoriewert berechnet sich gemäß Formel \eqref{eqn:Helmholtz} zu $B_{\text{Theorie,1}} = 3.829\:\unit{\milli\tesla}$.
\begin{figure}[H]
    \centering
    \caption{Die Messwerte des Helmholtzspulenpaars zum Abstand d = 10cm. Erstellt mit \textit{matplotlib} \cite{matplotlib}}
    \label{fig:PlotHH1}
    \includegraphics[width=0.85\textwidth]{build/plotHelmHoltz1.pdf}
\end{figure}
Für den Abstand von $d = 15\:\unit{\centi\metre}$ folgt $B_{\text{exp,2}} = 2.222\:\unit{\milli\tesla}$, wie in \autoref{fig:PlotHH2} zu sehen ist. Der dazugehörige Theoriewert lautet $B_{\text{Theorie,2}} = 2.110\:\unit{\milli\tesla}$.
\begin{figure}[H]
    \centering
    \caption{Die Messwerte des Helmholtzspulenpaars zum Abstand d = 15cm. Erstellt mit \textit{matplotlib} \cite{matplotlib}}
    \label{fig:PlotHH2}
    \includegraphics[width=0.85\textwidth]{build/plotHelmHoltz2.pdf}
\end{figure}
Zu $d = 20\:\unit{\centi\metre}$ ergeben sich die Werte $B_{\text{exp,3}} = 1.273\:\unit{\milli\tesla}$ und $B_{\text{Theorie,3}} = 1.197\:\unit{\milli\tesla}$.
\begin{figure}[H]
    \centering
    \caption{Die Messwerte des Helmholtzspulenpaars zum Abstand d = 20cm. Erstellt mit \textit{matplotlib} \cite{matplotlib}}
    \label{fig:PlotHH3}
    \includegraphics[width=0.85\textwidth]{build/plotHelmHoltz3.pdf}
\end{figure}
\subsection{Bestimmung der Hysteresekurve}
\label{A_Hysterese}

\begin{figure}
    \centering
    \caption{Hysteresekurve des Materials in der Ringspule. Erstellt mit \textit{matplotlib} \cite{matplotlib}}
    \label{fig:PlotHysterese}
    \includegraphics[width=0.85\textwidth]{build/plotHysterese.pdf}
\end{figure}

Zunächst wird das gemessene Magnetfeld gegen den Strom $I$ aufgetragen, wie in \autoref{fig:PlotHysterese} zu sehen ist. Für den experimentell bestimmten
Sättigungswert $B_S$ wird der letzte Messpunkt der Neukurve verwendet. Dieser kann \autoref{fig:PlotHysterese} entnommen werden und
liegt im Punkt ($I = 10\:\unit{\ampere}$, $B_s = 696.8\:\unit{\milli\tesla}$).
Die Remenanz $B_r$ enspricht dem $y$-Achsenabschnitt und lautet $ B_r = 124.5 \:\unit{\milli\tesla}$. Auch dies kann der Gafik \ref{fig:PlotHysterese},
beziehungsweise den Messwerten entnommen werden.
\begin{figure}
    \centering
    \caption{Lineare Regression zur Bestimmung der Koerzitivkraft. Erstellt mit \textit{matplotlib} \cite{matplotlib}}
    \label{fig:PlotHystereseFit}
    \includegraphics[width=0.85\textwidth]{build/plotHystereseFit.pdf}
\end{figure}
Die Koerzitivkraft $H_c$ kann nicht direkt aus den Messwerten entnommen werden. Da die Hysteresekurve einen relativ linearen Verlauf um die Stelle des Nulldurchgangs 
aufweist, lässt sich eine lineare Regression der Form
\begin{equation*}
    f(x) = ax + b
\end{equation*}
durchführen. Unter Verwendung der Messwerte zwischen $I = -1 \unit{\ampere}$ und $I = 1 \unit{\ampere}$ ergeben sich die Parameter
\begin{align*}
    a_1 &= 184.8 \: \unit{\milli\tesla\per\ampere}    &   a_2 &= 203.35\: \unit{\milli\tesla}   \\  
    b_1 &= 117.97\: \unit{\milli\tesla\per\ampere}    &   b_2 &= -132.3\: \unit{\milli\tesla}
\end{align*}
für die beiden Geraden an der linken- und rechten Flanke der Hysteresekurve. Daraus lassen sich die Nulldurchgänge der Geraden bestimmen, die der Koerzitivkraft entsprechen. 
Die Stromstärken bei denen diese auftritt lauten $I_1 = -0.638 \: \unit{\ampere}$ und $I_2 = 0.651 \: \unit{\ampere}$.
