\section{Theorie}
\label{sec:Theorie}
Schallwellen sind Druckschwankungen in einem Medium, welche sich gemäß der Wellengleichung
\begin{equation*}
    \label{eqn:Schallwelle}
    p(x,t) = p_0 + v_0 Z \cos \left(\omega t - kx\right)
\end{equation*}
ausbreiten.
Der Frequenzbereich des Schalls von $\qty{20}{\mega\hertz}$ bis $\qty{1}{\giga\hertz}$ befindet sich über dem Hörbaren und wird als \textit{Ultraschall} bezeichnet.
Schall dieses Frequenzbereich kann zur zerstörungsfreien Werkstoffprüfung genutzt werden und findet so unter anderem Anwendung in der Medizin.
