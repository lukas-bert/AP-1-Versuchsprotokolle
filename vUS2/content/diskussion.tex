\section{Diskussion}
\label{sec:Diskussion}
Die Theoriewerte für diesen Versuch wurden händisch mit einem Schieberegler bestimmt. Daher ergibt sich bei allen Theoriewerten eine kleine Unsicherheit, welche 
nicht bestimmt werden kann. Allerdings sollte diese vernachlässigbar klein sein, weshalb man für diesen Versuch die Theoriewerte als Fixwerte annehmen kann. 
Zuerst wurden mittels eines A-Scans die Tiefen aller Fehlstellen bestimmt. Bei diesen Werten fällt auf, dass sie um einen näherungsweise konstanten Wert vom Theoriewert abweichen.
Dies liegt an der Kontaktmittelschicht. Um diesen Fehler auszubessern wurde die Dicke der Kontaktmittelschicht bestimmt. Diese wurde mit einer Ungenauigkeit von $\qty{0.05}{\milli\metre}$
bestimmt. Aus dieser Ungenauigkeit ergeben sich zwar weitere Unsicherheiten nach Korrektur der Kontaktmittelschicht, allerdings ist dieser Wert hinreichend klein um diese Bestimmung als qualitativ anzusehen. 
Dies bestätigt sich nach Berechnung der relativen Abweichungen der Tiefen der Fehlstellen. Diese beträgt lediglich $\Delta l = 0.05\%$. Außerdem wurde mittels des A-Scans das
Auflösungsvermögen unterschiedlicher Sonden untersucht. Hierbei wurde erwartet, dass die Sonde mit höherer Frequenz ein höheres Auflösungsvermögen hat und somit die beiden Peaks 
seperieren kann. Dies hat sich im \autoref{subsec:auflösung} bestätigt. Bei Bestimmung der Fehlstellen durch den B-Scan treten neben die Fehlerquellen des A-Scans noch weitere Fehler auf.
Beim B-Scan muss die Tiefe aus einem 2-D Bild bestimmt werden. Dieses Bild kann allerdings nur leicht verschwommen aufgenommen werden, da der Sensor händisch über den Acrylblock bewegt 
werden muss. Dazu kommt, dass beim Auswerten des Bildes hängt es von subjektiver Abschätzung ab wo der Rand der Fehlstelle liegt. Denn dieser ist durch das verschommene Bild nicht 
eindeutig sichtbar. Daher ergeben sich bei diesem Verfahren relative Abweichungen zwischen $\Delta l = 0.07\%$ und $\Delta l = 20.9\%$. Dies ist eine große Abstand, weshalb 
sich dieses Verfahren zur Bestimmung der Fehlstellen weniger bewährt hat, als der A-Scan. Zuletzt wurde ein B-Scan des Brustmodells angefertigt. Bei diesem liegen dieselben Fehler des B-Scans
vor. Dazu kommt beim Brustmodell noch die schwierigere Messung aufgrund der unebenen Oberfläche. Mit diesem Versuch war es nicht qualitativ möglich das Brustmodell zu untersuchen, 
aber eine ungefähre Aussage über die Position und Größe lässt sich dennoch treffen. Der Versuch hat zu dem Schluss geführt, dass für ein einfaches Objekt, wie der Acrylblock, 
der A-Scan eine höhere Präzision hat. Allerdings ist es bei komplexeren Körpern, wie das Brustmodell, mittels eines A-Scan nicht möglich qualitative Aussagen zu treffen, weshalb man
einen B-Scan benötigt.