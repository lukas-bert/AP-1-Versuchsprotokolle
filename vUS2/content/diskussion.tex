\section{Diskussion}
\label{sec:Diskussion} 
Zuerst wurden mittels eines A-Scans die Tiefen aller Fehlstellen bestimmt. Bei diesen Werten fällt auf, dass sie um einen näherungsweise konstanten Wert vom Theoriewert abweichen.
Dies liegt an der Kontaktmittelschicht. Um diesen Fehler auszubessern wurde die Dicke der Kontaktmittelschicht zu $b_\text{A} = \qty{1.19 +- 0.05}{\milli\metre}$ bestimmt. 
Unter Berücksichtigung dieser Korrektur ergeben sich relative Abweichungen der Tiefen der Fehlstelle, die im Mittel lediglich $\Delta l = 0.05\%$ betragen. 

Im nächsten Teil des Versuches wurde das Auflösungsvermögen verschiedener Sonden untersucht. Hierbei wurde erwartet, dass die Sonde mit höherer Frequenz
ein höheres Auflösungsvermögen bietet und somit die beiden Peaks 
seperieren kann. Dies hat sich im \autoref{subsec:auflösung} bestätigt. 

Bei Bestimmung der Fehlstellen durch den B-Scan treten neben den Fehlerquellen des A-Scans noch weitere Ungenauigkeiten auf.
Beim B-Scan muss die Tiefe aus einem 2D-Bild bestimmt werden, welches jedoch auf Grund der Auflösung verschwommen wirkt. Außerdem führt die manuelle Bewegung der Messsonde
zu Unregelmäßigkeiten. Zusätzlich lässt sich bei der Auswertung der Bilder teilweise der Beginn einer Fehlstelle nur schwer differenzieren, denn dieser ist durch das 
verschommene Bild nicht eindeutig sichtbar. Durch diese Fehlerquellen ergeben sich bei diesem Verfahren relative 
Abweichungen zwischen $\Delta l = 0.07\%$ und $\Delta l = 20.9\%$, weshalb 
sich dieses Verfahren zur Bestimmung der Fehlstellen weniger bewährt hat als der A-Scan. 

Zuletzt wurde ein B-Scan des Brustmodells angefertigt. Bei diesem liegen dieselben Fehler des B-Scans
vor. Darüber hinaus ergibt sich beim Brustmodell eine schwierigere Messung aufgrund der unebenen Oberfläche.
Eine ungefähre Aussage über die Position und Größe der Tumore lässt sich dennoch treffen. 

Der Versuch hat zu dem Schluss geführt, dass für ein einfaches Objekt, wie den Acrylblock, 
der A-Scan eine höhere Präzision liefert. Allerdings ist es bei komplexeren Körpern, wie dem Brustmodell, mittels eines A-Scan nicht möglich qualitative Aussagen zu treffen,
weshalb ein B-Scan benötigt wird.
