\section{Durchführung}
\label{sec:Durchführung}
In diesem Versuch wird ein Acrylblock und ein Brustmodell mittels Ultraschall untersucht. Es genügt aber nicht die Ultraschallsonde auf das zu untersuchende Material zu halten, da sich dann
zwischen der Sonde und dem Material eine Luftschicht befindet. Diese überträgt den Ultraschall nicht genügend an die "Receiver"-Sonde. Dadurch gelingt die Messung nicht. Um dies zu vermeiden
wird eine Anpassungschicht auf das Material aufgetragen. Für den Acrylblock wird destilliertes Wasser verwendet. Diese Wasserschicht nimmt ebenfalls einen Einfluss auf die Messung. Daher werden
zunächst im Rahmen der Vorbereitungsaufgaben einige Materialkonstanten bestimmt.
\subsection{Vorbereitungsaufgaben}
\label{subsec:VBA}
(Ultra-)Schall ist eine Welle, welche sich durch Dichteschwankungen in einem Medium ausbreitet. Da Dichteschwankungen von der Dichte des Ausbreitungsmediums abhängen ist die Schallgeschwindigkeit
eine  Materialkonstante. Für den folgenden Versuch sind drei Schallgeschwindigkeiten relevant. Destilliertes Wasser hat bei $\qty{20}{\degree\celsius}$ eine Schallgeschwindigkeit von $\qty{1483}{\metre\per\second}$.
Die Schallgeschwindigkeit im Acrylblock liegt bei $\qty{2730}{\metre\per\second}$ und die von Luft bei $\qty{20}{\degree\celsius}$ lautet $\qty{344}{\metre\per\second}$.
Es sind in diesem Versuch 2 Messsonden vorhanden. Eine davon sendet Ultraschall mit eine Frequenz von $\qty{1}{\mega\hertz}$ und die andere mit $\qty{2}{\mega\hertz}$. Nun kann mittels Formel
\eqref{"VALLLLLLLLLLLLLLLLLLLLLLLAAAAAAAAAAAAAAAAAAAHHHHHHHHHHHHHHHHH JETZT MACH DIE FORMAL DIKKAGGA"} die Wellenlänge und Periodendauer in Acryl bestimmt werden. Dazu wird die zuvor erwähnte Schallgeschwindigkeit
verwendet. Mit einer Frequenz von $\qty{1}{\mega\hertz}$ ergibt sich dann eine Wellenlänge von $\qty{2.73}{\milli\metre}$ und eine Periodendauer von $\qty{1}{\micro\second}$. Bei $\qty{1}{\mega\hertz}$
liegt die Wellenlänge bei $\qty{1.365}{\milli\metre}$ und die Periodendauer bei $\qty{0.5}{\micro\second}$.
\subsection{Versuchsaufbau}
\label{subsec:versuchsaufbau}
