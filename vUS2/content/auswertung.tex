\section{Auswertung}
\label{sec:Auswertung}
Im folgendem wird die Schallgeschwindigkeit in Acryl als $c_{\text{Acryl}} = \qty{2730}{\metre\per\second}$ und die von destilliertem Wasser als 
$c_{\text{dest.Wasser}} = \qty{1483}{\metre\per\second}$ angenommen. Der verwendete Acrylblock hat eine Gesamthöhe $h_{\text{ges}} = \qty{80.55}{\milli\metre}$. 
Die Fehler wurden gemäß des Standardfehlers des Mittelwertes durch 
\begin{equation}
  \label{eqn:Fehler}
  \sigma_{\overline{\text{x}}} = \frac{\sigma}{\sqrt{n}}
\end{equation}
berechnet.
Alle anderen Fehler wurden gemäß der gaußschen Fehlerfortpflanzung berechnet:
\begin{equation}
  \label{eqn:Gauss}
  \Delta F = \sqrt{\sum_i\left(\frac{\symup{d}F}{\symup{d}y_i}\Delta y_i \right)^2}
\end{equation}

\subsection{Händische Abmessung des Acdrylblocks}
\label{subsec:schieblehre}
In \autoref{tab:handmessung} werden die Positionen und Durchmesser der Fehlestellen im Acrylblock aufgefürt. Dabei ist oben und unten aus \autoref{fig:acrylskizze} zu entnehmen.

\begin{table}
  \centering
  \caption{In dieser Tabelle sind die durch einen Messschieber gemessenen Daten der Fehlstellen aufgeführt. $N$ beschreibt die Lochnummer. $o$ beschreibt die Tiefe der Fehlstelle von oben. $u$ beschreibt die Tiefe der Fehlstelle von unten. $d$ beschreibt den Durchmesser.} 
  \label{tab:handmessung}
  \begin{tabular}{S[table-format = 2.0] S[table-format = 2.2] S[table-format = 2.2] S[table-format = 2.2]}
      \toprule
      {$N$} & {$o(N)$} & {$u(N)$} & {$d$}\\
      \midrule
      1 & 13.35 & 61.30 & 5.90 \\
      2 & 21.80 & 53.75 & 5.00 \\
      3 & 30.60 & 45.95 & 4.00 \\
      4 & 38.80 & 38.85 & 2.90 \\
      5 & 46.75 & 30.90 & 2.90 \\
      6 & 54.80 & 22.85 & 2.90 \\
      7 & 62.80 & 14.85 & 2.90 \\
      8 & 71.00 &  6.65 & 2.90 \\
      9 & 16.10 & 54.65 & 9.80 \\
     10 & 59.40 & 19.70 & 1.45 \\
     11 & 61.20 & 17.90 & 1.45 \\
     \bottomrule
  \end{tabular}   
\end{table}

\subsection{A-Scan zur Bestimmung der Dicke der Kontaktmittelschicht}
\label{subsec:ascankontakt}
Aus diesen Werten kann nun die theoretische Laufzeit des Ultraschallsignals durch den Acrylblock bestimmt werden. Diese wird gemäß der Formel \eqref{eqn:theorielaufzeit} bestimmt.
\begin{equation}
  \label{eqn:theorielaufzeit}
  t_{\text{Theorie}} = 2 \cdot c_{\text{Acryl}} \cdot o(N)
\end{equation}

Die gemessenen Werte der Laufzeit, welche in \autoref{tab:Messwerte} unterscheiden sich aufgrund der Kontaktmittelschicht von der Theorie. Um diesen Fehler auszubessern wird mittels einer Ausgleichsrechnung die Dicke
der Anpassungsschicht bestimmt. Dazu wird zunächst der Betrag der Differenz zwischen den Theoriewerten und denn Messwerten gebildet. Diese werden dann gemittelt, sodass man einen Mittelwert
und einen Mittelwertfehler\eqref{eqn:Fehler} bestimmen kann. Der Mittelwert der Differenzen mit Fehler ergibt sich zu $\overline{t_{\text{diff}}} = \qty{1.60+-0.07}{\micro\second}$.
Aus diesem Mittelwert kann man nun über das Weg-Zeit-Gesetz \eqref{eqn:WegZeit} und der Gaußschen Fehlerfortpflanzung\eqref{eqn:Gauss} die Dicke der Anpassungsschicht $b_{\text{Anpassungsschicht}}$ bestimmen.
Diese beträgt $b_{\text{Anpassungsschicht}} = \qty{1.19+-0.05}{\milli\metre}$.
\subsection{A-Scan zur Bestimmung der genauen Positionen der Fehlstellen}
\label{subsec:ascanpos}
Nun werden die Laufzeiten für alle Störstellen gemessen. Dabei wird der Acrylblock von beiden Seiten untersucht. Die Messwerte werden in \autoref{tab:Messwerte} dargestellt.
Beim Durchmesser wurde die Laufzeitkorrektur bereits vorgenommen.
\begin{table}
  \centering
  \caption{In dieser Tabelle sind die durch einen A-Scan gemessenen Daten der Fehlstellen aufgeführt. $N$ beschreibt die Lochnummer. $o$ beschreibt die Tiefe der Fehlstelle von oben. $u$ beschreibt die Tiefe der Fehlstelle von unten. $d$ beschreibt den Durchmesser.} 
  \label{tab:Messwerte}
  \begin{tabular}{S[table-format = 2.0] S[table-format = 2.2] S[table-format = 2.2] S[table-format = 2.2]}
      \toprule
      {$N$} & {$o(N)$} & {$u(N)$} & {$d$}\\
      \midrule
      1 & 14.8 & 62.3 &  5.83 \\
      2 & 22.9 & 54.9 &  5.13 \\
      3 & 31.6 & 47.3 &  4.03 \\
      4 & 40.1 & 40.1 &  2.73 \\
      5 & 48.0 & 32.1 &  2.83 \\
      6 & 55.9 & 24.2 &  2.83 \\
      7 & 63.8 & 16.3 &  2.83 \\
      8 &      &  8.4 &       \\
      9 & 16.3 & 56.4 & 10.23 \\
     10 & 60.5 & 20.6 &  1.83 \\
     11 & 62.1 & 19.2 &  1.63 \\
     \bottomrule
  \end{tabular}   
\end{table}

Nun wird noch die Laufzeitkorrektur für die einzelnen Messwerte unternommen. Dazu wird von den gemessenen Laufzeiten $2\cdot \frac{c_{\text{dest.Wasser}}}{b_{\text{Anpassungsschicht}}}$ abgezogen.
Die korregierten Laufzeiten werden nun zusammen mit den theoretischen Laufzeiten mittels des Weg-Zeit-Gesetzes \eqref{eqn:WegZeit} in Tiefen umgerechnet und dann in \autoref{fig:grafik} 
gemeinsam dargestellt.

\begin{figure}
  \centering
\includegraphics[width=\textwidth]{build/plot.pdf}
  \caption{In dieser Grafik sind die experimentell bestimmten Abmessungen der Fehlstellen dargestellt.}
  \label{fig:grafik}
\end{figure}

\subsection{Untersuchung des Auflösungsvermögens}
\label{subsec:auflösung}
Um das Auflösungsvermögen zu Untersuchen werden der Fehlerstellen zehn und elf mit Ultraschallsonden unterschiedlicher Frequenzen untersucht.
Wie man in \autoref{fig:einMHz} erkennt, zeigt diese Sonde nur einen Peak für beide Fehlstellen an. Daher kann nicht eindeutig die Tiefe einer der beiden Fehlstellen bestimmt werden.
Mit der 2MHz-Sonde, wie in \autoref{fig:zweiMHz} zu sehen ist, lässt sich der Peak in zwei nah beieinander liegende Peaks trennen. Anhand dieser Frequenz ist es also möglich 
die einzelnen Tiefen der Fehlstellen zu bestimmen.
\begin{figure}
  \centering
\includegraphics[width=0.5\textwidth]{content/einMHz.pdf}
  \caption{In dieser Grafik wird mit einer 1MHz-Sonde ein A-Scan der Fehlstellen 11 und 12 dargestellt.}
  \label{fig:einMHz}
\end{figure}
\begin{figure}
  \centering
\includegraphics[width=0.5\textwidth]{content/2MHz (unten).pdf}
  \caption{In dieser Grafik wird mit einer 2MHz-Sonde ein A-Scan der Fehlstellen 11 und 12 dargestellt.}
  \label{fig:zweiMHz}
\end{figure}