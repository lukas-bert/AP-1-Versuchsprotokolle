\input{../header.tex}

\begin{document}
    
\begin{table}
    \centering
    \caption{Reale Maße der Bohrungen vs. aus B-Scan ermittelte Längen.
    o: Abstand zur Oberkante des Acrylblocks, u: untere Kante}
    \label{tab:B_Scan}
    \begin{tabular}{c S[table-format = 2.2] S S S S S}
      \toprule
      {Loch-Nr.} & {$\symup{o}_\text{real} / \unit{\micro\meter}$}  & {o $ / \unit{\micro\meter}$} & {$\symup{\Delta}_\text{rel}(o)$} &%
      {$\symup{u}_\text{real} / \unit{\micro\meter}$} & {u $/ \unit{\micro\meter}$} & {$\symup{\Delta}_\text{rel}(u)$} \\
      \midrule
       1 & 13.35 & 10.95 & 17.98 & 61.3  & 61.72 &  0.69 \\
       2 & 21.8  & 20.16 &  7.52 & 53.75 & 53.71 &  0.07 \\
       3 & 30.6  & 29.56 &  3.40 & 45.95 & 45.9  &  0.11 \\
       4 & 38.8  & 38.37 &  1.11 & 38.85 & 38.28 &  1.47 \\
       5 & 46.75 & 46.6  &  0.32 & 30.9  & 30.07 &  2.69 \\
       6 & 54.8  & 55.02 &  0.40 & 22.85 & 21.47 &  6.04 \\
       7 & 62.8  & 63.24 &  0.70 & 14.85 & 13.27 & 10.64 \\
       8 & 71.0  &       &       &  6.65 &  5.26 & 20.90 \\
       9 & 16.1  & 13.5  & 16.15 & 54.65 & 55.67 &  1.87 \\
      10 & 59.4  & 60.11 &  1.18 & 19.7  & 16.59 & 15.79 \\
      11 & 61.2  & 62.07 &  1.42 & 17.9  & 16.0  & 10.61 \\
      \bottomrule
    \end{tabular}
\end{table}

\end{document}