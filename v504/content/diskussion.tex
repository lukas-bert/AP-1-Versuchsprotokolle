\section{Diskussion}
\label{sec:Diskussion}
Im ersten Teil des Versuches wurden Kennlinien der Hochvakuumdiode zu verschiedenen Heizströmen aufgenommen. Dabei ergab sich die Schwierigkeit, dass bei höheren
Stromstärken kein konstanter Wert des Sättigungsstromes erreicht werden konnte, weshalb nur ein ungefährer Wert angegeben werden kann. Diese Unsicherheit wirkt 
sich auf die im weiteren Verlauf berechneten Werte aus. 

Bei der Untersuchung des Langmuir-Schottkyschen Raumladungsgesetzes wurde der Exponent der Spannung-Strom Abhängigkeit zu $m = \num{1.429(0.017)}$ bestimmt.
Dies stellt eine relative Abweichung von $\symup{\Delta}_\text{rel}(m) = \qty{4.73}{\percent}$ zum Literaturwert $m_\text{Lit} = 1.5$ dar, welche sich gemäß
der Gleichung
\begin{equation*}
    \symup{\Delta}_\text{rel}(x) = \frac{|x - x^*|}{x^*}
\end{equation*}
berechnet. Dabei sei $x$ ein Messergebnis und $x^*$ der Literaturwert. Eine Ursache für diese Abweichung ist die allgemeine Unsicherheit der Messwerte und die
Tatsache, dass der Bereich des Raumladungsgebiets nicht mit absoluter Genauigkeit von anderen Messwerten differenziert werden kann. Des Weiteren erlaubt der 
Geradenansatz $f(x) = mx + b$, welcher zur Bestimmung des Exponenten $m$ verwendet wurde einen weiteren Freiheitsgrad $b$, der eigentlich durch die Vorfaktoren
des Raumladungsgesetzes festgelegt ist. Unter Beachtung dieser möglichen Fehlerquellen wurde $m$ mit ausreichender Präzision bestimmt.

Im nächsten Teil des Versuches wurde das Anlaufstromgebiet der Kennlinie zu $I_\text{H} = \qty{2.5}{\ampere}$ untersucht. Aus dieser kann die Kathodentemperatur
$T = \qty{2473+-80}{\kelvin}.$ ermittelt werden. Diese lässt sich nicht direkt mit Realwerten vergleichen, jedoch kann die Abweichung zu der aus der Leistungsbilanz
berechneten Kathodentemperatur $T_2 = \qty{2237.5}{\kelvin}$ bestimmt werden. Es ergibt sich eine Abweichung von $\symup{\Delta}_\text{rel} = \qty{10.53}{\percent}$.
Ursachen für diese Diskrepanz der beiden Messwerte sind die zur Berechnung getroffenen Näherungen und Annahmen, sowie die Unsicherheit der Messwerte, die sich
beispielsweise im Sprung der Messwerte in \autoref{fig:Anlaufstrom} verdeutlicht. Der genannte Sprung ist dabei auf einen Skalenwechsel des Messinstrumentes 
zurückzuführen.

Im letzten Teil des Versuches wurde ein experimenteller Wert der Austrittsarbeit ermittelt. Der Mittelwert aller Messreihen lautet 
$\overline{\phi} = \qty{4.59(0.19)}{\electronvolt}$. In der Literatur lässt sich für die Austrittsarbeit der Elektronen in Wolfram der Wert 
$\phi_\text{Lit} = \qty{4.55}{\electronvolt}$ \cite{Ingenieurwissen} finden. Die relative Abweichung des Messwertes ergibt sich zu 
$\symup{\Delta} (\phi) = \qty{0.88}{\percent}$. Im Anbetracht der Tatsache, dass die Messwerte einer gewissen Unsicherheit unterliegen und die reale Austrittsarbeit
der Kathode durch etwaige Verunreinigungen nicht genau dem Literaturwert entsprechen muss, bedarf es keiner Rechtfertigung der Abweichung. 

Insgesamt ist das Ziel des Versuches erfüllt und die experimentelle Bestimmung der geforderten Werte gelungen.
