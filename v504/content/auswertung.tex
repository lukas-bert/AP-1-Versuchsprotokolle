\section{Auswertung}
\label{sec:Auswertung}
Zu Beginn werden die Kennlinien der Hochvakuumdiode ausgewertet. In \autoref{fig:plot1} sind die Kennlinien der ersten $4$ eingestellten Stromstärken zu erkennen. 
Die Originalmessdaten sind der genannten Abbildung beziehungsweise dem Anhang zu entnehmen. Für den Sättgungswert einer Kennlinie wird das Maximum der Messwerte
festgestellt. Die ermittelten Sättigungswerte sind in \autoref{tab:Saettigung} aufgelistet.

\begin{figure}
  \centering
  \includegraphics{plot1.pdf}
  \caption{4 Kennlinien der Hochvakuumdiode. Der ungefähre Wert des Sättigungsstromes ist als gestrichelte Linie eingezeichnet.
           Erstellt mit \textit{matplotlib} \cite{matplotlib}.}
  \label{fig:plot1}
\end{figure}

Die Kennlinie zur maximalen Stromstärke $I = \qty{2.5}{\ampere}$ ist in \autoref{fig:plot2} abgebildet. Es fällt auf, dass der Graph sein Krümmungsverhalten nicht ändert, 
da der Sättigungsbereich eigentlich noch nicht erreicht wird. Dennoch wird auch hier die maximal gemessene Stromstärke als bestmöglicher Wert angenommen.

\begin{figure}[H]
  \centering
  \includegraphics{plot2.pdf}
  \caption{Aufgezeichnete Kennlinie zur maximalen Stromstärke $I = \qty{2.5}{\ampere}$. Es ist zu erkennen, dass der Sättigungsbereich noch nicht erreicht wurde.}
  \label{fig:plot2}
\end{figure}

\begin{table}[H]
  \centering
  \caption{Aus den Messwerten abgelesene Werte des Sättigungsstroms.}
  \label{tab:Saettigung}
  \begin{tabular}{l S[table-format = 1.3]}
    \toprule
      {Heizstrom} & {$I_\text{S} \mathbin{/} \unit{\milli\ampere}$} \\
      \midrule
      {$ I = \qty{1.9}{\ampere}$} & 0.042 \\
      {$ I = \qty{2.0}{\ampere}$} & 0.116 \\
      {$ I = \qty{2.1}{\ampere}$} & 0.243 \\
      {$ I = \qty{2.2}{\ampere}$} & 0.562 \\
      {$ I = \qty{2.5}{\ampere}$} & 1.391 \\
    \bottomrule
  \end{tabular}
\end{table}

\subsection{Gültigkeitsbereich des Langmuir-Schottkyschen Raumladungsgesetzes}
\label{subsec:A_Raumladung}
Anhand \autoref{eqn:Raumladung} lässt sich erkennen, dass für den Strom $I$ im Raumladungsgebiet ein Zusammenhang der Form $I = \tilde{b} \cdot U^m$ gilt. Durch Anwenden des 
Logarithmus auf beiden Seiten der Gleichung ergibt sich die Geradengleichung
\begin{align}
  \label{eqn:Raumgerade}
  \mathrm{log}(I) &= \mathrm{log}\left(\tilde{b} \cdot U^m \right) \nonumber \\
  \Leftrightarrow \mathrm{log}(I) &= m \cdot \mathrm{log}(U) + b   \nonumber \\ 
  \Leftrightarrow f(x) &= mx + b
\end{align}
die einen Zusammenhang zwischen den Logarithmen der Spannung ($x$) und des Stroms ($f(x)$) beschreibt. In \autoref{fig:Raumladung} sind diese Logarithmen auf den Achsen der Grafik aufgetragen,
da die Messwerte in dieser Darstellung annähernd eine Gerade modellieren ist anzunehmen, dass alle aufgenommenen Messpunkte in den Gültigkeitsbereich des Raumladungsgesetzes
fallen.

\begin{figure}
  \centering
  \includegraphics{Raumladung.pdf}
  \caption{Raumladungsbereich der $I = \qty{2.5}{\ampere}$ Kennlinie (logarithmisiert) und lineare Ausgleichsgeraden.}
  \label{fig:Raumladung}
\end{figure}

Mithilfe einer linearen Regression mittels \textit{scipy} \cite{scipy} werden die Parameter der Geradengleichung \eqref{eqn:Raumgerade} ermittelt. 
Es ergeben sich die Werte
\begin{align*}
  \label{eqn:Parameter1}
  m &= \num{1.429(0.017)} & b &= \num{-6.80(0.06)} \\
\end{align*}
für die Parameter der Regression. Die Größe $b$ ist nicht von Interesse. Für den Exponenten $m$ kann der Theoriewert aus \autoref{eqn:Raumladung} entnommen werden. Er lautet
$m_\text{theo} = \frac{3}{2}$.

\subsection{Untersuchung des Anlaufstromgebietes}
\label{subsec:A_Anlaufstrom}
An \autoref{eqn:j_Anlauf} kann abgelesen werden, dass im Bereich des Anlaufstroms ein exponentieller Zusammenhang zwischen Strom und Spannung besteht. Wie im vorherigen
Abschnitt, kann durch Anwenden des Logarithmus ein linearer Zusammenhang zwischen $\mathrm{log}(I \mathbin{/} \unit{\ampere})$ und der Gegenspannung $U_\text{G}$ erzeugt werden. 
Es kann wieder die Geradengleichung \eqref{eqn:Raumgerade} verwendet werden, um einen experimentellen Wert des Exponenten $m$ zu bestimmen.
Mit dem Exponenten aus \autoref{eqn:j_Anlauf} gilt
\begin{equation}
  \label{eqn:m}
  m = -\frac{e}{kT},
\end{equation}
woraus sich die Kathodentemperatur $T$ berechnen lässt.
In \autoref{fig:Anlaufstrom} wird der Logarithmus der gemessenen Stromstärke $I$ in Abhängigkeit zur Gegenspannung $U_\text{G}$ dargestellt.

\begin{figure}
  \centering
  \includegraphics{build/Anlaufstrom.pdf}
  \caption{Anlaufstromgebiet mit logarithmischer Darstellung der Stromstärken zur Bestimmung einer linearen Ausgleichsgeraden.}
  \label{fig:Anlaufstrom}
\end{figure}

Wie oben wird eine lineare Regression mit \textit{scipy} \cite{scipy} durchgeführt, die die Parameter 
\begin{align*}
  \label{eqn:Parameter2}
  m &= \qty{-4.69(0.15)}{\per\volt} & b &= \num{-18.41(0.09)} \\
\end{align*}
ergibt. Durch Auflösen des Zusammenhangs \eqref{eqn:m} nach der Temperatur folgt mit der Elementarladung $e = \qty{1.602e-19}{\coulomb}$ \cite{scipy} für die Kathodentemperatur
\begin{equation}
  \label{eqn:Temperatur}
  T = -\frac{e}{k \cdot m} = \qty{2473+-80}{\kelvin}.
\end{equation}

\subsection{Bestimmung der Kathodentemperaturen und der Austrittsarbeit von Wolfram}
\label{subsec:Austrittsarbeit}
Mit den in \autoref{tab:Saettigung} aufgeführten Sättgungsströmen kann die Austrittsarbeit des verwendeten Kathodenmaterials bestimmt werden, wenn die Kathodentemperatur
bekannt ist. Die Temperatur der Kathoden kann aus der Leistungsbilanz des Spannungsgeräts mithilfe der \autoref{eqn:Leistung} bestimmt werden. 
Die Wärmeleistung der verwendeten Kathode beträgt $N_\text{WL} = \qty{0.95}{\watt}$, die emittierende Fläche ist $f = \qty{0.32}{\square\centi\metre}$ und der
Emissionsgrad lautet $\eta = 0.28$.
Durch Auflösen nach T folgt aus \eqref{eqn:Leistung}
\begin{equation*}
T = \left(\frac{I\cdot U - N_\text{WL}}{f \eta \sigma}\right)^{\frac{1}{4}},
\end{equation*}
woraus die Temperaturen der Kathode zu den verschiedenen Leistungen ($U \cdot I$) bestimmt werden kann. Die entsprechenden Werte sind in \autoref{tab:Temperaturen}
dargestellt.

\begin{table}
  \centering
  \caption{Eingestellte Spannungen/Stromstärken und daraus resultierende Temperaturen der Kathode.}
  \label{tab:Temperaturen}
  \begin{tabular}{S[table-format = 1.1] S S[table-format = 4.1]}
    \toprule
      {$I_\text{H} \mathbin{/} \unit{\ampere}$} & {$U_\text{H} \mathbin{/} \unit{\volt}$} & {$T \mathbin{/} \unit{\kelvin}$} \\
      \midrule
      1.9 & 3.2 & 1780.3 \\
      2.0 & 3.5 & 1855.2 \\
      2.1 & 4.0 & 1954.3 \\
      2.2 & 4.3 & 2020.4 \\
      2.5 & 5.5 & 2237.5 \\
    \bottomrule
  \end{tabular}
\end{table}

Mit den Sättgungsströmen aus \autoref{tab:Saettigung} und den Temperaturen aus \autoref{tab:Temperaturen} kann die Austrittsarbeit $e \cdot \phi$ für das verwendete
Material -Wolfram- bestimmt werden. Durch Einsetzen von $j_\text{S} = \sfrac{I_\text{S}}{f}$ und Umstellen auf $\phi$ folgt aus \autoref{eqn:J_S}
\begin{equation*}
  \phi = -\frac{kT}{e} \cdot \mathrm{log}\left(\frac{I_\text{S} \cdot h^3}{4\symup{\pi}f \cdot e  m_0 k^2 T^2}\right)
\end{equation*}
für die Austrittsarbeit $\phi$ in Elektronenvolt. $m_0$ ist die Ruhemasse des Elektrons, $h$ das Plancksche Wirkungsquantum und $k$ die Boltzmannkonstante.
Mit den in Sättigungsströmen und Kathodentemperaturen der verschiedenen Messreihen folgen fünf experimentelle Werte der Austrittsarbeit von Wolfram. 
Die Ergebnisse sind in \autoref{tab:Austrittsarbeit} aufgelistet.

\begin{table}
  \centering
  \caption{Experimentell ermittelte Austrittsarbeiten von Wolfram zu den verschiedenen Messreihen.}
  \label{tab:Austrittsarbeit}
  \begin{tabular}{S[table-format = 1.1] S[table-format = 1.2] }
    \toprule
      {$I_\text{H} \mathbin{/} \unit{\ampere}$} & {$\phi \mathbin{/} \unit{\electronvolt}$} \\
      \midrule
      1.9 & 4.40 \\
      2.0 & 4.44 \\
      2.1 & 4.57 \\
      2.2 & 4.59 \\
      2.5 & 4.95 \\
    \bottomrule
  \end{tabular}
\end{table}

Aus diesen Ergebnissen wird ein Mittelwert gebildet. Der Standardfehler des Mittelwertes ergibt sich nach 
\begin{equation*}
  \sigma_{\overline{x}} = \sqrt{\frac{1}{n(n-1)} \sum_i \left(x_i - \overline{x}\right)^2}.
\end{equation*}
Der Mittelwert ergibt sich zu $\overline{\phi} = \qty{4.59(0.19)}{\electronvolt}$.
