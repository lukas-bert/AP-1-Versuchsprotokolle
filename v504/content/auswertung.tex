\section{Auswertung}
\label{sec:Auswertung}

\begin{figure}
  \centering
  \includegraphics{plot1.pdf}
  \caption{4 Kennlinien der Hochvakuumdiode. Der ungefähre Wert des Sättigungsstromes ist als gestrichelte Linie eingezeichnet.
           Erstellt mit \textit{matplotlib} \cite{matplotlib}.}
  \label{fig:plot}
\end{figure}


\begin{figure}
  \centering
  \includegraphics{plot2.pdf}
  \caption{Aufgezeichnete Kennlinie zur maximalen Stromstärke $I = \qty{2.5}{\ampere}$. Es ist zu erkennen, dass der Sättigungsbereich noch nicht erreicht wurde.}
  \label{fig:plot2}
\end{figure}

\begin{figure}
  \centering
  \includegraphics{Raumladung.pdf}
  \caption{Kennlinie mit Potenz $I^{2/3}$ zur Darstellung des Raumladungsgebiets (links). Messwerte und Ausgleichsfunktion (rechts).}
  \label{fig:Raumladung}
\end{figure}

\begin{figure}
  \centering
  \includegraphics{Anlaufstrom.pdf}
  \caption{Anlaufstromgebiet mit logarithmischer Darstellung der Stromstärken zur Bestimmung einer linearen Ausgleichsgeraden.}
  \label{fig:Anlaufstrom}
\end{figure}
